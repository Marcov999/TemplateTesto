\documentclass{article}
\usepackage{mstyle}
\usepackage{pgfplots}
\usetikzlibrary{intersections, pgfplots.fillbetween}

\title{Teoremi di rigidità per funzioni olomorfe nel disco}
\date{}
\author{Candidato: Marco Vergamini \qquad Relatore: Prof. Marco Abate}

\begin{document}
\maketitle
\newpage
\tableofcontents
\newpage


\section*{Introduzione}
\addcontentsline{toc}{section}{Introduzione}
L'obiettivo di questo scritto è dimostrare un teorema del 1994, il teorema di Burns-Krantz (Theorem 2.1 di \cite{BK}), attraverso risultati elementari. L'enunciato del teorema riguarda le funzioni olomorfe sul disco unitario con un certo andamento vicino al bordo, ed è un risultato di rigidità: se la funzione dista dall'identità al più per un $o\bigl((z-\sigma)^3\bigr)$, allora è proprio l'identità.

La dimostrazione originale del teorema non è lunga, ma un po' tecnica. In un recente articolo di Bracci, Kraus e Roth (\cite{BKR}) si trova una dimostrazione alternativa del teorema di Burns-Krantz. Come spiegato nel Remark 2.2 dell'articolo, è possibile passare dalle ipotesi del teorema di Burns-Krantz a quelle del Theorem 2.1 di \cite{BKR} (viene dimostrato nella Proposition 8.1 dello stesso articolo), che è un risultato più forte dal quale poi è facile concludere. Il Theorem 2.1 è sostanzialmente una versione al bordo del caso di unicità del lemma di Schwarz-Pick, sempre visto come un risultato di rigidità. \\

Bracci, Kraus e Roth dimostrano il Theorem 2.1 usando risultati più generali visti nell'articolo, ma complicati. Tuttavia, nel Remark 5.6 danno una traccia per una dimostrazione più elementare. L'idea è sfruttare una disuguaglianza dovuta a Golusin e vengono indicati vari articoli in cui è stata ridimostrata.

In particolare, l'articolo di Beardon e Minda del 2004 (\cite{BM}) contiene una serie di disuguaglianze di facile dimostrazione, delle quali il Corollary 3.7 ha a sua volta come corollario la disuguaglianza di Golusin. Queste disuguaglianze coinvolgono la distanza di Poincaré sul disco unitario e possono essere applicate per ottenere diversi altri risultati per funzioni olomorfe sul disco, come mostrato nell'articolo. \\

In questo scritto svilupperemo la traccia data nel Remark 5.6 di \cite{BKR}. Dimostreremo le disuguaglianze in \cite{BM} e vedremo alcune applicazioni, tra le quali anche la disuguaglianza di Golusin. Grazie ad essa, e alla Proposition 8.1 di \cite{BKR}, otterremo una dimostrazione elementare del Theorem 2.1 di Burns-Krantz e del risultato più generale dovuto a Bracci-Kraus-Roth.


\newpage

\section{Prerequisiti}

\subsection{Lemma di Schwarz-Pick e distanza di Poincaré}
\begin{defn}
  Sia $\Omega \subset \mathbb{C}$ un aperto. Una funzione $f:\Omega \longrightarrow \mathbb{C}$ si dice \textit{olomorfa} in $\Omega$ se è derivabile in senso complesso per ogni $z \in \Omega$, e scriviamo $f \in \mathcal{O}(\Omega)$. Se $\Ima(f) \subset \Omega'$ scriviamo $f \in \text{Hol}(\Omega, \Omega')$.
\end{defn}

\begin{defn}
  Se $f \in \text{Hol}(\Omega, \Omega)$ è biettiva, allora si può dimostrare che anche $f^{-1}$ è olomorfa. In tal caso $f$ è detta \textit{automorfismo} (in senso olomorfo) di $\Omega$ e scriviamo $f \in \text{Aut}(\Omega)$.
\end{defn}

Com'è noto, la condizione di olomorfia per funzioni a valori complessi è molto più forte della derivabilità in senso reale (in particolare, è equivalente all'analiticità). Fra i vari risultati noti per le funzioni olomorfe, ci interessa studiare il lemma di Schwarz-Pick, fino a dimostrarne una versione al bordo.

Notazione: indichiamo il disco unitario con $\mathbb{D}:=\{z \in \mathbb{C} \mid |z|<1\}$. Riportiamo ora alcuni risultati noti di analisi complessa che verranno usati nelle dimostrazioni.

\begin{thm}
  (formula integrale di Cauchy, \cite[Chapter 1.3, Theorems 9 and 10]{NN}) Sia $f \in \mathcal{O}(\Omega)$ e $D$ un disco chiuso di centro $a$ contenuto in $\Omega$. Allora per ogni $n \in \mathbb{N}$ si ha
  \begin{equation}
    f^{(n)}(a)=\frac{n!}{2\pi i} \int_{\partial D} \frac{f(\zeta)}{(\zeta-a)^{n+1}}\diff\zeta.
  \end{equation}
\end{thm}

\begin{prop} \label{estensione}
  (teorema di estensione di Riemann, \cite[Chapter 1.5, Theorem 2]{NN}) Sia $f \in \mathcal{O}(\Omega \setminus\{z_0\})$ con $z_0 \in \Omega$. Allora $f$ si estende a una $g \in \mathcal{O}(\Omega)$ se e solo se è limitata in un intorno di $z_0$. In tal caso, $z_0$ è detta \normalfont{singolarità rimovibile}.
\end{prop}

\begin{prop}
  (principio del massimo per funzioni olomorfe, \cite[Chapter 1.3, Corollary of Theorem 3 and Theorem 5]{NN}) Sia $\Omega \subset \mathbb{C}$ un aperto e sia $f \in \mathcal{O}(\Omega)$. Sia inoltre $U$ un aperto relativamente compatto in $\Omega$, cioè $\overline{U} \subset \Omega$ e $\overline{U}$ compatto. Allora per ogni $z \in U$ si ha
  $$|f(z)| \le \sup_{w \in \partial U} |f(w)|$$
  e vale l'uguale per qualche $z \in U$ solo se $f$ è costante sulla componente connessa di $U$ contenente $z$.
\end{prop}

Vediamo adesso i lemmi di Schwarz e Schwarz-Pick.

\begin{lm}
  (lemma di Schwarz) Sia $f \in \text{\normalfont{Hol}}(\mathbb{D},\mathbb{D})$ tale che $f(0)=0$. Allora per ogni $z \in \mathbb{D}$ si ha $|f(z)| \le |z|$ e $|f'(0)| \le 1$; inoltre, se vale l'uguaglianza nella prima per $z_0 \not=0$ oppure nella seconda allora $f(z)=e^{i\theta}z$ per qualche $\theta \in \mathbb{R}$.
\end{lm}

\begin{proof}
  Poiché $f(0)=0$, possiamo costruire $g \in \mathcal{O}(\mathbb{D})$ con $g(z)=\frac{f(z)}{z}$ estendendola per continuità in $0$ come $g(0)=f'(0)$. Fissiamo $0<r<1$.
  Per ogni $z \in \mathbb{D}$ tale che $|z| \le r$, per il principio del massimo per funzioni olomorfe si ha
  $$|g(z)| \le \max_{|w|=r} |g(w)|=\max_{|w|=r} \frac{|f(w)|}{r} \le \frac{1}{r}.$$
  Mandando $r$ a $1$ otteniamo che per ogni $z \in \mathbb{D}$ si ha $|g(z)| \le 1$, da cui $|f(z)|\le |z|$ e $|f'(0)| \le 1$.

  Se vale una delle due ugaglianze, allora esiste $z_0 \in \mathbb{D}$ tale che $|g(z_0)|=1$. Dunque, sempre per il principio del massimo $g$ è costantemente uguale a un valore di modulo $1$ in ogni disco di centro l'origine e raggio $|z_0|<r<1$, quindi su $\mathbb{D}$. Perciò $g(z)=e^{i\theta}$ con $\theta \in \mathbb{R}$, da cui $f(z)=e^{i\theta}z$.
\end{proof}

\begin{cor} \label{cor_schw}
  Se $f \in \text{\normalfont{Aut}}(\mathbb{D})$ è tale che $f(0)=0$, allora $f(z)=e^{i\theta}z$.
\end{cor}

\begin{proof}
  Se $f \in \text{Aut}(\mathbb{D})$ anche $f^{-1} \in \text{Aut}(\mathbb{D})$; inoltre $(f^{-1})'(0)=\dfrac{1}{f'(0)}$. Per il lemma di Schwarz, $|f'(0)| \le 1$ e $|(f^{-1})'(0)| \le 1$; dunque $|f'(0)|=1$, da cui la tesi sempre per il lemma di Schwarz.
\end{proof}

\begin{defn}
  Diciamo che un gruppo $G$ \textit{agisce fedelmente} su uno spazio $X$ se per ogni $g \in G$ è data una bigezione $\gamma_g:X \longrightarrow X$ tale che $\gamma_{e}=\id$ e $\gamma_{g_1} \circ \gamma_{g_2} =\gamma_{g_1g_2}$, inoltre $\gamma_{g_1}=\gamma_{g_2}$ se e solo se $g_1=g_2$.

  Chiamiamo inoltre \textit{gruppo di isotropia} di $x_0 \in X$ il sottogruppo di $G$ dato da $G_{x_0}=\{g \in G \mid \gamma_g(x_0)=x_0\}$.
\end{defn}

\begin{lm} \label{az_gr}
  Sia $G$ un gruppo che agisce fedelmente su uno spazio $X$ e sia $G_{x_0}$ il gruppo di isotropia di $x_0 \in X$. Supponiamo che per ogni $x \in X$ esista $g_x \in G$ tale che $\gamma_{g_x}(x)=x_0$ e sia $\Gamma=\{g_x \mid x \in X\}$.
  Allora $G$ è generato da $\Gamma$ e $G_{x_0}$, cioè ogni $g \in G$ è della forma $g=hg_x$ con $x \in X$ e $h \in G_{x_0}$.
\end{lm}

\begin{proof}
  Sia $g \in G$ e $x=\gamma_{g^{-1}}(x_0)$. Allora $(\gamma_{g_x}\circ \gamma_{g^{-1}})(x_0)=x_0$ da cui $\gamma_{g_x}\circ \gamma_{g^{-1}}=\gamma_{g_xg^{-1}}=\gamma_h$ con $h \in G_{x_0}$,
  dunque $g_xg^{-1}=h$ e quindi $g=h^{-1}g_x$ con $h^{-1} \in G_{x_0}$.
\end{proof}

\begin{prop} \label{aut}
  Si ha che $f \in \text{\normalfont{Aut}}(\mathbb{D})$ se e solo se esistono $\theta \in \mathbb{R}$ e $a \in \mathbb{D}$ tali che $f(z)=e^{i\theta}\dfrac{z-a}{1-\bar{a}z}$.
\end{prop}

\begin{proof}
  ($\Leftarrow$) Sia $f$ come nell'enunciato. Con semplici conti possiamo vedere che per $z,w \in \mathbb{C}$ con $\bar{w}z\not=1$ si ha
  \begin{equation} \label{formuletta}
    1-\left|\frac{z-w}{1-\bar{w}z}\right|^2=\frac{(1-|w|^2)(1-|z|^2)}{|1-\bar{w}z|^2},
  \end{equation}
  da cui segue che se $a, z \in \mathbb{D}$ allora
  $$1-|f(z)|^2=\frac{(1-|a|^2)(1-|z|^2)}{|1-\bar{a}z|^2}>0,$$
  per cui $|f(z)|<1$, cioè $f(z) \in \mathbb{D}$. L'inversa è $f^{-1}(z)=e^{-i\theta}\dfrac{z+ae^{i\theta}}{z+\bar{a}e^{-i\theta}z}$, della stessa forma. Si noti che $f(a)=0$.

  ($\implies$) Scriviamo per semplicità $f_{a, \theta}(z)=e^{i\theta}\dfrac{z-a}{1-\bar{a}z}$. Vediamo $\text{Aut}(\mathbb{D})$ come gruppo che agisce su $\mathbb{D}$. $\text{Aut}(\mathbb{D})_0$ è, per il Corollario \ref{cor_schw}, $\{f_{0, \theta} \mid \theta \in \mathbb{R}\}$,
  mentre possiamo prendere $\Gamma=\{f_{a, 0} \mid a \in \mathbb{D}\}$ poiché $f_{a, 0}(a)=0$.
  Per il Lemma \ref{az_gr}, $\text{Aut}(\mathbb{D})$ è generato da $\text{Aut}(\mathbb{D})_0$ e $\Gamma$, cioè ogni $\gamma \in \text{Aut}(\mathbb{D})$ è della forma $\gamma=f_{0, \theta} \circ f_{a, 0}=f_{a, \theta}$.
\end{proof}

\begin{oss} \label{dom}
  Dalla dimostrazione abbiamo anche che $f(\partial \mathbb{D}) \subset \partial \mathbb{D}$; inoltre, per $|z|>1$ con $z \not=1/\bar{a}$ si ha $|f(z)|>1$.
\end{oss}

\begin{oss} \label{transi}
  $\text{Aut}(\mathbb{D})$ agisce in modo transitivo su $\mathbb{D}$, cioè si ha che per ogni $z_0, z_1 \in \mathbb{D}$ esiste $\gamma \in \text{Aut}(\mathbb{D})$ tale che $\gamma(z_0)=z_1$. Infatti, basta prendere $\gamma=f_{z_1, 0}^{-1} \circ f_{z_0, 0}$.
\end{oss}

\begin{lm} \label{SP}
  (lemma di Schwarz-Pick) Sia $f \in \text{\normalfont{Hol}}(\mathbb{D},\mathbb{D})$.
  Allora per ogni $z, w \in \mathbb{D}$ si ha
  $$\left|\frac{f(z)-f(w)}{1-\overline{f(w)}f(z)}\right| \le \left|\frac{z-w}{1-\bar{w}z}\right| \text{ e } \frac{|f'(z)|}{1-|f(z)|^2} \le \frac{1}{1-|z|^2}.$$
  Inoltre se vale l'uguaglianza nella prima per $z_0, w_0$ con $z_0 \not=w_0$ o nella seconda per $z_0$ allora $f \in \text{\normalfont{Aut}}(\mathbb{D})$ e vale sempre l'uguaglianza.
\end{lm}

\begin{proof}
  Fissato $w \in \mathbb{D}$ siano $\gamma_1(z)=\dfrac{z+w}{1+\bar{w}z}$ e $\gamma_2(z)=\dfrac{z-f(w)}{1-\overline{f(w)}z}$. Si ha $\gamma_1, \gamma_2 \in \text{Aut}(\mathbb{D})$. Si ha anche che $\gamma_1(0)=w$ e $\gamma_2\bigl(f(w)\bigr)=0$; inoltre $\gamma_1^{-1}(z)=\dfrac{z-w}{1-\bar{w}z}$.
  Per il lemma di Schwarz applicato a $\gamma_2 \circ f \circ \gamma_1$ abbiamo che per ogni $\zeta \in \mathbb{D}$ si ha $|(\gamma_2 \circ f \circ \gamma_1)(\zeta)| \le |\zeta|$; prendendo $\zeta=\gamma_1^{-1}(z)$ otteniamo che per ogni $z \in \mathbb{D}$ si ha $|(\gamma_2 \circ f)(z)| \le |\gamma_1^{-1}(z)|$, che è la prima disuguaglianza.
  Abbiamo poi $|(\gamma_2 \circ f \circ \gamma_1)'(0)| \le 1$, da cui $|\gamma_2'\bigl(f(w)\bigr)f'(w)\gamma_1'(0)| \le 1$. Valgono le seguenti uguaglianze:
  \begin{gather*}
    \gamma_1'(z)=\frac{1+\bar{w}z-\bar{w}(z+w)}{(1+\bar{w}z)^2} \implies \gamma_1'(0)=1-|w|^2, \\
    \gamma_2'(z)=\frac{1-\overline{f(w)}z+\overline{f(w)}\bigl(z-f(w)\bigr)}{\bigl(1-\overline{f(w)}z\bigr)^2} \implies \gamma_2'\bigl(f(w)\bigr)=\dfrac{1}{1-|f(w)|^2}.
  \end{gather*}
  Sostituendo si ottiene la seconda disuguaglianza con $w$ al posto di $z$.

  Per l'uguaglianza, nel primo caso avremmo $|(\gamma_2 \circ f \circ \gamma_1)(\zeta)|=|\zeta|$, mentre nel secondo $|(\gamma_2\circ f\circ\gamma_1)'(0)|=1$. In entrambi i casi, per il lemma di Schwarz $\gamma_2 \circ f \circ \gamma_1=g \in \text{Aut}(\mathbb{D})$, dunque $f=\gamma_2^{-1}\circ g \circ \gamma_1^{-1} \in \text{Aut}(\mathbb{D})$.
\end{proof}

\begin{defn}
  Scriviamo $[z,w]:=f_{w,0}(z)$ e $p(z,w):=|[z,w]|$.
\end{defn}

\begin{oss} \label{muu}
  Se $f$ è un automorfismo del disco, esiste $\mu \in \partial \mathbb{D}$ tale che $[f(z),f(w)]=\mu[z,w]$. Infatti, la funzione $g(\zeta)=[\zeta,f(w)]$ sta in $\text{Aut}(\mathbb{D})$ e $[f(z),f(w)]=g\bigl(f(z)\bigr)$ è ancora un automorfismo. Si ha inoltre $[f(w),f(w)]=0$, dunque dev'essere proprio della forma $\mu[z,w]$ con $|\mu|=1$.
\end{oss}

Dal lemma di Schwarz-Pick abbiamo che la quantità $p(z,w)$ è contratta da $f \in \text{Hol}(\mathbb{D}, \mathbb{D})$. Vediamo adesso una distanza costruita a partire da questa quantità, con la quale dimostreremo una serie di disuguaglianze che ci permetteranno di dimostrare la disuguaglianza di Golusin, dalla quale seguirà la versione al bordo del lemma.

 Consideriamo $\omega(z,w):=\text{arctanh}\bigl(p(z,w)\bigr)=\dfrac{1}{2}\log\left(\dfrac{1+p(z,w)}{1-p(z,w)}\right)$.

\begin{prop} \label{eunadistanza}
  La funzione $\omega: \mathbb{D}\times \mathbb{D} \longrightarrow [0,+\infty)$ è ben definita ed è effettivamente una distanza.
\end{prop}

\begin{proof}
  Notiamo che per $z,w \in \mathbb{D}$ l'equazione \eqref{formuletta} ci dà immediatamente $p(z,w)<1$, per cui $\omega$ è ben definita e resta solo da mostrare che è una distanza.

  L'unica cosa non ovvia da dimostrare è la disuguaglianza triangolare. Applicando la tangente iperbolica a entrambi i membri della disuguaglianza triangolare per $\omega$ e sfruttando l'uguaglianza $\text{tanh}\,(a+b)=\frac{\text{tanh}\,(a)+\text{tanh}\,(b)}{1+\text{tanh}\,(a)\,\text{tanh}\,(b)}$ si ha
  \begin{align*}
    \text{tanh}\bigl(\omega(z_1,z_2)\bigr) & \le \text{tanh}\bigl(\omega(z_1, z_0)+\omega(z_0,z_2)\bigr) \\
    &=\frac{\text{tanh}\bigl(\omega(z_1, z_0)\bigr)+\text{tanh}\bigl(\omega(z_0,z_2)\bigr)}{1+\text{tanh}\bigl(\omega(z_1, z_0)\bigr)\text{tanh}\bigl(\omega(z_0,z_2)\bigr)};
  \end{align*}
  dalla definizione di $\omega$ troviamo
  \begin{equation}
    p(z_1,z_2) \le \frac{p(z_1,z_0)+p(z_0,z_2)}{1+p(z_1,z_0)p(z_0,z_2)}. \label{star}
  \end{equation}
  Notiamo che per il lemma di Schwarz-Pick abbiamo che $p$ è invariante sotto l'azione di $\text{Aut}(\mathbb{D})$. Grazie all'Osservazione \ref{transi}, possiamo dunque supporre senza perdita di generalità che $z_0=0$. Dato che $|1-\bar{z}_2z_1| \le 1+|z_1||z_2|$ e $1-|z_1|^2, 1-|z_2|^2>0$, ricordando l'equazione \eqref{formuletta}, per ogni $z_1, z_2 \in \mathbb{D}$ si ha che
  \begin{align*}
    1-\left|\frac{z_1-z_2}{1-\bar{z}_1z_2}\right|^2 & =\frac{(1-|z_1|^2)(1-|z_2|^2)}{|1-\bar{z}_1z_2|^2} \\
    & \ge \frac{(1-|z_1|^2)(1-|z_2|^2)}{(1+|z_1||z_2|)^2}=1-\left(\frac{|z_1|+|z_2|}{1+|z_1||z_2|}\right)^2,
  \end{align*}
  da cui
  $$\frac{|z_1-z_2|}{|1-\bar{z}_2z_1|} \le \frac{|z_1|+|z_2|}{1+|z_1||z_2|},$$
  che è quello che otteniamo inserendo $z_0=0$ nella disuguaglianza \eqref{star} e usando che $p(0,z)=|z|$.
\end{proof}

\begin{defn}
  La funzione $\omega:\mathbb{D}\times \mathbb{D} \longrightarrow [0,+\infty)$ è detta \textit{distanza di Poincaré (o iperbolica)} del disco.
\end{defn}

\begin{defn}
  Data $f \in \text{Hol}(\mathbb{D},\mathbb{D})$, la \textit{derivata iperbolica} è definita come
  $$f^h(w):=\lim_{z \longrightarrow w} \frac{[f(z),f(w)]}{[z,w]}=\frac{f'(w)(1-|w|^2)}{1-|f(w)|^2},$$
  mentre il \textit{rapporto iperbolico} è definito come
  $$f^*(z,w):=\begin{cases}
    \frac{[f(z),f(w)]}{[z,w]} & \mbox{per }z\not=w \\
    f^h(w) & \mbox{per }z=w.
  \end{cases}$$
\end{defn}

Notiamo che, poiché il limite di $f^*(z,w)$ per $z \longrightarrow w$ è ben definito per ogni $w$, per la Proposizione \ref{estensione} abbiamo che la funzione $f^*(z,w)$ è olomorfa in $z \in \mathbb{D}$ per ogni $w \in \mathbb{D}$ fissato.

\begin{oss} \label{oss1}
  \begin{nlist}
    \item Le disuguaglianze del lemma di Schwarz-Pick possono essere riscritte come $|f^*(z,w)| \le 1$, con uguaglianza se e solo se $f \in \text{Aut}(\mathbb{D})$;
    \item  un altro modo di scrivere le disuguaglianze del lemma di Schwarz-Pick è $p\bigl(f(z),f(w)\bigr) \le p(z,w)$, che è equivalente a $\omega\bigl(f(z),f(w)\bigr) \le \omega(z,w)$ in quanto $\text{arctanh}$ è strettamente crescente;
    \item $p(z,0)=|z|$, quindi $\omega(z,0)=\omega(|z|,0)$; analogamente, $\omega(0,z)=\omega(0,|z|)$;
    \item per definizione, $|f^*(z,w)|=|f^*(w,z)|$.
  \end{nlist}
  Questi risultati verranno usati nelle varie dimostrazioni e verranno esplicitati solo quando ciò che ne segue non è immediato.
\end{oss}

Per trattare i casi estremali delle disuguaglianze in \cite{BM}, ci servirà qualche risultato di tipo geometrico sul disco con la distanza iperbolica.

\begin{defn}
  Una \textit{geodetica} per $\omega$ è una curva $\sigma: \mathbb{R} \longrightarrow \mathbb{D}$ tale che per ogni $t_1,t_2 \in \mathbb{R}$ si ha $\omega\bigl(\sigma(t_1),\sigma(t_2)\bigr)=|t_1-t_2|$.

  Diciamo che tre punti $z_1, z_2, z_3$ appartengono \textit{nell'ordine} alla stessa geodetica se $z_j=\sigma(t_j)$ con $t_1 \le t_2 \le t_3$.
\end{defn}

\begin{oss} \label{geoingeo}
  Poiché $\omega$ è invariante sotto l'azione di $\text{Aut}(\mathbb{D})$, gli automorfismi mandano geodetiche in geodetiche; inoltre, conservano l'ordine.
\end{oss}

\begin{ex} \label{diam}
  Dato $z_0 \in \mathbb{D}$, la curva $\sigma(t)=\tanh(t)\frac{z_0}{|z_0|}$ (il diametro passante per $z_0$) è una geodetica. Che sia il diametro indicato segue dal fatto che la funzione $\tanh:\mathbb{R} \longrightarrow (-1,1)$ è biettiva. Che sia una geodetica segue dalla definizione di $\omega$, dalla formula già citata per $\tanh(a+b)$ e dal fatto che $\tanh$ è dispari.
\end{ex}

\begin{oss}
  Dati due punti $z_0$ e $z_1$, esiste sempre una geodetica passante per entrambi. Infatti, per l'Osservazione \ref{geoingeo} e per transitività di $\text{Aut}(\mathbb{D})$ possiamo supporre $z_1=0$; basta dunque prendere il diametro visto nell'Esempio \ref{diam}.
\end{oss}

\begin{lm} \label{nellordine}
  $z_1,z_2,z_3 \in \mathbb{D}$ appartengono nell'ordine alla stessa geodetica se e solo se $\omega(z_1,z_3)=\omega(z_1,z_2)+\omega(z_2,z_3)$.
\end{lm}

\begin{proof}
  Se appartengono nell'ordine alla stessa geodetica, l'uguaglianza segue dalla definizione.

  Supponiamo ora che valga $\omega(z_1,z_3)=\omega(z_1,z_2)+\omega(z_2,z_3)$. Per transitività di $\text{Aut}(\mathbb{D})$ possiamo supporre $z_2=0$. Usando la definizione di $\omega$, l'uguaglianza si riscrive come
  $$\text{arctanh}\left|\frac{z_1-z_3}{1-\bar{z}_3z_1}\right|=\text{arctanh}|z_1|+\text{arctanh}|z_3|;$$
  applicando la tangente iperbolica a entrambi i membri e usando ancora una volta la formula $\tanh(a+b)=\frac{\tanh(a)+\tanh(b)}{1+\tanh(a)\tanh(b)}$, otteniamo
  $$\left|\frac{z_1-z_3}{1-\bar{z}_3z_1}\right|=\frac{|z_1|+|z_3|}{1+|z_1||z_3|}.$$
  Nella dimostrazione della Proposizione \ref{eunadistanza} abbiamo visto una disuguaglianza che coinvolge le stesse quantità; ripercorrendo i passaggi, si trova che la condizione per l'uguaglianza è $|1-\bar{z}_3z_1|=1+|z_3||z_1|$. Ma questo dice proprio che $z_1$ e $z_3$ stanno sullo stesso diametro, da parti opposte rispetto a $0$.
\end{proof}


\subsection{Regioni di Stolz e limiti non tangenziali}
\begin{defn}
  Dati $\alpha \in (0,\pi/2)$ e $\sigma \in \partial\mathbb{D}$, chiamiamo \textit{settore di vertice $\sigma$ e angolo $2\alpha$} l'insieme $S(\sigma,\alpha) \subset \mathbb{D}$ tale che per ogni $z \in S(\sigma,\alpha)$ l'angolo compreso tra la retta congiungente $\sigma$ e $0$ e la retta congiungente $\sigma$ e $z$ ha modulo minore di $\alpha$.
  \begin{center}
    \begin{tikzpicture}[line cap=round,line join=round,>=triangle 45,x=2.5cm,y=2.5cm]
      \draw[->,color=black] (-1.13,0) -- (1.15,0);
      \foreach \x in {-1,1}
      \draw[shift={(\x,0)},color=black] (0pt,2pt) -- (0pt,-2pt);
      \draw[->,color=black] (0,-1.11) -- (0,1.12);
      \foreach \y in {-1,1}
      \draw[shift={(0,\y)},color=black] (2pt,0pt) -- (-2pt,0pt);
      \clip(-1.13,-1.11) rectangle (1.15,1.12);
      \draw(0,0) circle (2.5cm);
      \draw[name path=A] (0.49,0.87)-- (1,0);
      \draw[name path=B] (0.49,-0.87)-- (1,0);
      \tikzfillbetween[of=A and B]{blue, opacity=0.25};
      \draw[name path=C,smooth,samples=100,domain=-1:0.491] plot(\x,{sqrt(1-(\x)^2)});
      \draw[name path=D,smooth,samples=100,domain=-1:0.491] plot(\x,{0-sqrt(1-(\x)^2)});
      \tikzfillbetween[of=C and D]{blue, opacity=0.25};
    \end{tikzpicture}

    In blu, il settore $S(1,\pi/3)$
  \end{center}
\end{defn}

\begin{defn}
  Dati $\sigma \in \partial \mathbb{D}$ e $M>1$, chiamiamo \textit{regione di Stolz $K(\sigma,M)$} l'insieme $\left\{z \in \mathbb{D} \mid \dfrac{|\sigma-z|}{1-|z|} < M\right\}$.
  \begin{center}
    \begin{tikzpicture}[line cap=round,line join=round,>=triangle 45,x=2.5cm,y=2.5cm]
      \draw[->,color=black] (-1.16,0) -- (1.18,0);
      \foreach \x in {-1,1}
      \draw[shift={(\x,0)},color=black] (0pt,2pt) -- (0pt,-2pt);
      \draw[->,color=black] (0,-1.13) -- (0,1.15);
      \foreach \y in {-1,1}
      \draw[shift={(0,\y)},color=black] (2pt,0pt) -- (-2pt,0pt);
      \clip(-1.16,-1.13) rectangle (1.18,1.15);
      \draw(0,0) circle (2.5cm);
      \draw[name path=A, smooth,samples=100,domain=-0.1715705740864879:1] plot(\x,{sqrt(7-4*sqrt(3-2*(\x))-2*(\x)-(\x)^2)});
      \draw[name path=B, smooth,samples=100,domain=-0.1715705740864879:1] plot(\x,{0-sqrt(7-4*sqrt(3-2*(\x))-2*(\x)-(\x)^2)});
      \tikzfillbetween[of=A and B]{red, opacity=0.25};
    \end{tikzpicture}

    In rosso, la regione di Stolz $K(1,2)$
  \end{center}
\end{defn}

\begin{prop} \label{settori-stolz}
  Dato $M>1$, sia $\alpha=\text{\normalfont{arctan}}\sqrt{M^2-1} \in (0,\pi/2)$. Per ogni $\alpha'<\alpha$ esiste $\epsilon>0$ tale che, detto $B(\sigma,\epsilon)=\{z \in \mathbb{C} \mid |\sigma-z|<\epsilon\}$, si ha
  $$S(\sigma,\alpha')\cap B(\sigma,\epsilon) \subset K(\sigma,M) \subset S(\sigma,\alpha).$$
\end{prop}

\begin{proof}
  Per definizione, $S(\sigma,\alpha)$ corrisponde all'insieme $S(1,\alpha)$ ruotato moltiplicando per $\sigma$. Lo stesso vale per $K(\sigma,M)$: infatti, $\dfrac{|\sigma-z|}{1-|z|}=\dfrac{|1-\sigma^{-1}z|}{1-|\sigma^{-1}z|}$. Possiamo dunque supporre senza perdita di generalità $\sigma=1$. È utile osservare che in questo caso $S(1,\alpha)=\left\{z \in \mathbb{D} \mid |\mathfrak{Im}(z)|<(\tan{\alpha})\bigl(1-\mathfrak{Re}(z)\bigr)\right\}$.

  Mostriamo la seconda inclusione. Sia $z \in K(1,M)$ e, poiché $1>|z|>\mathfrak{Re}(z)$, abbiamo che
  $$M>\frac{|1-z|}{1-|z|}\ge \frac{|1-z|}{1-\mathfrak{Re}(z)},$$
  da cui
  $$M^2-1 > \frac{|1-z|^2}{\bigl(1-\mathfrak{Re}(z)\bigr)^2}-1=\frac{1-2\mathfrak{Re}(z)+|z|^2}{\bigl(1-\mathfrak{Re}(z)\bigr)^2}-1=\frac{|\mathfrak{Im}(z)|^2}{\bigl(1-\mathfrak{Re}(z)\bigr)^2};$$
  perciò
  $$\frac{|\mathfrak{Im}(z)|}{1-\mathfrak{Re}(z)}<\sqrt{M^2-1}=\tan{\alpha}.$$

  Mostriamo adesso la prima inclusione. Fissiamo $\alpha'<\alpha$. Supponiamo per assurdo che, per ogni $\epsilon>0$, esista $z \in S(1,\alpha')\cap B(1,\epsilon)$ tale che $z \not\in K(1,M)$. Per tali $z$ si ha allora $\dfrac{|1-z|}{1-|z|} \ge M$, da cui
  \begin{equation}
    \dfrac{1-|z|}{|1-z|} \le \dfrac{1}{M}; \label{star1}
  \end{equation}
  inoltre, poiché $z \in S(1,\alpha')$ abbiamo
  $$\frac{|\mathfrak{Im}(z)|}{1-\mathfrak{Re}(z)}<\tan{\alpha'}.$$
  Elevando al quadrato, sommando $1$ e sfruttando l'uguaglianza vista sopra otteniamo
  $$\frac{|1-z|^2}{\bigl(1-\mathfrak{Re}(z)\bigr)^2}=\frac{|\mathfrak{Im}(z)|^2}{\bigl(1-\mathfrak{Re}(z)\bigr)^2}+1<\tan^2{\alpha'}+1,$$
  che ci dà
  \begin{equation}
    \frac{|1-z|}{1-\mathfrak{Re}(z)}<\sqrt{\tan^2{\alpha'}+1}=:M', \label{star2}
  \end{equation}
  dove $\alpha'<\alpha$, quindi $\tan{\alpha'}<\tan{\alpha}$ e dunque $M'<M$. Moltiplicando tra loro le disuguaglianze \eqref{star1} e \eqref{star2} troviamo $\dfrac{1-|z|}{1-\mathfrak{Re}(z)}<\dfrac{M'}{M}<1$.
  Se mostriamo che $\displaystyle \lim_{\substack{z \longrightarrow 1, \\ z \in S(1,\alpha')}} \frac{1-|z|}{1-\mathfrak{Re}(z)}=1$ avremo trovato una contraddizione. Scriviamo dunque $x=\mathfrak{Re}(z)$ e $y=|\mathfrak{Im}(z)|$, per cui la condizione $z \in S(1,\alpha')$ si scrive come $y/(1-x)<\tan{\alpha'}$. Inoltre vale che
  $$\frac{1-|z|}{1-\mathfrak{Re}(z)}=\frac{1-\sqrt{x^2+y^2}}{1-x}=1-\frac{\sqrt{x^2+y^2}-x}{1-x}.$$
  Vogliamo mostrare che la quantità $\dfrac{\sqrt{x^2+y^2}-x}{1-x}$ tende a $0$ per $(x,y) \longrightarrow (1,0)$ all'interno del settore di angolo $2\alpha'$. Notiamo che, a parte lungo $y=0$ dove il limite è banale, è sempre maggiore di $0$, dunque ci basta stimarne il $\limsup$. Usando che $y/(1-x)<\tan{\alpha'}$ e moltiplicando numeratore e denominatore per $\sqrt{x^2+y^2}-x$ troviamo
  \begin{align*}
    \frac{\sqrt{x^2+y^2}-x}{1-x} & < \tan{\alpha'}\cdot\frac{\sqrt{x^2+y^2}-x}{y}\cdot\frac{\sqrt{x^2+y^2}+x}{\sqrt{x^2+y^2}+x} \\
    & =\tan{\alpha'}\cdot\frac{y^2}{y(\sqrt{x^2+y^2}+x)}=\tan{\alpha'}\cdot\frac{y}{\sqrt{x^2+y^2}+x};
  \end{align*}
  quest'ultima espressione tende a $0$ per $x \longrightarrow 1$ e $y \longrightarrow 0$.
\end{proof}

\begin{defn}
  Diciamo che una funzione $f:\mathbb{D} \longrightarrow \mathbb{C}$ ha \textit{limite non tangenziale} $L \in \mathbb{C}$ in $\sigma \in \partial\mathbb{D}$ e scriviamo
  $$\substack{\text{nt--lim} \\ z \longrightarrow \sigma} \, f(z)=L$$
  se  per ogni $M>1$ si ha $\displaystyle \lim_{\substack{z \longrightarrow \sigma, \\ z \in K(\sigma,M)}} f(z)=L$.

  Date altre due funzioni $g,h: \mathbb{D} \longrightarrow \mathbb{C}$ scriviamo che $f(z)=g(z)+o\bigl(h(z)\bigr)$ per $z \longrightarrow \sigma$ \textit{non tangenzialmente} se
  $$\substack{\text{nt--lim} \\ z \longrightarrow \sigma} \, \frac{f(z)-g(z)}{h(z)}=0.$$
\end{defn}

La seguente proposizione asserisce che, per $z \longrightarrow 1$ non tangenzialmente, un certo andamento di $f$ può essere tradotto nell'andamento di $|f^h|$. È questo che ci permetterà di dimostrare il teorema 2.1 di \cite{BK} passando per la versione al bordo del lemma di Schwarz-Pick.

\begin{prop} \label{o^3->o^2}
  Siano $f \in \text{\normalfont{Hol}}(\mathbb{D},\mathbb{D})$ e $\sigma \in \partial\mathbb{D}$ tali che
  \begin{equation} \label{o^3}
    f(z)=\sigma+(z-\sigma)+o\bigl((z-\sigma)^3\bigr)
  \end{equation}
  per $z \longrightarrow \sigma$ non tangenzialmente. Allora
  \begin{equation} \label{o^2}
    |f^h(z)|=1+o\bigl((z-\sigma)^2\bigr)
  \end{equation}
  per $z \longrightarrow \sigma$ non tangenzialmente.
\end{prop}

\begin{proof}
  A meno di considerare $g(z)=\sigma^{-1}f(\sigma z)$, possiamo supporre senza perdita di generalità $\sigma=1$. Infatti, è facile verificare che nell'ipotesi \eqref{o^3} si ha $g(z)=1+(z-1)+o\bigl((z-1)^3\bigr)$. Usando che $g'(z)=f'(\sigma z)$ si ha che vale
  $$|g^h(z)|=|g'(z)|\frac{1-|z|^2}{1-|g(z)|^2}=|f'(\sigma z)|\frac{1-|z|^2}{1-|\sigma^{-1}f(\sigma z)|^2};$$
  ricordando che $|\sigma|=1$ troviamo
  $$|g^h(z)|=|f'(\sigma z)|\frac{1-|\sigma z|^2}{1-|f(\sigma z)|^2}=|f^h(\sigma z)|.$$
  Se avessimo $|g^h(z)|=1+o\bigl((z-1)^2\bigr)$, mediante la sostituzione $\zeta=\sigma z$ si avrebbe
  $$o\bigl((z-1)^2\bigr)=o\bigl(\sigma^{-2}(\zeta-\sigma)^2\bigr)=o\bigl((\zeta-\sigma)^2\bigr)$$
  e $|f^h(\sigma z)|=|f^h(\zeta)|$; troviamo quindi l'equazione \eqref{o^2} con $\zeta$ al posto di $z$.

  Sia $M>1$ e consideriamo $z \in K(1,M)$. Allora per la Proposizione \ref{settori-stolz} si ha $z \in S(1,\alpha)$ dove $\alpha=\text{arctan}\sqrt{M^2-1}$.
  Sia inoltre $\beta \in (0,\pi/2)$ con $\beta>\alpha$ e sia $C(z)$ il cerchio di centro $z$ e raggio $r(z)=\text{dist}\bigl(z, \partial S(1,\beta)\bigr)$, la distanza euclidea di $z$ dal bordo di $S(1,\beta)$. Allora per la formula integrale di Cauchy applicata alla funzione $f(z)-z$ si ha
  $$f'(z)-1=\frac{1}{2\pi i} \int_{C(z)} \frac{f(w)-w}{(w-z)^2}\diff w:=I(z) \implies f'(z)=1+I(z).$$
  \begin{center}
    \definecolor{qqffqq}{rgb}{0,1,0}
    \definecolor{qqqqff}{rgb}{0,0,1}
    \definecolor{uququq}{rgb}{0.25,0.25,0.25}
    \definecolor{ffqqqq}{rgb}{1,0,0}
    \begin{tikzpicture}[line cap=round,line join=round,>=triangle 45,x=3.0cm,y=3.0cm]
      \draw[->,color=black] (-1.12,0) -- (1.2,0);
      \foreach \x in {-1,1}
      \draw[shift={(\x,0)},color=black] (0pt,2pt) -- (0pt,-2pt);
      \draw[->,color=black] (0,-1.11) -- (0,1.11);
      \foreach \y in {-1,1}
      \draw[shift={(0,\y)},color=black] (2pt,0pt) -- (-2pt,0pt);
      \clip(-1.12,-1.11) rectangle (1.2,1.11);
      \draw [shift={(1,0)},color=ffqqqq,fill=ffqqqq,fill opacity=0.1] (0,0) -- (180:0.26) arc (180:244.77:0.26) -- cycle;
      \draw [shift={(1,0)},color=qqqqff,fill=qqqqff,fill opacity=0.1] (0,0) -- (-180:0.14) arc (-180:-127.09:0.14) -- cycle;
      \draw [shift={(1,0)},color=qqffqq,fill=qqffqq,fill opacity=0.1] (0,0) -- (115.23:0.33) arc (115.23:143.3:0.33) -- cycle;
      \draw(0,0) circle (3cm);
      \draw (1,0)-- (0.27,0.96);
      \draw (0.64,0.77)-- (1,0);
      \draw (0.27,-0.96)-- (1,0);
      \draw (1,0)-- (0.64,-0.77);
      \draw [shift={(1,0)},color=ffqqqq] (180:0.26) arc (180:244.77:0.26);
      \draw [shift={(1,0)},color=ffqqqq] (180:0.23) arc (180:244.77:0.23);
      \draw(0.5,0.37) circle (0.888cm);
      \draw (0.5,0.37)-- (0.765,0.502);
      \draw (0.27,0.54)-- (1,0);
      \draw [shift={(1,0)},color=qqffqq] (115.23:0.33) arc (115.23:143.3:0.33);
      \draw [shift={(1,0)},color=qqffqq] (115.23:0.3) arc (115.23:143.3:0.3);
      \draw [shift={(1,0)},color=qqffqq] (115.23:0.28) arc (115.23:143.3:0.28);
      \begin{scriptsize}
        \draw[color=ffqqqq] (0.84,-0.09) node {$\beta$};
        \fill [color=black] (0.5,0.37) circle (1.5pt);
        \draw[color=black] (0.48,0.32) node {$z$};
        \draw[color=black] (0.18,0.14) node {$C(z)$};
        \draw[color=black] (0.54,0.465) node {$r(z)$};
        \fill [color=black] (1,0) circle (1.5pt);
        \draw[color=black] (1.04,0.05) node {$1$};
        \fill [color=black] (0.26,0.545) circle (1.5pt);
        \draw[color=black] (0.18,0.58) node {$w_0$};
        \fill [color=black] (1,0) circle (1.5pt);
        \draw[color=qqqqff] (0.92,-0.03) node {$\alpha$};
        \draw[color=qqffqq] (0.83,0.16) node {$\gamma$};
      \end{scriptsize}
    \end{tikzpicture}
  \end{center}

  Per la Proposizione \ref{settori-stolz} esiste $\delta>0$ tale che, se $w \in S(1,\beta) \cap B(1,\delta)$, si ha $w \in K(1,M')$ con $M'>\sqrt{\tan^2\beta+1}$.
  Dato $\epsilon>0$ fissato e prendendo $\delta$ sufficientemente piccolo, per ipotesi abbiamo che $|f(w)-w|<\epsilon|1-w|^3$ per ogni $w \in K(1,M') \cap B(1,\delta)$ e di conseguenza per ogni $w \in S(1,\beta) \cap B(1,\delta)$.
  Poiché $S(1,\beta) \subset \mathbb{D}$, dev'essere $r(z) \le \text{dist}(z,\mathbb{D})=1-|z|$. Si ha anche $1-|z| \le |1-z|$; quindi prendendo $z \in B(1,\delta/2)$ abbiamo $r(z) \le |z-1|<\delta/2$. Dunque per ogni $w \in C(z)$ troviamo che $|w-1| \le |w-z|+|z-1|=r(z)+|z-1|<\delta$. Per questi $z$, facendo la sostituzione $w=z+r(z)e^{i\theta}$, vale che
  \begin{align*}
    |I(z)| & =\left|\frac{1}{2\pi i} \int_{C(z)} \frac{f(w)-w}{(w-z)^2}\diff w\right| \le \frac{1}{2\pi} \int_0^{2\pi} \frac{\epsilon\left|1-\bigl(z+r(z)e^{i\theta}\bigr)\right|^3}{\left|\bigl(z+r(z)e^{i\theta}\bigr)-z\right|^2}r(z)\diff\theta \\
    & \le \frac{\epsilon}{r(z)}\max_{\theta \in [0,2\pi]} \left|1-\bigl(z+r(z)e^{i\theta}\bigr)\right|^3=\frac{\epsilon}{r(z)}\max_{w \in C(z)}|1-w|^3.
  \end{align*}
  Il massimo è raggiunto per l'intersezione più lontana da $1$ tra la circonferenza $C(z)$ e la retta passante per $1$ e $z$ (il punto $w_0$ in figura); abbiamo che vale $|1-w_0|=r(z)+|z-1|$. Allora si ha
  $$|I(z)| \le \frac{\epsilon}{r(z)}\bigl(r(z)+|z-1|\bigr)^3=\epsilon r(z)^2\left(1+\frac{|z-1|}{r(z)}\right)^3.$$
  Detto $\gamma$ l'angolo tra la retta congiungente $1$ e $z$ e il tratto affine di $\partial S(1,\beta)$ più vicino a $z$ (che è effettivamente il tratto di bordo più vicino a $z$ se lo si prende sufficientemente vicino a $1$), si ha $\dfrac{|z-1|}{r(z)}=\csc\gamma$.
  Poiché vale che $\gamma \ge \beta-\alpha$ e $r(z) \le |z-1|$, troviamo $r(z)^2(1+\csc\gamma)^3 \le |z-1|^2\bigl(1+\csc(\beta-\alpha)\bigr)^3$. Concatenando le disuguaglianze appena viste, risulta che
  $$|I(z)| \le \epsilon |z-1|^2\bigl(1+\csc(\beta-\alpha)\bigr)^3.$$
  Otteniamo dunque $f'(z)=1+o\bigl((z-1)^2\bigr)$ per $z \longrightarrow 1$ non tangenzialmente.

  Notiamo che all'interno della regione di Stolz $K(1,M)$ si ha $1 \le \dfrac{|z-1|}{1-|z|} \le M$, il che ci  permette di usare indipendentemente $z-1$ o $1-|z|$ negli $o$-piccoli per $z \longrightarrow 1$ non tangenzialmente. Per ipotesi
  $$\frac{1-|f(z)|}{1-|z|}=\frac{1-|z|+o\bigl((z-1)^3\bigr)}{1-|z|}=1+o\bigl((z-1)^2\bigr)$$
  per $z \longrightarrow 1$ non tangenzialmente. Possiamo quindi concludere che
  $$|f^h(z)|=|f'(z)|\frac{1-|z|^2}{1-|f(z)|^2}=1+o\bigl((z-1)^2\bigr)$$
  per $z \longrightarrow 1$ non tangenzialmente.
\end{proof}


\newpage

\section{Lemmi di Schwarz-Pick multi-punto}

\subsection{Teorema di Beardon-Minda e corollari}
Adesso possiamo procedere a dimostrare la serie di disuguaglianze di \cite{BM}, che coinvolgono la distanza di Poincaré $\omega$ e le funzioni olomorfe dal disco in sé che non sono automorfismi.

L'approccio generale è il seguente: stiamo studiando $\text{Hol}(\mathbb{D},\mathbb{D})$. Le funzioni di questo insieme che sono anche automorfismi conservano la distanza iperbolica. Sembra dunque ragionevole vedere come si traduce il lemma di Schwarz-Pick in termini iperbolici (abbiamo già visto che contrae $\omega$) e a quali conseguenze porta.

\begin{defn}
  La \textit{rotazione iperbolica} di ordine due attorno a un punto $a \in \mathbb{D}$ è la funzione $r_a \in \text{Aut}(\mathbb{D})\setminus\{\id\}$ tale che $r_a \circ r_a=\id$ e $r_a(a)=a$.
  Chiamiamo $a$ il \textit{centro di rotazione} di $r_a$.
\end{defn}

\begin{oss}
  Le condizioni imposte sono sufficienti a determinare un unico automorfismo $r_a$. Infatti, a meno di coniugare possiamo supporre $a=0$, per cui $r_0$ dev'essere una rotazione di ordine due, cioè di $\pi$. Inoltre, è facile vedere che $r_a$ è caratterizzata dall'equazione $[r_a(z),a]=-[z,a]$; la si può dunque vedere come l'analogo della rotazione euclidea di angolo $\pi$.
\end{oss}

\begin{oss} \label{rotegeo}
  $z, a$ e $r_a(z)$ appartengono, nell'ordine, alla stessa geodetica. Infatti, a meno di comporre con un opportuno automorfismo possiamo supporre $a=0$; ma poiché $r_0(z)=e^{i\pi}z$, l'affermazione risulta ovvia.
\end{oss}

\begin{defn}
  Dati $a_1,\dots,a_n \in \mathbb{D}$ e $\theta \in \mathbb{R}$, chiamiamo \textit{prodotto di Blaschke} di grado $n$ la funzione
  $$e^{i\theta}\prod_{j=1}^n \frac{z-a_j}{1-\bar{a}_jz}.$$
  Indichiamo con $\mathcal{B}_n$ i prodotti di Blaschke di grado $n$.
\end{defn}

\begin{oss}
  I prodotti di Blaschke sono funzioni olomorfe sul disco unitario, con zeri assegnati. Quelli di grado $1$ sono $\text{Aut}(\mathbb{D})$.
\end{oss}

\begin{lm} \label{Blaschke-car}
  Tra le funzioni $f$ continue in $\overline{\mathbb{D}}$ e olomorfe in $\mathbb{D}$, i prodotti di Blaschke di grado $n$ sono caratterizzati dalle seguenti proprietà:
  \begin{nlist}
    \item se $|z|=1$ allora $|f(z)|=1$;
    \item $f$ ha esattamente $n$ zeri in $\mathbb{D}$ contati con molteplicità.
  \end{nlist}
\end{lm}

\begin{proof}
  Se $f$ è un prodotto di Blaschke di grado $n$, che soddisfi (ii) è ovvio e che soddisfi (i) segue dall'Osservazione \ref{dom}.

  Fissiamo ora $f$ che soddisfi (i) e (ii); consideriamo $B$ il prodotto di Blaschke definito con $\theta=0$ e $a_n$ gli zeri in $\mathbb{D}$ di $f$, contati con molteplicità. Allora $f/B$ e $B/f$ sono due funzioni olomorfe su $\mathbb{D}$ e continue in $\overline{\mathbb{D}}$, di modulo $1$ sul bordo. Per il principio del massimo per funzioni olomorfe, deve essere $|f/B| \le 1$ e $|B/f| \le 1$ sul disco unitario, da cui $|f/B|=1$ e $f/B$ è costante in $\mathbb{D}$.
\end{proof}

\begin{prop} \label{blaschke-prop}
  Valgono le seguenti affermazioni:
  \begin{nlist}
    \item date $F \in \text{Hol}(\mathbb{D},\mathbb{D})$ e $S \in \text{Aut}(\mathbb{D})$, si ha che $F \in \mathcal{B}_n$ se e solo se $S\circ F \in \mathcal{B}_n$;
    \item si ha che $f \in \mathcal{B}_{n+1}$ se e solo se $f^*(z,w) \in \mathcal{B}_n$, con $w$ un qualsiasi elemento di $\mathbb{D}$ fissato.
  \end{nlist}
\end{prop}

\begin{proof}
  Per dimostrare (i), basta mostrare che $S$ conserva le proprietà del Lemma \ref{Blaschke-car}. La prima segue dall'Osservazione \ref{dom}. Per transitività di $\text{Aut}(\mathbb{D})$, la seconda corrisponde a dover dimostrare che, dato $c \in \mathbb{D}$, l'equazione $F(z)=c$ ha esattamente $n$ zeri contati con molteplicità in $\mathbb{D}$, dove $F \in B_n$. Scriviamo $F(z)=\displaystyle e^{i\theta}\prod_{j=1}^n \frac{z-a_j}{1-\bar{a}_jz}$. Allora $F(z)=c$ si riscrive come
  \begin{equation} \label{pol-eq}
    c\prod_{j=1}^n (1-\bar{a}_jz)=e^{i\theta}\prod_{j=1}^n(z-a_j).
  \end{equation}
  Stiamo uguagliando due polinomi di grado $n$ con coefficienti direttivi diversi: quello al membro sinistro è di modulo minore di $1$, mentre quello al membro destro ha modulo esattamente $1$. Dunque la nostra equazione ha esattamente $n$ soluzioni, contate con molteplicità, in $\mathbb{C}$.
  Per l'Osservazione \ref{dom}, se $z \not\in \mathbb{D}$ e $z\not=1/\bar{a}_j$ per $j=1,\dots,n$, si ha $|F(z)| \ge 1$; se $z=1/\bar{a}_j$ per qualche $j$, il membro sinistro di \eqref{pol-eq} è $0$ ma il membro destro no, quindi non ci sono soluzioni in quel caso. Perciò, per $|c|<1$, tutte le soluzioni trovate sono in $\mathbb{D}$ come voluto.

  Per definizione di quoziente iperbolico, $[f(z),f(w)]=[z,w]f^*(z,w)$. Il membro di sinistra è della forma $S\bigl(f(z)\bigr)$, dove $S \in \text{Aut}(\mathbb{D})$ è $S(z)=[z,f(w)]$; scriviamo anche $T(z)=[z,w]$.
  Se $f^*(z,w) \in \mathcal{B}_n$, allora $T(z)f^*(z,w) \in \mathcal{B}_{n+1}$; dunque $S\circ f \in \mathcal{B}_{n+1}$ e per il punto (i) abbiamo $f \in \mathcal{B}_{n+1}$. Viceversa, se $f \in \mathcal{B}_{n+1}$ si ha $S\circ f \in \mathcal{B}_{n+1}$.
  Sappiamo anche che $S\bigl(f(w)\bigr)=0$, dunque nel prodotto ci dev'essere il fattore $[z,w]$; segue dunque che $f^*(z,w) \in \mathcal{B}_n$.
\end{proof}

\begin{lm} \label{z^2}
  Sia $f \in \mathcal{B}_2$, esistono $S, T \in \text{Aut}(\mathbb{D})$ tali che $S\circ f\circ T(z)=z^2$.
\end{lm}

\begin{proof}
  Per transitività di $\text{Aut}(\mathbb{D})$ su $\mathbb{D}$, ci basta dimostrare che esiste un $c \in \mathbb{D}$ tale che $f(z)-c$ abbia un unico zero doppio $z_0$ nel disco, il punto critico di $f$. È sufficiente imporre $f'(z_0)=0$, dato che sappiamo già che $f(z_0) \in \mathbb{D}$. Scriviamo $f(z)=e^{i\theta}\cdot\dfrac{z-a_1}{1-\bar{a}_1z}\cdot\dfrac{z-a_2}{1-\bar{a}_2z}$.
  Sempre per transitività, usando $T$ possiamo supporre $a_1=0$ e poniamo $a_2=a$. Si ha
  $$f'(z)=e^{i\theta}\left(\frac{z-a}{1-\bar{a}z}+z\cdot\frac{1-|a|^2}{1-\bar{a}z}\right).$$
  Con qualche passaggio algebrico, l'equazione $f'(z)=0$ diventa
  $$\bar{a}z^2-2z+a=0.$$
  Le soluzioni sono $(1 \pm \sqrt{1-|a|^2})/\bar{a}$; da $|a|<1$, abbiamo che quella con il più sta fuori da $\mathbb{D}$ mentre quella con il meno sta dentro, dunque è la soluzione cercata.
\end{proof}

Notazione: sia $f \in \mathcal{B}_2$; indichiamo con $R_f$ la rotazione iperbolica di ordine due attorno al punto $z_0$, trovato nella dimostrazione del Lemma \ref{z^2}.

\begin{cor} \label{rotazioni}
  Sia $f \in \mathcal{B}_2$. Allora $f^*\bigl(R_f(w),w\bigr)=0$.
\end{cor}

\begin{proof}
  Siano $S, T \in \text{Aut}(\mathbb{D})$ date dal Lemma \ref{z^2} tali che $S\circ f\circ T=F$ con $F(z)=z^2$. Abbiamo $[R_F(z),0]=-[z,0]$, dunque per l'Osservazione \ref{muu} troviamo $\mu[T\bigl(R_F(z)\bigr),T(0)]=-\mu[T(z),T(0)]$.
  Per costruzione, $T(0)$ dev'essere il punto critico di $f$, quindi la rotazione $R_f$ è caratterizzata dall'equazione $[R_f(z),T(0)]=-[z,T(0)]$. Ne segue che $T\bigl(R_F(z)\bigr)=R_f\bigl(T(z)\bigr)$. Perciò abbiamo $f=S^{-1}\circ F\circ T^{-1}$ e $R_f=T\circ R_F\circ T^{-1}$, inoltre $F\circ R_F=F$. Ne deduciamo che
  $$f\circ R_f=S^{-1}\circ F\circ R_F\circ T^{-1}=S^{-1}\circ F\circ T^{-1}=f,$$
  che implica la tesi.
\end{proof}

Il seguente risultato rende chiaro perché nelle varie disuguaglianze si richiede che $f$ non sia un automorfismo.

\begin{prop} \label{24}
  Siano $f \in \text{\normalfont{Hol}}(\mathbb{D},\mathbb{D})\setminus\text{\normalfont{Aut}}(\mathbb{D})$ e $v \in \mathbb{D}$. Allora per ogni $z \in \mathbb{D}$ si ha che $f^*(z,v) \in \mathbb{D}$ e la funzione $z \longmapsto f^*(z,v)$ è olomorfa.
\end{prop}

\begin{proof}
  L'olomorfia l'abbiamo già vista quando abbiamo definito il rapporto iperbolico. Per il lemma di Schwarz-Pick, $|f^*(z,v)| \le 1$; inoltre, vale l'uguaglianza in qualche punto solo se $f$ è un automorfismo. Dunque le ipotesi su $f$ assicurano che vale la disuguaglianza stretta sempre, cioè $f^*(z,v) \in \mathbb{D}$ per ogni $z \in \mathbb{D}$.
\end{proof}

L'idea dell'articolo di Beardon-Minda è di applicare il lemma di Schwarz-Pick alla funzione $f^*(z,v)$, con $v$ fissato. Per $f \in \text{Aut}(\mathbb{D})$, avremmo una funzione costante di modulo uguale a $1$, in particolare in $\mathcal{O}(\mathbb{D})\setminus\text{Hol}(\mathbb{D},\mathbb{D})$; non potremmo quindi fare quanto appena detto.

\begin{thm} \label{31}
  (Beardon-Minda, 2004) Sia $f \in \text{\normalfont{Hol}}(\mathbb{D},\mathbb{D})\setminus\text{\normalfont{Aut}}(\mathbb{D})$. Allora per ogni $z, w, v \in \mathbb{D}$ vale
  \begin{equation} \label{3.1}
    \omega\bigl(f^*(z,v),f^*(w,v)\bigr) \le \omega(z,w).
  \end{equation}
  Si ha l'uguaglianza se e solo se $f \in \mathcal{B}_2$.
\end{thm}

\begin{proof}
  Poiché $f$ non è un automorfismo, per la Proposizione \ref{24} la funzione $z \longmapsto f^*(z,v)$ è in $\text{Hol}(\mathbb{D}, \mathbb{D})$; perciò il membro sinistro della disuguaglianza \eqref{3.1} è ben definito e la tesi segue dal lemma di Schwarz-Pick e dall'osservazione \ref{oss1}, punto (ii).

  Sempre dal lemma di Schwarz-Pick, si ha l'uguaglianza se e solo se abbiamo $f^*(z,v) \in \text{Aut}(\mathbb{D})=\mathcal{B}_1$. Per il punto (ii) della Proposizione \ref{blaschke-prop}, questo è equivalente a $f \in \mathcal{B}_2$.
\end{proof}

Vogliamo vedere che questa versione a tre punti è effettivamente un miglioramento rispetto al lemma di Schwarz-Pick.

\begin{ex}
  Consideriamo una funzione $f \in \text{Hol}(\mathbb{D},\mathbb{D})\setminus\text{Aut}(\mathbb{D})$ tale che $f(0)=0$. Il lemma di Schwarz-Pick diventa semplicemente il lemma di Schwarz e ci dice che $f(z)/z \in \mathbb{D}$. Prendendo invece $w=0$ e $v=0$ in \eqref{3.1}, troviamo $\omega\bigl(f(z)/z,f'(0)\bigr) \le \omega(z,0)$.
  Dunque $f(z)/z$ appartiene al disco iperbolico di centro $f'(0)$ e raggio $\omega(z,0)$, che è un sottoinsieme di $\mathbb{D}$.
\end{ex}

\begin{cor} \label{32}
  Sia $f \in \text{\normalfont{Hol}}(\mathbb{D},\mathbb{D})\setminus\text{\normalfont{Aut}}(\mathbb{D})$. Allora per ogni $z, w, v \in \mathbb{D}$ vale
  \begin{equation}
    \omega\bigl(0, f^*(z,v)\bigr) \le \omega\bigl(0,f^*(w,v)\bigr)+\omega(z,w).
  \end{equation}
  Si ha l'uguaglianza se e solo se $f \in \mathcal{B}_2$ e $R_f(v)$, $w$ e $z$ giacciono sulla stessa geodetica, in quest'ordine.
\end{cor}

\begin{proof}
  Applicando la disuguaglianza triangolare per $\omega$ e il Teorema \ref{31}, si ha
  \begin{align*}
    \omega\bigl(0,f^*(z,v)\bigr) & \le \omega\bigl(0,f^*(w,v)\bigr)+\omega\bigl(f^*(w,v),f^*(z,v)\bigr) \\
    & \le \omega\bigl(0,f^*(w,v)\bigr)+\omega(z,w).
  \end{align*}

  Si ha l'uguaglianza se e solo se vale in entrambe le disuguaglianze appena viste. La seconda è esattamente il caso di uguaglianza del Teorema \ref{31}, che è equivalente a $f \in \mathcal{B}_2$. Sia $T_v(z)=f^*(z,v)$; per il Teorema \ref{blaschke-prop}, $f \in \mathcal{B}_2$ è equivalente a $T_v \in \text{Aut}(\mathbb{D})$. Ricordiamo che $p$, dunque anche $\omega$, è invariante sotto l'azione di $\text{Aut}(\mathbb{D})$.
  Allora il caso di uguaglianza nella prima delle due disuguaglianze si riscrive come $\omega\bigl(T_v^{-1}(0),z\bigr)=\omega\bigl(T_v^{-1}(0),w\bigr)+\omega(w,z)$; per il Lemma \ref{nellordine}, quest'uguaglianza caratterizza l'appartenenza, nell'ordine, alla stessa geodetica. Per il Corollario \ref{rotazioni} si ha $T_v^{-1}(0)=R_f(v)$.
\end{proof}

\begin{cor} \label{33}
  Sia $f \in \text{\normalfont{Hol}}(\mathbb{D},\mathbb{D})\setminus\text{\normalfont{Aut}}(\mathbb{D})$. Allora per ogni $z, w, v, u \in \mathbb{D}$ vale
  \begin{equation} \label{eq33}
    \omega\bigl(0, f^*(z,v)\bigr) \le \omega\bigl(0, f^*(u,w)\bigr)+\omega(z,w)+\omega(v,u).
  \end{equation}
  Si ha l'uguaglianza se e solo se $f \in \mathcal{B}_2$ e $R_f(v), R_f(u), w$ e $z$ giacciono sulla stessa geodetica, in quest'ordine.
\end{cor}
\begin{proof}
  Applicando il Corollario \ref{32} si ha
  \begin{align*}
    \omega\bigl(0,f^*(z,v)\bigr) & \le \omega\bigl(0,f^*(w,v)\bigr)+\omega(z,w) =\omega\bigl(0,|f^*(w,v)|\bigr)+\omega(z,w) \\
    & =\omega\bigl(0,|f^*(v,w)|\bigr)+\omega(z,w)=\omega\bigl(0,f^*(v,w)\bigr)+\omega(z,w).
  \end{align*}
  Sempre per il Corollario \ref{32} abbiamo
  $$\omega\bigl(0,f^*(v,w)\bigr) \le \omega\bigl(0,f^*(u,w)\bigr)+\omega(z,w)+\omega(v,u).$$
  Mettendo assieme le due disuguaglianze otteniamo la \eqref{eq33}.

  Se si ha l'uguaglianza, dobbiamo studiarla nelle due applicazioni del Corollario \ref{32}. In entrambi i casi ci dice che $f \in \mathcal{B}_2$. La prima ci dice anche che $R_f(v), w$ e $z$ appartengono, nell'ordine, alla stessa geodetica. Dalla seconda deduciamo la stessa cosa per $R_f(w), u$ e $v$. Poiché $R_f$ è un automorfismo, abbiamo che lascia invariate le geodetiche e l'ordine dei punti sulle stesse; allora anche $w, R_f(u)$ e $R_f(v)$ stanno, nell'ordine, sulla stessa geodetica. Segue il secondo enunciato della tesi. Viceversa, se valgono tutte queste condizioni è facile vedere che si ha l'uguaglianza.
\end{proof}

Il risultato seguente non ci servirà nel seguito, ma viene riportato per completezza.

\begin{cor} \label{35}
  Sia $f \in \normalfont{\text{Hol}}(\mathbb{D},\mathbb{D})$ e siano $z, w \in \mathbb{D}$. Sia $\sigma$ una geodetica con $\sigma(t_1)=z, \sigma(t_2)=v$ e sia $w=\sigma(t)$ con $t_1<t<t_2$. Allora
  \begin{equation} \label{geod}
    2\omega\bigl(f(z),f(v)\bigr) \le \log\Bigl(\cosh\bigl(2\omega(z,v)\bigr)+|f^h(w)|\sinh\bigl(2\omega(z,v)\bigr)\Bigr).
  \end{equation}
  Si ha l'uguaglianza se e solo se $f \in \mathcal{B}_2$, $z$ e $v$ giacciono sulla stessa geodetica passante per $c$, il punto critico di $f$, e $w=z$ e sta fra $c$ e $v$ oppure $w=v$ e sta fra $c$ e $z$.
\end{cor}

\begin{proof}
  Ricordando le definizioni di $\omega$ e $p$, per ogni $z,w \in \mathbb{D}$ abbiamo
  $$\tanh\bigl(\omega(0,z)\bigr)=|z|\text{ e }\tanh\bigl(\omega(z,w)\bigr)=p(z,w)=\tanh\bigl(\omega(0,[z,w])\bigl).$$
  Dalla definizioni di $f^*(z,w)$ e $p$ e usando le uguaglianze appena citate, abbiamo
  \begin{align*}
    \tanh\Bigl(\omega\bigl(f(z),f(v)\bigr)\Bigr)&=p\bigl(f(z),f(v)\bigr)=p(z,v)|f^*(z,v)| \\
    &=p(z,v)\tanh\Bigl(\omega\bigl(0,f^*(z,v)\bigr)\Bigr).
  \end{align*}
  Applicando il Corollario \ref{33} con $u=w$ e usando il fatto che $z,w$ e $v$ stanno, nell'ordine, sulla stessa geodetica, troviamo
  $$\omega\bigl(0,f^*(z,v)\bigr) \le \omega\bigl(0,f^h(w)\bigr)+\omega(z,v);$$
  ricordando che la tangente iperbolica è strettamente crescente e sfruttando la formula $\tanh(a+b)=\frac{\tanh{a}+\tanh{b}}{1+\tanh{a}\tanh{b}}$ e le uguaglianze di prima, si ha
  $$\tanh\Bigl(\omega\bigl(0,f^*(z,w)\bigr)\Bigr) \le\frac{|f^h(w)|+p(z,v)}{1+|f^h(w)|p(z,v)}.$$
  Mettendo assieme le varie disuguaglianze, risulta che
  $$\tanh\Bigl(\omega\bigl(f(z),f(v)\bigr)\Bigr) \le p(z,v)\left(\frac{|f^h(w)|+p(z,v)}{1+|f^h(w)|p(z,v)}\right);$$
  prendendo $\tanh^{-1}(x)=\frac{1}{2}\log\left(\frac{1+x}{1-x}\right)$ otteniamo
  \begin{align*}
    \omega\bigl(f(z),f(v)\bigr) & \le \frac{1}{2}\log\left(\frac{1+p(z,v)|f^h(w)|+p(z,v)(|f^h(w)|+p(z,v))}{1+p(z,v)|f^h(w)|-p(z,v)(|f^h(w)|+p(z,v))}\right) \\
    & =\frac{1}{2}\left(\frac{1+p^2(z,v)}{1-p^2(z,v)}+|f^h(w)|\frac{2p(z,v)}{1-p^2(z,v)}\right).
  \end{align*}
  Usando le uguaglianze
  $$\frac{1+p^2}{1-p^2}=\cosh(2\omega) \text{ e } \frac{2p}{1-p^2}=\sinh(2\omega)$$
  otteniamo la disuguaglianza della tesi.

  Il caso di uguaglianza segue dall'uguaglianza nel Corollario \ref{33}: ricordando l'Osservazione \ref{rotegeo}, notiamo che se $w\not=z,v$ non possono essere soddisfatti tutti gli allineamenti richiesti nell'ordine giusto.
\end{proof}

\begin{oss}
  Poiché $|f^*| \le 1$, anche $|f^h| \le 1$; per stretta crescenza del logaritmo, otteniamo che la disuguaglianza \eqref{geod} è un altro miglioramento rispetto al lemma di Schwarz-Pick.
\end{oss}

Vorremmo ora dedurre dei risultati simili al lemma di Schwarz-Pick anche per la derivata iperbolica. In generale $f^h$ non è olomorfa, dunque non è possibile applicare il lemma come si è fatto per $f^*(z,v)$. Tuttavia, quanto visto finora può essere usato per mostrare disuguaglianze che si avvicinano a qualcosa di tipo Schwarz-Pick, a costo di un fattore $2$.

La prima richiede che uno dei due punti sia $0$.

\begin{cor} \label{36}
  Sia $f \in \text{\normalfont{Hol}}(\mathbb{D},\mathbb{D})\setminus\text{\normalfont{Aut}}(\mathbb{D})$ tale che $f(0)=0$. Allora
  \begin{equation}
    \omega\bigl(f^h(0),f^h(z)\bigr) \le 2\omega(0,z).
  \end{equation}
  Inoltre, $2$ è la migliore costante possibile.
\end{cor}

\begin{proof}
  Da $f(0)=0$ si ha $f^*(z,0)=f^*(0,z)$. Dalla disuguaglianza triangolare per $\omega$ abbiamo che
  \begin{align*}
    \omega\bigl(f^h(0),f^h(z)\bigr) & = \omega\bigl(f^*(0,0),f^*(z,z)\bigr) \\
    & \le \omega\bigl(f^*(0,0),f^*(z,0)\bigr)+\omega\bigl(f^*(0,z),f^*(z,z)\bigr),
  \end{align*}
  inoltre per il Teorema \ref{31} si ha
  $$\omega\bigl(f^*(0,0),f^*(z,0)\bigr)+\omega\bigl(f^*(0,z),f^*(z,z)\bigr)\le 2\omega(0,z).$$
  Mettendo assieme troviamo la disuguaglianza della tesi.

  Per dire che $2$ è la migliore costante possibile, basta prendere $f(z)=z^2$ e $z \in \mathbb{D}$ con $|z|=1/3$ per ottenere l'uguaglianza.
\end{proof}

Il prossimo risultato è quello che ci permetterà di dimostrare la disuguaglianza di Golusin. È valido per ogni coppia di punti nel disco, ma dobbiamo considerare il modulo della derivata iperbolica.

\begin{cor} \label{quasigolusin}
  Sia $f \in \text{\normalfont{Hol}}(\mathbb{D},\mathbb{D})\setminus\text{\normalfont{Aut}}(\mathbb{D})$. Allora per ogni $z, w \in \mathbb{D}$ vale
  \begin{equation} \label{quasigol}
    \omega\bigl(|f^h(z)|, |f^h(w)|\bigr) \le 2\omega(z,w).
  \end{equation}
  Si ha l'uguaglianza se e solo se $f \in \mathcal{B}_2$ e $z$ e $w$ giacciono sulla stessa geodetica, passante per il punto critico di $f$.
\end{cor}

\begin{proof}
  Siano $z, w \in \mathbb{D}$; senza perdita di generalità possiamo supporre $|f^h(z)| \ge |f^h(w)|$. Allora dalla definizione di $\omega$ abbiamo
  \begin{align*}
    \omega\bigl(|f^h(z)|, |f^h(w)|\bigr) & =\frac{1}{2}\log\left(\frac{1+\frac{|f^h(z)|-|f^h(w)|}{1-|f^h(w)||f^h(z)|}}{1-\frac{|f^h(z)|-|f^h(w)|}{1-|f^h(w)||f^h(z)|}}\right) \\
    & =\frac{1}{2}\log\left(\frac{1-|f^h(w)||f^h(z)|+|f^h(z)|-|f^h(w)|}{1-|f^h(w)||f^h(z)|+|f^h(w)|-|f^h(z)|}\right) \\
    & =\frac{1}{2}\log\left(\frac{1+|f^h(z)|}{1-|f^h(z)|}\cdot\frac{1-|f^h(w)|}{1+|f^h(w)|}\right) \\
    & =\frac{1}{2}\log\left(\frac{1+|f^h(z)|}{1-|f^h(z)|}\right)-\frac{1}{2}\log\left(\frac{1+|f^h(w)|}{1-|f^h(w)|}\right)
  \end{align*}
  Usando di nuovo la definizione di $\omega$ otteniamo dunque
  \begin{align*}
    \omega\bigl(|f^h(z)|, |f^h(w)|\bigr)&=\omega\bigl(0,|f^h(z)|\bigr)-\omega\bigl(0,|f^h(w)|\bigr) \\
    & =\omega\bigl(0,f^h(z)\bigr)-\omega\bigl(0,f^h(w)\bigr) \le 2\omega(z,w),
  \end{align*}
  dove l'ultima disuguaglianza segue dal Corollario \ref{33} prendendo $u=w$ e $v=z$. Il caso di uguaglianza segue facilmente.
\end{proof}

Concludiamo la sezione con il lemma di Dieudonné \cite[Chapitre III, paragraphe 8, équation (25)]{D}, per il quale l'approccio dell'articolo di Beardon e Minda semplifica le dimostrazioni.

\begin{lm}
  (lemma di Dieudonné) Sia $f \in \text{Hol}(\mathbb{D},\mathbb{D})$ tale che $f(0)=0$ e sia $z_0 \in \mathbb{D}$. Allora
  \begin{equation} \label{dieu}
    |f'(z_0)-f(z_0)/z_0| \le \frac{|z_0|^2-|f(z_0)|^2}{|z_0|(1-|z_0|^2)}.
  \end{equation}
  In particolare,
  \begin{equation} \label{dieuprimo}
    |f'(z)| \le \begin{cases}
      1 & \mbox{se } |z| \le \sqrt{2}-1 \\
      \dfrac{(1+|z|^2)^2}{4|z|(1-|z|^2)} & \mbox{se } |z| \ge \sqrt{2}-1.
  \end{cases}
  \end{equation}
\end{lm}

\begin{proof}
  Per il Teorema \ref{31} con $z=v=z_0$ e $w=0$ abbiamo
  \begin{align*}
    \omega\bigl(f^h(z_0),f^*(0,z_0)\bigr) & \le \omega(0,z_0) \\
    \iff p\bigl(f^h(z_0),f^*(0,z_0)\bigr) & \le p(0,z_0)=|z_0|,
  \end{align*}
  dove l'equivalenza fra le due disuguaglianze segue dal fatto che $\text{arctanh}$ è strettamente crescente. Per semplificare, scriviamo $f^h(z_0)=a, f^*(0,z_0)=b$ e $|z_0|=r$. Vogliamo portare la disuguaglianza in forma euclidea. Abbiamo
  $$\left|\frac{a-b}{1-\bar{b}a}\right|=p(a,b) \le r,$$
  che si riscrive come
  \begin{align*}
    (a-b)(\bar{a}-\bar{b}) & \le r^2(1-\bar{b}a)(1-b\bar{a}) \\
    & \iff |a|^2(1-r^2|b|^2)-a\bar{b}(1-r^2)-\bar{a}b(1-r^2) \le r^2-|b|^2 \\
    & \iff |a|^2-a\cdot\frac{\bar{b}(1-r^2)}{1-r^2|b|^2}-\bar{a}\cdot\frac{b(1-r^2)}{1-r^2|b|^2} \le \frac{r^2-|b|^2}{1-r^2|b|^2};
  \end{align*}
  ponendo $\alpha=\dfrac{b(1-r^2)}{1-r^2|b|^2}$ e $R^2=\dfrac{r^2-|b|^2}{1-r^2|b|^2}+|\alpha|^2$, si ha
  $$|a|^2-a\bar{\alpha}-\bar{a}\alpha \le R^2-|\alpha|^2 \iff (a-\alpha)(\bar{a}-\bar{\alpha}) \le R^2\iff |a-\alpha| \le R.$$
  Ricordando che $r=|z_0|$ e osservando che $b=f^*(0,z_0)=\frac{[f(0),f(z_0)]}{[0,z_0]}=\frac{f(z_0)}{z_0}$, troviamo $\alpha=\dfrac{f(z_0)(1-|z_0|^2)}{z_0\bigl(1-|f(z_0)|^2\bigr)}$ e $R=\dfrac{|z_0|^2-|f(z_0)|^2}{|z_0|\bigl(1-|f(z_0)|^2\bigr)}$.
  Riprendendo infine la definizione di $a$, cioè $a=f^h(z_0)=\frac{f'(z_0)(1-|z_0|^2)}{1-|f(z_0)|^2}$, otteniamo che
  $$\left|\frac{f'(z_0)(1-|z_0|^2)}{1-|f(z_0)|^2}-\frac{f(z_0)(1-|z_0|^2)}{z_0\bigl(1-|f(z_0)|^2\bigr)}\right| \le \frac{|z_0|^2-|f(z_0)|^2}{|z_0|\bigl(1-|f(z_0)|^2\bigr)},$$
  che è equivalente alla \eqref{dieu} moltiplicando entrambi i membri per $\frac{1-|f(z_0)|^2}{1-|z_0|^2}$.

  Mostriamo ora la \eqref{dieuprimo}. Dalla \eqref{dieu} con $z$ al posto di $z_0$ si ha
  \begin{align*}
    |f'(z)| & \le |f'(z)-f(z)/z|+|f(z)/z| \\
    & \le \frac{|z|^2-|f(z)|^2}{|z|(1-|z|^2)}+\frac{|f(z)|}{|z|}=\frac{(|f(z)|+|z|^2)(1-|f(z)|)}{|z|(1-|z|^2)}.
  \end{align*}
  Adesso, per $|z|$ fissato un rapido conto mostra che il massimo si ottiene per $|f(z)|=\frac{1-|z|^2}{2}$; poiché per il lemma di Schwarz dev'essere $|f(z)| \le |z|$, questo caso può essere raggiunto solo se $|z| \ge \sqrt{2}-1$ e sostituendo otteniamo la seconda espressione della \eqref{dieuprimo}. Nell'altro caso il massimo si ottiene per $|f(z)|=|z|$ e sostituendo si trova che è proprio $1$.
\end{proof}


\subsection{Applicazioni dei lemmi di Schwarz-Pick multi-punto}
Vediamo ora alcune applicazioni dei risultati visti nella sezione precedente.

Apriamo la sezione con il lemma di Dieudonné \cite[Chapitre III, paragraphe 8, équation (25)]{D}, per il quale l'approccio dell'articolo di Beardon e Minda semplifica la dimostrazione.

\begin{lm}
  (lemma di Dieudonné) Sia $f \in \text{Hol}(\mathbb{D},\mathbb{D})$ tale che $f(0)=0$ e sia $z_0 \in \mathbb{D}$. Allora
  \begin{equation} \label{dieu}
    |f'(z_0)-f(z_0)/z_0| \le \frac{|z_0|^2-|f(z_0)|^2}{|z_0|(1-|z_0|^2)}.
  \end{equation}
  In particolare,
  \begin{equation} \label{dieuprimo}
    |f'(z)| \le \begin{cases}
      1 & \mbox{se } |z| \le \sqrt{2}-1 \\
      \dfrac{(1+|z|^2)^2}{4|z|(1-|z|^2)} & \mbox{se } |z| \ge \sqrt{2}-1.
  \end{cases}
  \end{equation}
\end{lm}

\begin{proof}
  Per il Teorema \ref{31} con $z=v=z_0$ e $w=0$ abbiamo
  \begin{align*}
    \omega\bigl(f^h(z_0),f^*(0,z_0)\bigr) & \le \omega(0,z_0) \\
    \iff p\bigl(f^h(z_0),f^*(0,z_0)\bigr) & \le p(0,z_0)=|z_0|,
  \end{align*}
  dove l'equivalenza fra le due disuguaglianze segue dal fatto che $\text{arctanh}$ è strettamente crescente. Per semplificare, scriviamo $f^h(z_0)=a, f^*(0,z_0)=b$ e $|z_0|=r$. Vogliamo portare la disuguaglianza in forma euclidea. Abbiamo
  $$\left|\frac{a-b}{1-\bar{b}a}\right|=p(a,b) \le r,$$
  che si riscrive come
  \begin{align*}
    (a-b)(\bar{a}-\bar{b}) & \le r^2(1-\bar{b}a)(1-b\bar{a}) \\
    & \iff |a|^2(1-r^2|b|^2)-a\bar{b}(1-r^2)-\bar{a}b(1-r^2) \le r^2-|b|^2 \\
    & \iff |a|^2-a\cdot\frac{\bar{b}(1-r^2)}{1-r^2|b|^2}-\bar{a}\cdot\frac{b(1-r^2)}{1-r^2|b|^2} \le \frac{r^2-|b|^2}{1-r^2|b|^2};
  \end{align*}
  ponendo $\alpha=\dfrac{b(1-r^2)}{1-r^2|b|^2}$ e $R^2=\dfrac{r^2-|b|^2}{1-r^2|b|^2}+|\alpha|^2$, si ha
  $$|a|^2-a\bar{\alpha}-\bar{a}\alpha \le R^2-|\alpha|^2 \iff (a-\alpha)(\bar{a}-\bar{\alpha}) \le R^2\iff |a-\alpha| \le R.$$
  Ricordando che $r=|z_0|$ e osservando che $b=f^*(0,z_0)=\frac{[f(0),f(z_0)]}{[0,z_0]}=\frac{f(z_0)}{z_0}$, troviamo $\alpha=\dfrac{f(z_0)(1-|z_0|^2)}{z_0\bigl(1-|f(z_0)|^2\bigr)}$ e $R=\dfrac{|z_0|^2-|f(z_0)|^2}{|z_0|\bigl(1-|f(z_0)|^2\bigr)}$.
  Riprendendo infine la definizione di $a$, cioè $a=f^h(z_0)=\frac{f'(z_0)(1-|z_0|^2)}{1-|f(z_0)|^2}$, otteniamo che
  $$\left|\frac{f'(z_0)(1-|z_0|^2)}{1-|f(z_0)|^2}-\frac{f(z_0)(1-|z_0|^2)}{z_0\bigl(1-|f(z_0)|^2\bigr)}\right| \le \frac{|z_0|^2-|f(z_0)|^2}{|z_0|\bigl(1-|f(z_0)|^2\bigr)},$$
  che è equivalente alla \eqref{dieu} moltiplicando entrambi i membri per $\frac{1-|f(z_0)|^2}{1-|z_0|^2}$.

  Mostriamo ora la \eqref{dieuprimo}. Dalla \eqref{dieu} con $z$ al posto di $z_0$ si ha
  \begin{align*}
    |f'(z)| & \le |f'(z)-f(z)/z|+|f(z)/z| \\
    & \le \frac{|z|^2-|f(z)|^2}{|z|(1-|z|^2)}+\frac{|f(z)|}{|z|}=\frac{(|f(z)|+|z|^2)(1-|f(z)|)}{|z|(1-|z|^2)}.
  \end{align*}
  Adesso, per $|z|$ fissato un rapido conto mostra che il massimo si ottiene per $|f(z)|=\frac{1-|z|^2}{2}$; poiché per il lemma di Schwarz dev'essere $|f(z)| \le |z|$, questo caso può essere raggiunto solo se $|z| \ge \sqrt{2}-1$ e sostituendo otteniamo la seconda espressione della \eqref{dieuprimo}. Nell'altro caso il massimo si ottiene per $|f(z)|=|z|$ e sostituendo si trova che è proprio $1$.
\end{proof}

\begin{thm} \label{distortion}
  Dato $b \in [0,1)$, scriviamo $F_b(z)=\dfrac{z(z+b)}{1+b z}$. Consideriamo $f \in \normalfont{\text{Hol}}(\mathbb{D},\mathbb{D})$ tale che $f(0)=0$. Se $|f'(0)|<1$, allora per ogni $z \in \mathbb{D}$ si ha
  \begin{equation}
    \left|\frac{f^h(0)-f^h(z)}{1-\overline{f^h(0)}f^h(z)}\right| \le \frac{2|z|}{1+|z|^2}
  \end{equation}
  e
  \begin{equation}
    F_{|f^h(0)|}^h(-|z|) \le |f^h(z)| \le F_{|f^h(0)|}^h(|z|).
  \end{equation}
\end{thm}

\begin{proof}
  Poiché $|f'(0)|<1$, per il lemma di Schwarz si ha $f \not\in \text{Aut}(\mathbb{D})$. Inoltre $f(0)=0$, perciò possiamo applicare il Corollario \ref{36}; si ha dunque
  $$\omega\bigl(f^h(0),f^h(z)\bigr) \le 2\omega(0,z).$$
  Applicando la tangente iperbolica, sfruttando l'uguaglianza $\tanh(2x)=\frac{2\tanh{x}}{1+\tanh^2{x}}$ e ricordando la definizione di $\omega$ si ha
  $$p\bigl(f^h(0),f^h(z)\bigr) \le \frac{2p(0,z)}{1+p^2(0,z)},$$
  da cui
  $$\left|\frac{f^h(0)-f^h(z)}{1-\overline{f^h(0)}f^h(z)}\right| \le \frac{2|z|}{1+|z|^2}.$$

  Per dimostrare la seconda disuguaglianza, supponiamo dapprima che si abbia $f^h(0)=b \in [0,1)$. Possiamo ripetere i passaggi svolti nella dimostrazione del lemma di Dieudonné ponendo $a=f^h(z)$ e $r=\frac{2|z|}{1+|z|^2}$. Otteniamo la disuguaglianza $|f^h(z)-\alpha| \le R$, dove $\alpha=\dfrac{b(1-r^2)}{1-r^2b^2}$ e $R^2=\dfrac{r^2-b^2}{1-r^2b^2}+\alpha^2$. Sostituendo troviamo
  $$\alpha=\frac{b(1-|z|^2)^2}{(1+2b|z|+|z|^2)(1-2b|z|+|z|^2)},$$
  $$R=\frac{2|z|(|z|^2+1)(1-b^2)}{(1+2b|z|+|z|^2)(1-2b|z|+|z|^2)}.$$
  Consideriamo adesso $F_b^h(z)=\dfrac{bz^2+2z+b}{|z|^2+2b\,\mathfrak{Re}z+1}\left(\dfrac{|1+b z|}{1+b z}\right)^2$. Si ha
  $$F_b^h(|z|)=\dfrac{b|z|^2+2|z|+b}{|z|^2+2b|z|+1} \text{ e } F_b^h(-|z|)=\dfrac{b|z|^2-2|z|+b}{|z|^2-2b|z|+1}.$$
  Notiamo che $\alpha=\bigl(F_b^h(|z|)+F_b^h(-|z|)\bigr)/2$ e $R=\bigl(F_b^h(|z|)-F_b^h(-|z|)\bigr)/2$, perciò la disuguaglianza $|f^h(z)-\alpha| \le R$ ci dice che $f^h(z)$ appartiene al cerchio con diametro sull'asse reale passante per i punti $F_b^h(|z|)$ e $F_b^h(-|z|)$. Con semplici considerazioni geometriche otteniamo la seguente disuguaglianza:
  $$F_b^h(-|z|) \le \mathfrak{Re}f^h(z) \le |f^h(z)| \le F_b^h(|z|),$$
  la quale, ricordando che $b=f^h(0)$, ci dà
  $$F_{f^h(0)}^h(-|z|) \le |f^h(z)| \le F_{f^h(0)}^h(|z|).$$
  Per passare al caso generale consideriamo la funzione $g(z)=|f^h(0)|f(z)/f'(0)$. Osserviamo che $f(0)=0$ ci dice che $f'(0)=f^h(0)$, dunque $|g(z)|=|f(z)|$ e $|g'(z)|=|f'(z)|$, pertanto $|g^h(z)|=|f^h(z)|$; inoltre si ha anche $g(0)=0$, da cui $g^h(0)=g'(0)=|f^h(0)|$. Perciò applicando l'ultima disuguaglianza trovata alla funzione $g$ otteniamo proprio la seconda disuguaglianza della tesi.
\end{proof}

\begin{cor} \label{distorto}
  Sia $f \in \text{Hol}(\mathbb{D},\mathbb{D})$ tale che $f(0)=0$ e $f'(0) \in [0,1)$. Allora $\mathfrak{Re}f'(z)>0$ per $|z|<f^h(0)/\Bigl(1+\sqrt{1-\bigl(f^h(0)\bigr)^2}\Bigr)$.
\end{cor}

\begin{proof}
  Per $0 \le b<1$ e $z \in \mathbb{D}$ si ha $|z|^2-2b|z|+1>|z|^2-2|z|+1>0$, dunque abbiamo che il segno di $F_b^h(-|z|)$ coincide con quello di $b|z|^2-2|z|+b$. Quest'ultima quantità è minore di $0$ per $|z| \in \bigl((1-\sqrt{1-b^2})/b, (1+\sqrt{1-b^2})/b\bigr)$, zero agli estremi e maggiore di $0$ altrove.
  Prendendo $b=f'(0)=f^h(0)$, nella dimostrazione del Teorema \ref{distortion} abbiamo visto che $\mathfrak{Re}f^h(z) \ge F_{f^h(0)}^h(-|z|)$; per gli $z$ tali che $|z|<\Bigl(1-\sqrt{1-\bigl(f^h(0)\bigr)^2}\Bigr)/f^h(0)=f^h(0)/\Bigl(1+\sqrt{1-\bigl(f^h(0)\bigr)^2}\Bigr)$ si ha quindi $\mathfrak{Re}f^h(z)>0$.
  Ricordando che $f^h(z)=\frac{f'(z)(1-|z|^2)}{1-|f(z)|^2}$ e $f \in \text{Hol}(\mathbb{D},\mathbb{D})$, per tali $z$ si ha anche $\mathfrak{Re}f'(z)>0$.
\end{proof}

Vediamo ora il risultato che, come già anticipato, ci permetterà di dimostrare i due teoremi di rigidità. L'enunciato originale si trova in \cite{GMG}, ma vedremo una formulazione che ci tornerà più utile, in particolare perché coinvolge la funzione $f^h$.

\begin{thm} \label{golusin}
  (disuguaglianza di Golusin, 1945) Sia $f \in \text{\normalfont{Hol}}(\mathbb{D},\mathbb{D})\setminus\text{\normalfont{Aut}}(\mathbb{D})$. Allora per ogni $z \in \mathbb{D}$ vale
  \begin{equation} \label{gol}
    |f^h(z)| \le \frac{|f^h(0)|+\frac{2|z|}{1+|z|^2}}{1+|f^h(0)|\frac{2|z|}{1+|z|^2}}.
  \end{equation}
\end{thm}

\begin{proof}
  Con passaggi analoghi a quelli della dimostrazione del Corollario \ref{quasigolusin} abbiamo che valgono le seguenti uguaglianze:
  \begin{gather*}
    \omega\bigl(|f^h(z)|,|f^h(0)|\bigr)=\left|\frac{1}{2}\log\left(\frac{1+|f^h(z)|}{1-|f^h(z)|}\cdot\frac{1-|f^h(0)|}{1+|f^h(0)|}\right)\right|\text{ e}\\
    \omega(z, 0)=\omega(|z|,0)=\frac{1}{2}\log\left(\frac{1+|z|}{1-|z|}\right),
  \end{gather*}
  dove nella prima abbiamo preso il modulo perché non possiamo più imporre senza perdita di generalità $|f^h(z)| \ge |f^h(0)|$. Ponendo $w=0$ nella disuguaglianza \eqref{quasigol} otteniamo
  \begin{align*}
    \frac{1}{2}\log\left(\frac{1+|f^h(z)|}{1-|f^h(z)|}\cdot\frac{1-|f^h(0)|}{1+|f^h(0)|}\right) & \le \omega\bigl(|f^h(z)|,|f^h(0)|\bigr) \\
    & \le 2\omega(z,0)=\log\left(\frac{1+|z|}{1-|z|}\right),
  \end{align*}
  da cui
  \begin{equation}
    \frac{1+|f^h(z)|}{1-|f^h(z)|} \le \frac{1+|f^h(0)|}{1-|f^h(0)|}\left(\frac{1+|z|}{1-|z|}\right)^2. \label{golprimo}
  \end{equation}
  Adesso, dalla Proposizione \ref{24} sappiamo che $f^h(z),f^h(0) \in \mathbb{D}$, in particolare $|f^h(z)|,|f^h(0)|<1$, perciò è giustificato il seguente passaggio:
  \begin{align*}
    |f^h(z)| & \le \frac{\frac{1+|f^h(0)|}{1-|f^h(0)|}\left(\frac{1+|z|}{1-|z|}\right)^2-1}{\frac{1+|f^h(0)|}{1-|f^h(0)|}\left(\frac{1+|z|}{1-|z|}\right)^2+1} \\
    & =\frac{(1+|f^h(0)|)(1+2|z|+|z|^2)-(1-|f^h(0)|)(1-2|z|+|z|^2)}{(1+|f^h(0)|)(1+2|z|+|z|^2)+(1-|f^h(0)|)(1-2|z|+|z|^2)} \\
    & =\frac{2|f^h(0)|+2|f^h(0)||z|^2+4|z|}{2+2|z|^2+4|f^h(0)||z|}=\frac{|f^h(0)|+\frac{2|z|}{1+|z|^2}}{1+|f^h(0)|\frac{2|z|}{1+|z|^2}}.
  \end{align*}
\end{proof}


\subsubsection{Il teorema di Pick-Nevanlinna}
Dedichiamo una sottosezione al seguente risultato, dimostrato indipendentemente da Pick nel 1916 \cite{P} e Nevanlinna nel 1919; è un teorema di interpolazione interessante di per sé, inoltre vedremo un paio di esempi in cui le ipotesi vengono riformulate in termini del lemma di Schwarz-Pick, usando in un caso anche il rapporto iperbolico. Seguiamo la dimostrazione vista in \cite[Chapter 1, Theorem 2.2]{JBG}.
\marginpar{le fonti dicono 1916, ma andando a cercare l'articolo originale di Pick è del 1915. Quello di Nevanlinna è introvabile. Mah, mistero!}

\begin{thm}
  (Pick-Nevanlinna) Siano dati $n$ punti distinti $z_1, \dots, z_n \in \mathbb{D}$ e altri $n$ punti distinti (non necessariamente diversi dai primi) $w_1, \dots, w_n \in \mathbb{D}$. Consideriamo la forma quadratica
  $$A_n(t_1,\dots,t_n)=\sum_{i,j=1}^n\frac{1-w_i\bar{w}_j}{1-z_i\bar{z}_j}t_i\bar{t}_j.$$
  Allora esiste una funzione $f \in \normalfont{\text{Hol}}(\mathbb{D},\mathbb{D})$ tale che $f(z_i)=w_i$ per $j=1, \dots, n$ se e solo se $A_n$ è semidefinita positiva. In tal caso, si può trovare $f$ che sia un prodotto di Blaschke di grado al più $n$.
\end{thm}

\begin{proof}
  Procediamo per induzione su $n$. Il caso $n=1$ è banale per transitività di $\text{Aut}(\mathbb{D})$. Supponiamo adesso $n>1$. Poniamo
  $$z_i'=\frac{z_i-z_n}{1-\bar{z}_nz_i} \,\, \text{e} \,\, w_i'=\frac{w_i-w_n}{1-\bar{w}_nw_i} \,\, \text{per} \,\, 1 \le i \le n.$$
  Allora esiste $f$ olomorfa dal disco in sé che risolve l'interpolazione se e solo se la funzione
  $$g(z)=\frac{\Bigg(f\left(\dfrac{z+z_n}{1+\bar{z}_nz}\right)-w_n\Bigg)}{1-\bar{w}_n\Bigg(f\left(\dfrac{z+z_n}{1+\bar{z}_nz}\right)\Bigg)}$$
  appartiene a $\text{Hol}(\mathbb{D},\mathbb{D})$ e soddisfa $g(z_i')=w_i'$ per $1 \le i \le n$. Infatti, si tratta solo di comporre con i giusti automorfismi. La forma quadratica $A_n'$ definita con i punti $z_i',w_i'$ è legata alla forma $A_n$. Per mostrarlo, poniamo
  $$\frac{1-z_i'\bar{z}_j'}{1-z_i\bar{z}_j}=\frac{1-|z_n|^2}{(1-\bar{z}_nz_i)(1-z_n\bar{z}_j)}=\alpha_i\bar{\alpha}_j$$
  e
  $$\frac{1-w_i'\bar{w}_j'}{1-w_i\bar{w}_j}=\frac{1-|w_n|^2}{(1-\bar{w}_nw_i)(1-w_n\bar{w}_j)}=\beta_i\bar{\beta}_j,$$
  dove $\alpha_i=\frac{\sqrt{1-|z_n|^2}}{1-\bar{z}_nz_i}$ per $1 \le i \le n$ e analogamente per i $\beta_i$. Allora si ha
  \begin{align*}
    A_n'(t_1,\dots,t_n) & =\sum_{i,j=1}^n \frac{1-w_i'\bar{w}_j'}{1-z_i'\bar{z}_j'}t_i\bar{t}_j \\
    & =\sum_{i,j=1}^n \frac{1-w_i\bar{w}_j}{1-z_i\bar{z}_j}\left(\frac{\beta_i}{\alpha_i}t_i\right)\left(\frac{\bar{\beta}_j}{\bar{\alpha}_i}\bar{t}_j\right)=A_n\left(\frac{\beta_1}{\alpha_1}t_1,\dots,\frac{\beta_n}{\alpha_n}t_n\right).
  \end{align*}
  Poiché $z_i,w_i \in \mathbb{D}$ si ha $\alpha_i,\beta_i\not=0$; perciò $A_n$ è semidefinita positiva se e solo se lo è $A_n'$. Dato che $z_n'=w_n'=0$, a meno di cambiare $f$ con $g$ possiamo supporre senza perdita di generalità $z_n=w_n=0$. La condizione dell'enunciato diventa dunque $f(0)=0$ e $f(z_i)=w_i$ per $1 \le i \le n-1$.
  Tale funzione esiste se e solo se la funzione $h(z)=f(z)/z$, con $h(0)=f'(0)$, appartiene a $\text{Hol}(\mathbb{D},\mathbb{D})$ e soddisfa $h(z_i)=w_i/z_i$ per $1 \le i \le n-1$. Per il punto (ii) della proposizione \ref{blaschke-prop} con $w=0$, abbiamo anche che $f \in \mathcal{B}_d$ se e solo se $h \in \mathcal{B}_{d-1}$.
  Vogliamo dire che $A_n$ è semidefinita positiva se e solo se la forma quadratica $A_n''$, costruita usando i punti $z_i, w_i/z_i$, è semidefinita positiva. Dato che $z_n=w_n=0$, completando il quadrato abbiamo
  \begin{align*}
    A_n(t_1,\dots,t_n)&=|t_n|^2+2\cdot\mathfrak{Re}\sum_{i=1}^{n-1}\bar{t}_it_n+\sum_{i,j=1}^{n-1}\frac{1-w_i\bar{w}_j}{1-z_i\bar{z}_j}t_i\bar{t}_j \\
    &=\left|\sum_{i=1}^n t_i\right|^2+\sum_{i,j=1}^{n-1}\left(\frac{1-w_i\bar{w}_j}{1-z_i\bar{z}_j}-1\right)t_i\bar{t}_j \\
    &=\left|\sum_{i=1}^n t_i\right|^2+\sum_{i,j=1}^{n-1}\frac{1-(w_i/z_i)\overline{(w_j/z_j)}}{1-z_i\bar{z}_j}z_i\bar{z}_jt_i\bar{t}_j,
  \end{align*}
  quindi
  $$A_n(t_1,\dots,t_n)=\left|\sum_{i=1}^n t_i\right|^2+A_n''(z_1t_1,\dots,z_{n-1}t_{n-1}).$$
  Allora se $A_n''$ è semidefinita positiva lo è anche $A_n$, mentre per l'implicazione opposta basta prendere $\displaystyle t_n=-\sum_{i=1}^n t_i$ (ricordiamo che per ipotesi $z_i\not=z_n=0$ per $1 \le i \le n-1$).
\end{proof}

Vediamo che il caso $n=2$ è equivalente alla disuguaglianza del lemma di Schwarz-Pick. La forma quadratica $A_2$ è semidefinita positiva se e solo se i determinanti dei minori principali della matrice associata, che è hermitiana, sono non negativi. Per ipotesi $\frac{1-|w_1|^2}{1-|z_1|^2} \ge 0$, mentre per il determinante della matrice abbiamo
$$\frac{1-|w_1|^2}{1-|z_1|^2}\cdot\frac{1-|w_2|^2}{1-|z_2|^2}-\left|\frac{1-w_1\bar{w}_2}{1-z_1\bar{z}_2}\right|^2 \ge 0,$$
ovvero
$$\frac{|1-z_1\bar{z}_2|^2}{(1-|z_1|^2)(1-|z_2|^2)} \ge \frac{|1-w_1\bar{w}_2|^2}{(1-|w_1|^2)(1-|w_2|^2)};$$
riarrangiando i denominatori si ottiene
$$\frac{|1-z_1\bar{z}_2|^2}{|1-\bar{z}_2z_1|^2-|z_1-z_2|^2} \ge \frac{|1-w_1\bar{w}_2|^2}{|1-\bar{w}_2w_1|^2-|w_1-w_2|^2},$$
che è equivalente a
\begin{align*}
  \frac{1}{1-\left|\frac{z_1-z_2}{1-\bar{z}_2z_1}\right|^2} & \ge \frac{1}{1-\left|\frac{w_1-w_2}{1-\bar{w}_2w_1}\right|^2} \\
  & \Leftrightarrow \frac{1}{1-p^2(w_1,w_2)} \le \frac{1}{1-p^2(z_1,z_2)} \Leftrightarrow \omega(w_1,w_2) \le \omega(z_1,z_2).
\end{align*}
Per invarianza di $p$, e dunque anche di $\omega$, sotto l'azione di $\text{Aut}(\mathbb{D})$, si può porre $w_1=z_1=0$ e la funzione di interpolazione si trova immediatamente.

Andiamo adesso a ridimostrare il caso $n=3$ con una formulazione differente; otteniamo così una sorta di inverso del Teorema \ref{31}.

\begin{thm}
  Siano $z_1, z_2, z_3$ e $w_1, w_2, w_3$ due triple di punti distinti in $\mathbb{D}$. Allora esiste $f \in \normalfont{\text{Hol}}(\mathbb{D},\mathbb{D}) \setminus \normalfont{\text{Aut}}(\mathbb{D})$ tale che $f(z_i)=w_i$ per $i=1,2,3$ se e solo se valgono le seguenti condizioni:
  \begin{nlist}
    \item $\omega(w_i,w_j)<\omega(z_i,z_j)$ per $i,j=1,2,3$ e $i\not=j$;
    \item $\omega\left(\dfrac{[w_2,w_1]}{[z_2,z_1]},\dfrac{[w_3,w_1]}{[z_3,z_1]}\right) \le \omega(z_2,z_3)$.
  \end{nlist}
\end{thm}

\begin{proof}
  Supponiamo che esista siffatta $f$. Allora la condizione (i) segue dal lemma di Schwarz-Pick. La condizione (ii) invece si riscrive come $\omega\bigl(f^*(z_2,z_1),f^*(z_3,z_1)\bigr) \le \omega(z_2,z_3)$, che è l'enunciato del Teorema \ref{31}.

  Adesso dimostriamo l'altra freccia. Vediamola prima nel caso $z_1=w_1=0$. Allora per la condizione (i) abbiamo $\omega(0,w_i) < \omega(0,z_i)$, quindi $|w_i/z_i|<1$ per $i=2,3$. La condizione (ii) si riscrive invece come $\omega(w_2/z_2,w_3/z_3) \le \omega(z_2,z_3)$, cioè $p(w_2/z_2,w_3/z_3) \le p(z_2,z_3)$.
  Dunque, per il caso $n=2$ del teorema di Pick-Nevanlinna, esiste $g \in \text{Hol}(\mathbb{D},\mathbb{D})$ tale che $g(z_2)=w_2/z_2$ e $g(z_3)=w_3/z_3$. Allora basta prendere $f(z)=zg(z)$.

   Mostriamo che ci si può ridurre a questo caso. Consideriamo $h, g \in \text{Aut}(\mathbb{D})$ date da
   $$g(z)=\frac{z-z_1}{1-\bar{z}_1z} \,\, \text{e} \,\, h(z)=\frac{z-w_1}{1-\bar{w}_1z}.$$
   Allora esiste $f$ come quella richiesta dal Teorema se e solo se esiste $F \in \text{Hol}(\mathbb{D},\mathbb{D})$, con $F=h \circ f \circ g^{-1}$, tale che $F(0)=0$, $F\bigl(g(z_2)\bigr)=h(w_2)$ e $F\bigl(g(z_3)\bigr)=h(w_3)$.
   Questo corrisponde proprio al caso precedente, quindi tale $F$ esiste se e solo se
   $$\omega\bigl(h(w_i),h(w_j)\bigr) \le \omega\bigl(g(z_i),g(z_j)\bigr)$$
   per $i,j=1,2,3$ con $i\not=j$ e
   $$\omega\left(\frac{h(w_2)}{g(z_2)},\frac{h(w_3)}{g(z_3)}\right) \le \omega\bigl(g(z_2),g(z_3)\bigr).$$
   Poiché $\omega$ è invariante per azione di $\text{Aut}(\mathbb{D})$, la prima disuguaglianza è equivalente alla condizione (i). Sempre per questo motivo, sostituendo $h(z)=[z,w_1]$ e $g(z)=[z,z_1]$ otteniamo che la seconda è equivalente alla condizione (ii).
\end{proof}

L'argomento viene trattato più in dettaglio in \cite{BRW}, dove il caso $n$ generico viene studiato usando i rapporti iperbolici iterati.


\newpage

\section{Dalla disuguaglianza di Golusin al teorema di Burns-Krantz}

\subsection{Rigidità al bordo}
Dalla parte di unicità del lemma di Schwarz-Pick possiamo dedurre il seguente risultato di rigidità: sia $f \in \text{Hol}(\mathbb{D},\mathbb{D})$ tale che
$$f(z)=z+o(z-z_0) \text{ per } z \longrightarrow z_0 \in \mathbb{D};$$
allora $f$ è l'identità. Lo si può dimostrare osservando che $f'(z_0)=1$ e usando quest'altro risultato più forte, che è una diretta conseguenza del lemma di Schwarz-Pick: se $f \in \text{Hol}(\mathbb{D},\mathbb{D})$ è tale che
$$|f^h(z)|=1+o(1) \text{ per } z \longrightarrow z_0 \in \mathbb{D},$$
allora $f \in \text{Aut}(\mathbb{D})$. A questo punto, per dimostrare il primo enunciato basta supporre $z_0=0$, a meno di comporre con opportuni automorfismi.
\marginpar{devo approfondire con i dettagli o va bene questa spiegazione?}

Questi sono risultati di rigidità piuttosto forti. Ad esempio, il primo può essere riassunto così: se una funzione olomorfa dal disco in sé dista dall'identità per termini di grado minore al primo (e ovviamente non potremmo chiedere di meglio), allora è forzata a essere l'identità. Il secondo è simile: se la derivata iperbolica di una funzione si comporta come quella di un automorfismo (cioè la funzione e la sua derivata si comportano come quelle di un automorfismo), allora la funzione è necessariamente un automorfismo.

Questi sono risultati che valgono per punti interni al disco; quello che andremo a fare in questa sezione e nella successiva è dimostrare i loro analoghi per punti sul bordo, andando a perdere due gradi nel resto delle ipotesi. Useremo la disuguaglianza di Golusin, seguendo la traccia data nel Remark 5.6 di \cite{BKR}.

\begin{thm} \label{boundary_schwarz_pick}
  (Bracci-Kraus-Roth, 2020) Sia $f \in \text{\normalfont{Hol}}(\mathbb{D},\mathbb{D})$ tale che
  \begin{equation} \label{n_o^2}
    |f^h(z_n)|=1+o\bigl((|z_n|-1)^2\bigr)
  \end{equation}
  per qualche successione $\{z_n\}_{n \in \mathbb{N}} \subset \mathbb{D}$ con $|z_n| \longrightarrow 1$. Allora $f \in \text{\normalfont{Aut}}(\mathbb{D})$.
\end{thm}

\begin{proof}
  Supponiamo per assurdo che $f \not\in \text{Aut}(\mathbb{D})$. Possiamo applicare la disuguaglianza di Golusin \ref{golusin} nella forma \eqref{golprimo}, che riscriviamo come
  $$\frac{1+|f^h(0)|}{\bigl(1-|f^h(0)|\bigr)\bigl(1+|f^h(z_n)|\bigr)}\bigl(1-|f^h(z_n)|\bigr) \ge \frac{(1-|z_n|)^2}{(1+|z_n|)^2}.$$
  Per ipotesi vale \eqref{n_o^2}, dunque
  $$\frac{1+|f^h(0)|}{\bigl(1-|f^h(0)|\bigr)\bigl(1+|f^h(z_n)|\bigr)}o\bigl((|z_n|-1)^2\bigr) \ge \frac{(1-|z_n|)^2}{(1+|z_n|)^2}$$
  da cui
  $$\frac{\bigl(1+|f^h(0)|\bigr)(1+|z_n|)^2}{\bigl(1-|f^h(0)|\bigr)\bigl(1+|f^h(z_n)|\bigr)}o(1) \ge 1.$$
  Se $f \not\in \text{Aut}(\mathbb{D})$, per il lemma di Schwarz-Pick si ha necessariamente $|f^h(0)|<1$, dunque $\displaystyle \lim_{n \longrightarrow +\infty} \frac{\bigl(1+|f^h(0)|\bigr)(1+|z_n|)^2}{\bigl(1-|f^h(0)|\bigr)\bigl(1+|f^h(z_n)|\bigr)}=\frac{2\bigl(1+|f^h(0)|\bigr)}{1-|f^h(0)|} < +\infty$ e otteniamo una contraddizione.
\end{proof}

Siamo ora pronti a dimostrare il Theorem 2.1 di \cite{BK}.


\subsection{Teorema di Burns-Krantz}
\begin{thm} \label{burns_krantz}
  (Burns-Krantz, 1994) Siano $f \in \text{\normalfont{Hol}}(\mathbb{D},\mathbb{D})$ e $\sigma \in \partial\mathbb{D}$ tali che
  \begin{equation} \label{o^3bis}
    f(z)=\sigma+(z-\sigma)+o\bigl((z-\sigma)^3\bigr)
  \end{equation}
  per $z \longrightarrow \sigma$ non tangenzialmente. Allora $f$ è l'identità del disco.
\end{thm}

\begin{proof}
  Come già visto nella dimostrazione della Proposizione \ref{o^3->o^2}, a meno di considerare $\sigma^{-1}f(\sigma z)$ possiamo supporre senza perdita di generalità $\sigma=1$.
  Dalla Proposizione \ref{o^3->o^2} segue anche che vale
  $$|f^h(z)|=1+o\bigl((z-1)^2\bigr)$$
  per $z \longrightarrow 1$ non tangenzialmente, quindi esiste una successione $z_n$ che soddisfa le ipotesi del Teorema \ref{boundary_schwarz_pick} (usiamo di nuovo che, non tangenzialmente, $|z-1|$ e $1-|z|$ possono essere scambiati negli $o$-piccoli); dunque $f$ è un automorfismo.
  Per la Proposizione \ref{aut} esistono $\theta \in \mathbb{R}$ e $a \in \mathbb{D}$ tali che $f(z)=e^{i\theta}\dfrac{z-a}{1-\bar{a}z}$. Poiché vale \eqref{o^3bis}, dev'essere $f''(1)=0$. Si ha $f''(z)=\dfrac{e^{i\theta}\bar{a}(1-|a|^2)}{(1-\bar{a}z)^3}$;
  siccome $e^{i\theta}\not=0$ e $|a|<1$, deve necessariamente essere $\bar{a}=0$, perciò $f(z)=e^{i\theta}z$. Il fatto che $f(z)=z$ segue da $f(1)=1$ sempre per \eqref{o^3bis}.
\end{proof}

\begin{ex}
  Sia $f:\mathbb{C}\setminus\{\pm i\sqrt{3}\} \longrightarrow \mathbb{C}$ data da $f(z)=\dfrac{1+3z^2}{3+z^2}$. Verifichiamo che $f(\mathbb{D}) \subset \mathbb{D}$. Se $z \in \mathbb{D}$ allora $|z|<1$ da cui $1-|z|^4>0$.
  Dunque abbiamo $1-|z|^4 < 9(1-|z|^4)$ che riscriviamo come $1+9|z|^4 < 9+|z|^4$, perciò
  \begin{align*}
    (1+3z^2)(1+3\bar{z}^2) &=1+3z^2+3\bar{z}^2+9|z|^4 \\
    & < 9+3z^2+3\bar{z}^2+|z|^4=(3+z^2)(3+\bar{z}^2),
  \end{align*}
  quindi
  $$|f(z)|^2=\frac{(1+3z^2)(1+3\bar{z}^2)}{(3+z^2)(3+\bar{z}^2)} < 1$$
  e l'ultima disuguaglianza ci dice che $|f(z)|<1$, cioè $f(z) \in \mathbb{D}$.

  Ovviamente $f$ non può essere iniettiva su $\mathbb{D}$ perché $f(z)=f(-z)$; dunque non è un automorfismo. Adesso mostriamo che $f(z)-1-(z-1)$ è $O\bigl((z-1)^3\bigr)$ ma non $o\bigl((z-1)^3\bigr)$ per $z \longrightarrow 1$:
  \begin{align*}
    g(z) & := f(z)-z=\frac{1+3z^2}{3+z^2}-z \\
    & =\frac{1+3z^2-3z-z^3}{3+z^2}=\frac{(1-z)^3}{3+z^2}.
  \end{align*}
  Poiché $\displaystyle \lim_{z \longrightarrow 1} g(z)/(z-1)^3=-1/4$ si ha che $g(z)$ è $O\bigl((z-1)^3\bigr)$ ma non $o\bigl((z-1)^3\bigr)$ per $z \longrightarrow 1$. Allora il termine $o\bigl((z-\sigma)^3\bigr)$ nel Teorema \ref{burns_krantz} non è migliorabile.
\end{ex}

\begin{oss}
  Lo stesso esempio mostra che anche nel Teorema \ref{boundary_schwarz_pick} il termine $o\bigl((|z_n|-1)^2\bigr)$ non può essere migliorato. Si ha infatti
  $$|f^h(z)|=\frac{|f'(z)|(1-|z|^2)}{1-|f(z)|^2}=\frac{2|z|}{1+|z|^2},$$
  da cui $\displaystyle \lim_{|z| \longrightarrow 1} \frac{|f^h(z)|-1}{(|z|-1)^2}=-\frac{1}{2}$; allora la quantità $|f^h(z)|-1$ è $O\bigl((|z|-1)^2\bigr)$ ma non $o\bigl((|z|-1)^2\bigr)$ per qualunque successione che si avvicina al bordo, e abbiamo già visto che $f$ non è un automorfismo.
\end{oss}


\newpage

\begin{thebibliography}{widest entry}
  \bibitem[BK]{BK} D. M. Burns, S. G. Krantz: Rigidity of holomorphic mappings and a new Schwarz lemma at the boundary. \textit{Journal of the American Mathematical Society}, \textbf{7} (1994), no. 3, 661--676
  \bibitem[BKR]{BKR} F. Bracci, D. Kraus, O. Roth: A new Schwarz-Pick Lemma at the boundary and rigidity of holomorphic maps. Preprint, ArXiv:2003.02019v1 (2020)
  \bibitem[BM]{BM} A. F. Beardon, D. Minda: A multi-point Schwarz-Pick lemma. \textit{Journal d'Analyse Mathématique}, \textbf{92} (2004), 81--104
  \bibitem[BRW]{BRW} L. Baribeau, P. Rivard, E. Wegert: On Hyperbolic Divided Differences and the Nevanlinna-Pick Problem. \textit{ Computational Methods and Function Theory}, \textbf{9} (2009), no. 2, 391--405
  \bibitem[D]{D} J. Dieudonné: Recherches sur quelques problèmes relatifs aux polynômes et aux fonctions bornées d'une variable complexe. \textit{Annales Scientifiques de l'École Normale Supérieure}, \textbf{48} (1931), 247--358
  \bibitem[GMG]{GMG} G. M. Golusin: Some estimations of derivatives of bounded functions. \textit{Recueil Mathématique [Matematicheskiĭ Sbornik]}, \textbf{16(58)} (1945), no. 3, 295--306
  \bibitem[JBG]{JBG} J. B. Garnett: \textbf{Bounded Analytic Functions (Revised First Edition)}. Springer, New York, 2007
  \bibitem[N]{N} R. Nevanlinna: Über beschränkte Funktionen, die in gegebenen Punkten vorgeschriebene Werte annehmen. \textit{Annales Academiae Scientiarum Fennicae, Series A}, \textbf{13} (1919) no. 1
  \bibitem[NN]{NN} R. Narasimhan, Y. Nievergelt: \textbf{Complex analysis in one variable (2nd edition)}. Springer, New York, 2001
  \bibitem[P]{P} G. Pick: Über die Beschränkungen analytischer Funktionen, welche durch vorgegebene Funktionswerte bewirkt werden. \textit{ Mathematische Annalen}, \textbf{77} (1915), no. 1, 7--23
\end{thebibliography}


\section*{Ringraziamenti}
\addcontentsline{toc}{section}{Ringraziamenti}
Inizio ringraziando il mio relatore, il professor Marco Abate, per un'infinità di cose: l'argomento proposto, l'attenzione ai dettagli, tutti i consigli e gli insegnamenti su come scrivere matematica con un'esposizione chiara ed efficace, oltre che corretta, e su come presentarla ponendo l'attenzione sui passaggi importanti e le idee dietro le dimostrazioni.

Ciò che ho imparato durante la stesura di questa tesi mi sarà di grande aiuto negli anni a venire. \\

Ringrazio anche i compagni con cui ho passato questi tre anni, in particolare Maria Chiara, Lucrezia, Giorgio, Federico e infine Alessio, con il quale ho trascorso più tempo di tutti tra dispense, computer e videogiochi.

Abbiamo studiato insieme, ma soprattutto ci siamo divertiti. È anche grazie a voi se gli ultimi anni resteranno sempre tra i migliori della mia vita. \\

Per finire, ringrazio tutti i miei parenti: loro, a differenza mia, non hanno mai dubitato delle mie capacità.

Un grazie speciale ai miei genitori: mi hanno sempre sostenuto, ma hanno anche dovuto sopportarmi per mesi causa pandemia; mi conosco, e ancora non mi capacito di come siano riusciti nell'impresa. Vi ringrazio con tutto il cuore!


\end{document}
