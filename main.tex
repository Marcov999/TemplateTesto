\documentclass{article}
\usepackage{mstyle}
\usepackage{pgfplots}
\usetikzlibrary{intersections, pgfplots.fillbetween}

\title{\begin{figure}[t!]
  \centering
  \includegraphics[trim=0 55 0 60, clip, width=0.5\textwidth]{Stemma_unipi.jpg}
\end{figure}
\vspace{-17.5mm}
\textsc{\Large Università di Pisa}\\
\textsc{\large Corso di Laurea in Matematica}\\
\, \\
{\large Tesi di Laurea Triennale}\\
\, \\
Teoremi di rigidità per funzioni olomorfe nel disco}
\author{Candidato:  \hspace{200px} Relatore:\\
\textbf{Marco Vergamini} \hfill Prof. \textbf{Marco Abate}}
\date{}

\begin{document}
\maketitle
\vspace*{\fill}
\begin{center}
  24 settembre 2021
  \par\noindent\rule{\textwidth}{0.5pt}
  \Large Anno Accademico 2020/2021
\end{center}
\newpage
\tableofcontents
\newpage


\section*{Introduzione}
\addcontentsline{toc}{section}{Introduzione}
L'obiettivo di questo scritto è dimostrare un teorema del 1994, il teorema di Burns-Krantz (teorema 2.1 di \cite{BK}), attraverso risultati elementari. L'enunciato del teorema riguarda le funzioni olomorfe sul disco unitario con un certo andamento vicino a $1$: se la funzione dista dall'identità al più per un $\mathcal{O}((z-1)^4)$, allora è proprio l'identità.

La dimostrazione originale del teorema non è lunga, ma un po' tecnica. In un recente articolo di Bracci, Kraus e Roth (\cite{BKR}) si trova una dimostrazione alternativa del teorema di Burns-Krantz. Come spiegato nel remark 2.2 dell'articolo, è possibile passare dalle ipotesi del teorema di Burns-Krantz a quelle del teorema 2.1 di \cite{BKR} (si veda la proposizione 8.1 dello stesso articolo), dal quale poi è facile concludere. Il teorema 2.1 è sostanzialmente una versione al bordo del lemma di Schwarz-Pick. \\

Bracci, Kraus e Roth dimostrano il teorema 2.1 usando risultati più generali visti nell'articolo, ma complicati. Tuttavia, nel remark 5.6 danno una traccia per una dimostrazione più elementare. L'idea è sfruttare una disuguaglianza dovuta a Golusin e vengono indicati vari articoli in cui è stata ridimostrata.

In particolare, l'articolo di Beardon e Minda del 2004 (\cite{BM}) dimostra una serie di disuguaglianze, delle quali il corollario 3.7 ha a sua volta come corollario la disuguaglianza di Golusin. Queste disuguaglianze coinvolgono la distanza iperbolica sul disco unitario. In effetti, i vari risultati più generali di \cite{BK} riguardano pseudometriche sul disco e lo stesso teorema 2.1 può essere riformulato in termini della distanza iperbolica, come spiegato nell'articolo.


\newpage

\section{Prerequisiti}

\subsection{Lemma di Schwarz-Pick e distanza di Poincaré}
\begin{defn}
  Sia $\Omega \subset \mathbb{C}$ un aperto. Una funzione $f:\Omega \longrightarrow \mathbb{C}$ si dice \textit{olomorfa} in $\Omega$ se è derivabile in senso complesso per ogni $z \in \Omega$.
\end{defn}

\begin{defn}
  Se $f: \Omega \longrightarrow \Omega$ olomorfa è biettiva, allora si può dimostrare che anche $f^{-1}$ è olomorfa. In tal caso $f$ è detta \textit{automorfismo} (in senso olomorfo di $\Omega$).
\end{defn}

Com'è noto, la condizione di olomorfia per funzioni a valori complessi è molto più forte della derivabilità in senso reale (in particolare, è equivalente all'analiticità). Fra i vari risultati che si possono dimostrare per le funzioni olomorfe, ci interessa studiare il lemma di Schwarz-Pick. \\

Notazione: indichiamo il disco unitario con $\mathbb{D}=\{z \in \mathbb{C} \mid |z|<1\}$.

\begin{lm}
  (Schwarz) Sia $f:\mathbb{D} \longrightarrow \mathbb{D}$ una funzione olomorfa t.c. $f(0)=0$. Allora per ogni $z \in \mathbb{D}$ $|f(z)| \le |z|$ e $|f'(0)| \le 1$; inoltre, se vale l'uguale nella prima per $z \not=0$ oppure nella seconda allora $f(z)=e^{i\theta}z, \theta \in \mathbb{R}$.
\end{lm}

\begin{lm}
  (Schwarz-Pick) Sia $f:\mathbb{D} \longrightarrow \mathbb{D}$ una funzione olomorfa.
  Allora per ogni $z, w \in \mathbb{D}$
  $$\left|\frac{f(z)-f(w)}{1-\overline{f(w)}f(z)}\right| \le \left|\frac{z-w}{1-\bar{w}z}\right|, \qquad \frac{|f'(z)|}{1-|f(z)|^2} \le \frac{1}{1-|z|^2}.$$
  Inoltre se vale l'uguale nella prima per $z_0, w_0$ con $z_0 \not=w_0$ o nella seconda per $z_0$ allora $f$ è un automorfismo e vale l'uguale sempre.
\end{lm}

Il lemma di Schwarz-Pick può essere riformulato usando due funzioni di due variabili sul disco, una delle quali è nota come distanza iperbolica (in realtà, anche l'altra è una distanza, ma questo non lo useremo). Con queste funzioni dimostreremo una serie di disuguaglianze che ci permetteranno di dimostrare la disuguaglianza di Golusin, dalla quale seguirà una versione al bordo del lemma.

\begin{defn}
  Dati $z, w \in \mathbb{D}$ poniamo
  $$[z,w]:=\frac{z-w}{1-\bar{w}z}, \qquad p(z,w):=|[z,w]|, \qquad d(z,w):=\log\left(\frac{1+p(z,w)}{1-p(z,w)}\right).$$
\end{defn}

$d$ è ben definita, in quanto $p(z,w)<1$. Infatti, dobbiamo verificare che
  $$\frac{|z-w|}{|1-\bar{w}z|} < 1$$
  $$|z-w|^2 < |1-\bar{w}z|^2$$
  $$|z|^2+|w|^2-\bar{w}-w\bar{z} < 1+|wz|^2-\bar{w}z-w\bar{z}$$
  $$1+|wz|^2-|z|^2-|w|^2 > 0$$
  $$(1-|w|^2)(1-|z|^2) > 0,$$
che è vera perché $z, w \in \mathbb{D}$.

\begin{prop}
  $d$ è una distanza (la sopracitata distanza iperbolica).
\end{prop}

\begin{proof}
  Mostriamo preliminarmente che $p$ è una distanza. In entrambi i casi, l'unica cosa non ovvia da controllare è la disuguaglianza triangolare. Perciò, dati $z_0, z_1, z_2 \in \mathbb{D}$, vogliamo $p(z_1,z_2) \le p(z_1,z_0)+p(z_0,z_2)$. Osserviamo che, per il lemma di Schwarz-Pick, $p$ è invariante applicando automorfismi, perciò supponiamo senza perdita di generalità $z_1=0$ (possiamo farlo perché il gruppo degli automorfismidi $\mathbb{D}$ è transitivo). A questo punto la disuguaglianza da dimostrare diventa
  \marginpar{Come si dimostra? Qui c'è una dim, ma ponendo $z_0=0$: https://mathoverflow.net/questions/21604/nice-proof-of-the-triangle-inequality-for-the-metric-of-the-hyperbolic-plane}
  \begin{align*}
    |z_2| \le |z_0|+\frac{|z_0-z_2|}{|1-\bar{z}_2z_0|}.
  \end{align*}
  (c'è da fare la dimostrazione) \\
  A questo punto, possiamo osservare che $d(z,w) =2\,\text{arctanh}\,(p(z,w))$, perciò... ACHTUNG: LA DIM DEGLI APPUNTI DI ECA SEMBRA ESSERE FALLACE, ARCTANH NON È SUBADDITIVA SUI POSITIVI
  %Altrimenti, si può brutalmente espandere il conto e ricondursi a una disuguaglianza la cui dimostrazione si trova qui: https://www.math3ma.com/blog/the-pseudo-hyperbolic-metric-and-lindelofs-inequality
\end{proof}

\begin{defn}
  Data una funzione $f: \mathbb{D} \longrightarrow \mathbb{D}$, poniamo
  $$f^*(z,w):=\frac{[f(z),f(w)]}{[z,w]}$$
  e
  $$f^h(z):=|f^*(z,z)|:=\left|\lim_{w \longrightarrow z} f^*(z,w)\right|=\left|\lim_{w \longrightarrow z} \frac{[f(z),f(w)]}{[z,w]}\right|=\frac{|f'(z)|(1-|z|^2)}{1-|f(z)|^2}.$$
\end{defn}

\begin{oss}
  \begin{nlist}
    \item la disuguaglianza del lemma di Schwarz-Pick può essere riscritta come $|f^*(z,w)| \le 1$;
    \item  un altro modo di scrivere la disuguaglianza del lemma di Schwarz-Pick è $p(f(z),f(w)) \le p(z,w)$;
    \item $p(0,z)=|z| \implies d(0,z)=d(0,|z|)$;
    \item per definizione, $|f^*(z,w)|=|f^*(w,z)|$ e $f^h(z)$ è reale non negativo.
  \end{nlist}
  Questi risultati verranno usati nelle varie dimostrazioni e verranno esplicitati solo quando ciò che ne segue non è immediato.
\end{oss}

Seguono alcuni risultati noti di analisi complessa noti che verranno usati nelle dimostrazioni.

\begin{thm}
  (formula integrale di Cauchy) Sia $f$ olomorfa sull'aperto $\Omega$ e $D$ un disco chiuso di centro $a$ contenuto in $\Omega$. Allora
  \begin{equation}
    f^{(n)}(a)=\frac{n!}{2\pi i} \int_{\partial D} \frac{f(\zeta)}{(\zeta-a)^{n+1}}\diff\zeta.
  \end{equation}
\end{thm}

\begin{prop}
  Sia $f$ olomorfa sull'aperto $\Omega \setminus\{z_0\}$, con $z_0 \in \Omega$. Allora $f$ si estende a una funzione olomorfa $g$ definita su tutto $\Omega$ se e solo se è limitata in un intorno di $z_0$. In tal caso, $z_0$ è detta \textit{singolarità rimovibile}.
\end{prop}


\subsection{Regioni di Stolz e limiti non tangenziali}
\begin{defn}
  Chiamiamo \textit{settore di vertice $1$ e angolo d'apertura $2\alpha$} l'insieme $S \subset \mathbb{D}$ tale che per ogni $z \in S$ l'angolo compreso tra l'asse reale e la retta congiungente $1$ e $z$ è minore di $\alpha$. \\
  Quando scriviamo $z \longrightarrow 1$ \textit{non tangenzialmente}, s'intende che stiamo considerando i punti appartenenti a un settore $S$ fissato.
  \begin{center}
    \begin{tikzpicture}[line cap=round,line join=round,>=triangle 45,x=2.5cm,y=2.5cm]
      \draw[->,color=black] (-1.13,0) -- (1.15,0);
      \foreach \x in {-1,1}
      \draw[shift={(\x,0)},color=black] (0pt,2pt) -- (0pt,-2pt);
      \draw[->,color=black] (0,-1.11) -- (0,1.12);
      \foreach \y in {-1,1}
      \draw[shift={(0,\y)},color=black] (2pt,0pt) -- (-2pt,0pt);
      \clip(-1.13,-1.11) rectangle (1.15,1.12);
      \draw(0,0) circle (2.5cm);
      \draw[name path=A] (0.49,0.87)-- (1,0);
      \draw[name path=B] (0.49,-0.87)-- (1,0);
      \tikzfillbetween[of=A and B]{blue, opacity=0.25};
      \draw[name path=C,smooth,samples=100,domain=-1:0.491] plot(\x,{sqrt(1-(\x)^2)});
      \draw[name path=D,smooth,samples=100,domain=-1:0.491] plot(\x,{0-sqrt(1-(\x)^2)});
      \tikzfillbetween[of=C and D]{blue, opacity=0.25};
    \end{tikzpicture}

    In blu, il settore di angolo d'apertura $\frac{2}{3}\pi$
  \end{center}
\end{defn}

La seguente proposizione asserisce che, per $z \longrightarrow 1$ non tangenzialmente, un certo andamento di $f$ può essere tradotto nell'andamento di $f^h$. È questo che ci permetterà di dimostrare il teorema 2.1 di \cite{BK} passando per la versione al bordo del lemma di Schwarz-Pick.

\begin{prop} \label{o^3->o^2}
  Sia $f:\mathbb{D} \longrightarrow \mathbb{D}$ una funzione olomorfa tale che
  \begin{equation} \label{o^3}
    f(z)=1+(z-1)+o((z-1)^3)
  \end{equation}
  per $z \longrightarrow 1$ non tangenzialmente. Allora
  \begin{equation} \label{o^2}
    f^h(z)=1+o((z-1)^2)
  \end{equation}
  per $z \longrightarrow 1$ non tangenzialmente.
\end{prop}

\begin{proof}

  Sia $S$ un settore di vertice $1$ e angolo d'apertura $2\alpha$, e $S'$ uno un po' più grande di vertice $1$ e angolo $2\beta$, $\beta>\alpha$. Per $z \in S$, sia $C(z)$ il cerchio di centro $z$ e raggio $r(z)=\text{dist}(z, \partial S')$ (la distanza di $z$ dal bordo di $S'$). Allora per la formula integrale di Cauchy
  \begin{align*}
    f'(z) & =\frac{1}{2\pi i} \int_{C(z)} \frac{f(w)}{(w-z)^2}\diff w \\
    & =\frac{1}{2\pi i} \int_{C(z)} \frac{w-1+(f(w)-w)}{(w-z)^2}\diff w \\
    & =\frac{1}{2\pi i} \int_{C(z)} \frac{1}{w-z}\diff w+\frac{1}{2\pi i} \int_{C(z)} \frac{z-1+f(w)-w}{(w-z)^2}\diff w \\
    & =1+\frac{1}{2\pi i} \int_{C(z)} \frac{f(w)-w}{(w-z)^2}\diff w=:1+I(z).
  \end{align*}

  \begin{center}
    \definecolor{qqffqq}{rgb}{0,1,0}
    \definecolor{qqqqff}{rgb}{0,0,1}
    \definecolor{uququq}{rgb}{0.25,0.25,0.25}
    \definecolor{ffqqqq}{rgb}{1,0,0}
    \begin{tikzpicture}[line cap=round,line join=round,>=triangle 45,x=3.0cm,y=3.0cm]
      \draw[->,color=black] (-1.12,0) -- (1.2,0);
      \foreach \x in {-1,1}
      \draw[shift={(\x,0)},color=black] (0pt,2pt) -- (0pt,-2pt);
      \draw[->,color=black] (0,-1.11) -- (0,1.11);
      \foreach \y in {-1,1}
      \draw[shift={(0,\y)},color=black] (2pt,0pt) -- (-2pt,0pt);
      \clip(-1.12,-1.11) rectangle (1.2,1.11);
      \draw [shift={(1,0)},color=ffqqqq,fill=ffqqqq,fill opacity=0.1] (0,0) -- (180:0.26) arc (180:244.77:0.26) -- cycle;
      \draw [shift={(1,0)},color=qqqqff,fill=qqqqff,fill opacity=0.1] (0,0) -- (-180:0.14) arc (-180:-127.09:0.14) -- cycle;
      \draw [shift={(1,0)},color=qqffqq,fill=qqffqq,fill opacity=0.1] (0,0) -- (115.23:0.33) arc (115.23:143.3:0.33) -- cycle;
      \draw(0,0) circle (3cm);
      \draw (1,0)-- (0.27,0.96);
      \draw (0.64,0.77)-- (1,0);
      \draw (0.27,-0.96)-- (1,0);
      \draw (1,0)-- (0.64,-0.77);
      \draw [shift={(1,0)},color=ffqqqq] (180:0.26) arc (180:244.77:0.26);
      \draw [shift={(1,0)},color=ffqqqq] (180:0.23) arc (180:244.77:0.23);
      \draw(0.5,0.37) circle (0.888cm);
      \draw (0.5,0.37)-- (0.77,0.49);
      \draw (0.27,0.54)-- (1,0);
      \draw [shift={(1,0)},color=qqffqq] (115.23:0.33) arc (115.23:143.3:0.33);
      \draw [shift={(1,0)},color=qqffqq] (115.23:0.3) arc (115.23:143.3:0.3);
      \draw [shift={(1,0)},color=qqffqq] (115.23:0.28) arc (115.23:143.3:0.28);
      \begin{scriptsize}
        \draw[color=ffqqqq] (0.84,-0.09) node {$\beta$};
        \fill [color=black] (0.5,0.37) circle (1.5pt);
        \draw[color=black] (0.48,0.32) node {$z$};
        \draw[color=black] (0.18,0.14) node {$C(z)$};
        \draw[color=black] (0.54,0.465) node {$r(z)$};
        \fill [color=black] (1,0) circle (1.5pt);
        \draw[color=black] (1.04,0.05) node {$1$};
        \fill [color=black] (0.26,0.545) circle (1.5pt);
        \draw[color=black] (0.20,0.58) node {$A$};
        \fill [color=black] (1,0) circle (1.5pt);
        \draw[color=qqqqff] (0.92,-0.03) node {$\alpha$};
        \draw[color=qqffqq] (0.83,0.16) node {$\gamma$};
      \end{scriptsize}
    \end{tikzpicture}
  \end{center}

  Dato $\epsilon>0$ fissato, per ipotesi esiste $\delta>0$ tale che $|f(w)-w|<\epsilon|1-w|^3$ per ogni $w \in S'$ con $|w-1|<\delta$. $B(z,r(z)) \subset \mathbb{D} \implies r(z) \le 1-|z|$. Se $|z-1|<\delta/2$, $r(z) \le 1-|z|=|z-1-z|-|z| \le |z-1|+|z|-|z|=|z-1|<\delta/2$, dunque per ogni $w \in C(z)$ abbiamo $|w-1| \le |w-z|+|z-1|=r(z)+|z-1|<\delta$. Per questi $z$ vale che
  \begin{align*}
    |I(z)| & \le \frac{\epsilon}{2\pi} \int_0^{2\pi} \frac{|1-(z+r(z)e^{i\theta})|^3}{|(z+r(z)e^{i\theta})-z|^2}r(z)\diff\theta \\
    & \le \frac{\epsilon}{r(z)}\max_{\theta \in [0,2\pi]} |1-(z+r(z)e^{i\theta})|^3 \\
    & =\frac{\epsilon}{r(z)}\max_{w \in C(z)}|1-w|^3.
  \end{align*}
  Il massimo è raggiunto per l'intersezione più lontana da $1$ tra la circonferenza $C(z)$ e la retta passante per $1$ e $z$ (il punto $A$ in figura), perciò, detto $\gamma$ l'angolo tra $\partial S'$ (per essere precisi, la retta contenente il tratto affine più vicino a $z$) e la retta congiungente $1$ e $z$:
  \begin{align*}
    |I(z)| & \le \frac{\epsilon}{r(z)}(r(z)+|z-1|)^3 \\
    & =\epsilon r(z)^2\left(1+\frac{|z-1|}{r(z)}\right)^3 \\
    & =\epsilon r(z)^2(1+\csc\gamma)^3 \\
    & \le \epsilon r(z)^2(1+\csc(\beta-\alpha))^3 \\
    & \le \epsilon |z-1|^2(1+\csc(\beta-\alpha))^3.
  \end{align*}
  La penultima disuguaglianza segue da $\gamma \ge \beta-\alpha$ e dal fatto che $\csc$ è decrescente sui positivi, mentre l'ultima segue da quanto visto sopra. Otteniamo dunque $f'(z)=1+o((z-1)^2)$ per $z \longrightarrow 1$ non tangenzialmente.

  Adesso ci servirà il seguente lemma.

  \begin{lm} \label{opiccoli}
    Per $z \longrightarrow 1$ non tangenzialmente, $|z-1|$ e $1-|z|$ hanno gli stessi $o$-piccoli.
  \end{lm}

  \begin{proof}
    $1-|z| \le |z-1|$ l'abbiamo già visto. Per concludere ci basta dunque mostrare che, per $z$ appartenente a un settore $S$ di angolo $2\alpha$ fissato, vale una disuguaglianza opposta, a meno di una qualche costante.
    $$|z-1|=r(z)\frac{|z-1|}{r(z)} \le r(z)\csc(\beta-\alpha) \le (1-|z|)\csc(\beta-\alpha),$$
    dove $\beta>\alpha$ è stato scelto come sopra e le disuguaglianze le abbiamo già viste.
  \end{proof}

  Per ipotesi
  $$\frac{1-|f(z)|}{1-|z|}=\frac{1-|z|+o((z-1)^3)}{1-|z|}=1+o((z-1)^2)$$
  per $z \longrightarrow 1$ non tangenzialmente (abbiamo usato il lemma \ref{opiccoli} per poter usare indipendentemente $z-1$ o $1-|z|$ negli $o$-piccoli). \\
  Possiamo quindi concludere che
  $$f^h(z)=|f'(z)|\frac{1-|z|^2}{1-|f(z)|^2}=1+o((z-1)^2)$$
  per $z \longrightarrow 1$ non tangenzialmente.
\end{proof}


\newpage

\section{Lemmi di Schwarz-Pick multi-punto}

\subsection{Teorema di Beardon-Minda e corollari}
Adesso possiamo procedere a dimostrare la serie di disuguaglianze di \cite{BM}, che coinvolgono la distanza di Poincaré $\omega$ e le funzioni olomorfe dal disco in sé che non sono automorfismi.

\begin{prop} \label{24}
  Siano $f \in \text{\normalfont{Hol}}(\mathbb{D},\mathbb{D})\setminus\text{\normalfont{Aut}}(\mathbb{D})$ e $v \in \mathbb{D}$. Allora per ogni $z \in \mathbb{D}$ si ha che $f^*(z,v) \in \mathbb{D}$ e la funzione $z \longmapsto f^*(z,v)$ è olomorfa.
\end{prop}

\begin{proof}
  Per quanto riguarda l'olomorfia, dalla definizione sappiamo che l'unico punto che potrebbe dar problemi è $v$; abbiamo però visto che la funzione ammette limite finito per $z \longrightarrow v$, perciò $v$ è una singolarità rimovibile. Per il lemma di Schwarz-Pick, $|f^*(z,w)| \le 1$; inoltre, vale l'uguaglianza in qualche punto solo se $f$ è un automorfismo. Dunque le ipotesi su $f$ assicurano che vale la disuguaglianza stretta sempre, cioè $f^*(z,v) \in \mathbb{D}$ per ogni $z \in \mathbb{D}$.
\end{proof}

\begin{thm} \label{31}
  (Beardon-Minda, 2004) Sia $f \in \text{\normalfont{Hol}}(\mathbb{D},\mathbb{D})\setminus\text{\normalfont{Aut}}(\mathbb{D})$. Allora per ogni $z, w, v \in \mathbb{D}$ vale
  \begin{equation} \label{3.1}
    \omega\bigl(f^*(z,v),f^*(w,v)\bigr) \le \omega(z,w).
  \end{equation}
\end{thm}

\begin{proof}
  Poiché $f$ non è un automorfismo, per la Proposizione \ref{24} la funzione $z \longmapsto f^*(z,v)$ è olomorfa dal disco unitario in sé; perciò il membro sinistro della disuguaglianza \eqref{3.1} è ben definito e la tesi segue dal lemma di Schwarz-Pick e dall'osservazione \ref{oss1}, punto (ii).
\end{proof}

\begin{cor} \label{32}
  Sia $f \in \text{\normalfont{Hol}}(\mathbb{D},\mathbb{D})\setminus\text{\normalfont{Aut}}(\mathbb{D})$. Allora per ogni $z, w, v \in \mathbb{D}$ vale
  \begin{equation}
    \omega\bigl(0, f^*(z,v)\bigr) \le \omega\bigl(0,f^*(w,v)\bigr)+\omega(z,w).
  \end{equation}
\end{cor}

\begin{proof}
  Si ha
  \begin{align*}
    \omega\bigl(0,f^*(z,v)\bigr) & \le \omega\bigl(0,f^*(w,v)\bigr)+\omega\bigl(f^*(w,v),f^*(z,v)\bigr) \\
    & \le \omega\bigl(0,f^*(w,v)\bigr)+\omega(z,w),
  \end{align*}
  dove la prima è la disuguaglianza triangolare per la distanza $\omega$ e la seconda segue dal Teorema \ref{31}.
\end{proof}

\begin{cor} \label{33}
  Sia $f \in \text{\normalfont{Hol}}(\mathbb{D},\mathbb{D})\setminus\text{\normalfont{Aut}}(\mathbb{D})$. Allora per ogni $z, w, v, u \in \mathbb{D}$ vale
  \begin{equation}
    \omega\bigl(0, f^*(z,v)\bigr) \le \omega\bigl(0, f^*(u,w)\bigr)+\omega(z,w)+\omega(v,u).
  \end{equation}
\end{cor}

\begin{proof}
  Si ha
  \begin{align*}
    \omega\bigl(0,f^*(z,v)\bigr) & \le \omega\bigl(0,f^*(w,v)\bigr)+\omega(z,w) \\
    & =\omega\bigl(0,|f^*(w,v)|\bigr)+\omega(z,w) \\
    & =\omega\bigl(0,|f^*(v,w)|\bigr)+\omega(z,w) \\
    & =\omega\bigl(0,f^*(v,w)\bigr)+\omega(z,w) \\
    & \le \omega\bigl(0,f^*(u,w)\bigr)+\omega(z,w)+\omega(v,u),
  \end{align*}
  dove le due disuguaglianze seguono dal Corollario \ref{32}.
\end{proof}

I due enunciati seguenti non ci serviranno nel seguito, ma vengono riportati per completezza. Prima di enunciare il primo, è necessario dare una definizione.

\begin{defn}
  Una \textit{geodetica} per $\omega$ è una curva $\sigma: \mathbb{R} \longrightarrow \mathbb{D}$ tale che per ogni $t_1,t_2 \in \mathbb{R}$ si ha $\omega\bigl(\sigma(t_1),\sigma(t_2)\bigr)=|t_1-t_2|$.
\end{defn}

\begin{cor} \label{35}
  Sia $f \in \normalfont{\text{Hol}}(\mathbb{D},\mathbb{D})$ e siano $z, w \in \mathbb{D}$. Sia $\sigma$ una geodetica con $\sigma(t_1)=z, \sigma(t_2)=v$ e sia $w=\sigma(t)$ con $t_1<t<t_2$. Allora
  \begin{equation} \label{geod}
    2\omega\bigl(f(z),f(v)\bigr) \le \log\Bigl(\cosh\bigl(2\omega(z,v)\bigr)+|f^h(w)|\sinh\bigl(2\omega(z,v)\bigr)\Bigr).
  \end{equation}
\end{cor}

\begin{proof}
  Osserviamo che se $f \in \text{Aut}(\mathbb{D})$, allora per il lemma di Schwarz-Pick $|f^h(w)|=1$ e il membro destro della disuguaglianza \eqref{geod} è esattamente $2\omega(z,v)$. In questo caso, per il lemma di Schwarz-Pick si ha proprio l'uguaglianza.

  Supponiamo ora $f \not\in \text{Aut}(\mathbb{D})$, allora possiamo applicare il Corollario \ref{33} con $u=v$ per ottenere
  \begin{gather*}
    \omega\bigl(0,f^*(z,v)\bigr) \le \omega\bigl(0,f^h(w)\bigr)+\omega(z,v) \\
    p\bigl(0,f^*(z,v)\bigr) \le \tanh\Bigl(\omega\bigl(0,f^h(w)\bigr)+\omega(z,v)\Bigr) \\
    \frac{p\bigl(f(z),f(v)\bigr)}{p(z,v)} \le \frac{|f^h(w)|+p(z,v)}{1+|f^h(w)|p(z,v)},
  \end{gather*}
  dove abbiamo usato $p=\tanh\omega$, $f^*(z,v)=\frac{[f(z),f(v)]}{[z,v]}$, $\tanh(a+b)=\frac{\tanh{a}+\tanh{b}}{1+\tanh{a}\tanh{b}}$ e $p(0,\zeta)=|\zeta|$. Riscriviamo come
  \begin{gather*}
    \frac{p\bigl(f(z),f(v)\bigr)}{p(z,v)}-p(z,v) \le |f^h(w)|\Bigl(1-p\bigl(f(z),f(v)\bigr)\Bigr) \\
    \frac{p\bigl(f(z),f(v)\bigr)-p^2(z,v)}{p(z,v)\Bigl(1-p\bigl(f(z),f(v)\bigr)\Bigr)} \le |f^h(w)| \\
    \frac{2\Bigl(p\bigl(f(z),f(v)\bigr)-p^2(z,v)\Bigr)}{\bigl(1-p^2(z,v)\bigr)\Bigl(1-p\bigl(f(z),f(v)\bigr)\Bigr)} \le |f^h(w)|\cdot \frac{2p(z,v)}{1-p^2(z,v)}.
  \end{gather*}
  Adesso usiamo le seguenti uguaglianze:
  $$\frac{1+p^2}{1-p^2}=\cosh(2\omega), \quad \frac{2p}{1-p^2}=\sinh(2\omega).$$
  Sommando appunto la quantità $\dfrac{1+p^2(z,v)}{1-p^2(z,v)}$ all'ultima disuguaglianza ottenuta, il membro destro diventa $\cosh\bigl(2\omega(z,v)\bigr)+|f^h(w)|\sinh\bigl(2\omega(z,v)\bigr)$. Ci basta dunque mostrare che il membro sinistro è uguale a $\exp\Bigl(2\omega\bigl(f(z),f(v)\bigr)\Bigr)$, cioè $\dfrac{1+p\bigl(f(z),f(v)\bigr)}{1-p\bigl(f(z),f(v)\bigr)}$. Si ha infatti
  \begin{align*}
    & \frac{2\Bigl(p\bigl(f(z),f(v)\bigr)-p^2(z,v)\Bigr)}{\bigl(1-p^2(z,v)\bigr)\Bigl(1-p\bigl(f(z),f(v)\bigr)\Bigr)}+\frac{1+p^2(z,v)}{1-p^2(z,v)}= \\
    & = \frac{p\bigl(f(z),f(v)\bigr)-p^2(z,v)+1-p^2(z,v)p\bigl(f(z),f(v)\bigr)}{\bigl(1-p^2(z,v)\bigr)\Bigl(1-p\bigl(f(z),f(v)\bigr)\Bigr)} \\
    & = \frac{\bigl(1-p^2(z,v)\bigr)\Bigl(1+p\bigl(f(z),f(v)\bigr)\Bigr)}{\bigl(1-p^2(z,v)\bigr)\Bigl(1-p\bigl(f(z),f(v)\bigr)\Bigr)}=\frac{1+p\bigl(f(z),f(v)\bigr)}{1-p\bigl(f(z),f(v)\bigr)}.
  \end{align*}
\end{proof}

\begin{cor} \label{36}
  Sia $f \in \text{\normalfont{Hol}}(\mathbb{D},\mathbb{D})\setminus\text{\normalfont{Aut}}(\mathbb{D})$ tale che $f(0)=0$. Allora
  \begin{equation}
    \omega\bigl(f^h(0),f^h(z)\bigr) \le 2\omega(0,z).
  \end{equation}
  Inoltre, $2$ è la migliore costante possibile.
\end{cor}

\begin{proof}
  Notiamo che $f(0)=0 \implies f^*(z,0)=f^*(0,z)$, dunque si ha
  \begin{align*}
    \omega\bigl(f^h(0),f^h(z)\bigr) & = \omega\bigl(f^*(0,0),f^*(z,z)\bigr) \\
    & \le \omega\bigl(f^*(0,0),f^*(z,0)\bigr)+\omega\bigl(f^*(0,z),f^*(z,z)\bigr) \\
    & \le 2\omega(0,z)
  \end{align*}
  dove la prima è la disuguaglianza triangolare per $\omega$ e la seconda segue applicando il Teorema \ref{31}.

  Per dire che $2$ è la migliore costante possibile, basta prendere $f(z)=z^2$ e $z \in \mathbb{D}$ con $|z|=1/3$ per ottenere l'uguaglianza.
\end{proof}

Il prossimo risultato è quello che ci permetterà di dimostrare la disuguaglianza di Golusin.

\begin{cor} \label{quasigolusin}
  Sia $f \in \text{\normalfont{Hol}}(\mathbb{D},\mathbb{D})\setminus\text{\normalfont{Aut}}(\mathbb{D})$. Allora per ogni $z, w \in \mathbb{D}$ vale
  \begin{equation} \label{quasigol}
    \omega\bigl(|f^h(z)|, |f^h(w)|\bigr) \le 2\omega(z,w).
  \end{equation}
\end{cor}

\begin{proof}
  Siano $z, w \in \mathbb{D}$; senza perdita di generalità possiamo supporre $|f^h(z)| \ge |f^h(w)|$. Allora
  \begin{align*}
    \omega\bigl(|f^h(z)|, |f^h(w)|\bigr) & =\frac{1}{2}\log\left(\frac{1+\frac{|f^h(z)|-|f^h(w)|}{1-|f^h(w)||f^h(z)|}}{1-\frac{|f^h(z)|-|f^h(w)|}{1-|f^h(w)||f^h(z)|}}\right) \\
    & =\frac{1}{2}\log\left(\frac{1-|f^h(w)||f^h(z)|+|f^h(z)|-|f^h(w)|}{1-|f^h(w)||f^h(z)|+|f^h(w)|-|f^h(z)|}\right) \\
    & =\frac{1}{2}\log\left(\frac{1+|f^h(z)|}{1-|f^h(z)|}\cdot\frac{1-|f^h(w)|}{1+|f^h(w)|}\right) \\
    & =\frac{1}{2}\log\left(\frac{1+|f^h(z)|}{1-|f^h(z)|}\right)-\frac{1}{2}\log\left(\frac{1+|f^h(w)|}{1-|f^h(w)|}\right) \\
    & =\omega\bigl(0,|f^h(z)|\bigr)-\omega\bigl(0,|f^h(w)|\bigr) \\
    & =\omega\bigl(0,f^h(z)\bigr)-\omega\bigl(0,f^h(w)\bigr) \le 2\omega(z,w),
  \end{align*}
  dove l'ultima disuguaglianza segue dal Corollario \ref{33} prendendo $u=w$ e $v=z$.
\end{proof}

La sezione si conclude con due lemmi sulle funzioni olomorfe dal disco in sé, per i quali l'approccio dal punto di vista dell'articolo di Beardon e Minda semplifica le dimostrazioni.

\marginpar{aggiungere delle reference, vedasi [9] e [18] tra le reference di \cite{BM}}

Notazione: dati $E \subset \mathbb{D}$ e $z \in \mathbb{D}$, scriviamo $zE=\{zw \mid w \in E\}$. Inoltre, dati $\gamma \subset \mathbb{D}$ e $r>0$, scriviamo $\Sigma(\gamma,r)=\{w \mid \omega(z,w)<r, z \in \gamma\}$.

\begin{lm}
  (lemma di Rogosinski) Sia $f \in \normalfont{\text{Hol}}(\mathbb{D},\mathbb{D})\setminus\normalfont{\text{Aut}}(\mathbb{D},\mathbb{D})$ tale che $f(0)=0$ e $f'(0) \in \mathbb{R}$. Allora per ogni $z \in \mathbb{D}$ si ha $f(z) \in z\Sigma\bigl((-1,1),\omega(0,z)\bigr)$.
\end{lm}

\begin{proof}
  Sia $g$ definita come nella dimostrazione del lemma di Schwarz, il quale ci dice anche che $g \in \text{Hol}(\mathbb{D},\mathbb{D})$. La tesi è vera per $z=0$, supponiamo dunque $z\not=0$. Per il lemma di Schwarz-Pick si ha
  \begin{gather*}
    \omega\bigl(g(0),g(z)\bigr) \le \omega(0,z) \\
    \omega\bigl(f'(0),f(z)/z\bigr) \le \omega(0,z).
  \end{gather*}
  Per il lemma di Schwarz dev'essere $f'(0) \in (-1,1)$. Abbiamo quindi che $f(z)/z \in \Sigma\bigl((-1,1),\omega(0,z)\bigr) \implies f(z) \in z\Sigma\bigl((-1,1),\omega(0,z)\bigr)$, come voluto.
\end{proof}

\begin{lm}
  (lemma di Dieudonné) Siano $z_0,w_0 \in \mathbb{D}$ con $|w_0| \le |z_0|$ e sia $f \in \text{Hol}(\mathbb{D},\mathbb{D})$ tale che $f(0)=0$ e $f(z_0)=w_0$. Allora
  \begin{equation}
    |f'(z_0)-w_0/z_0| \le \frac{|z_0|^2-|w_0|^2}{|z_0|(1-|z_0|^2)}.
  \end{equation}
\end{lm}

\begin{proof}
  Per il Teorema \ref{31} con $z=v=z_0$ e $w=0$ abbiamo
  \begin{align*}
    \omega\bigl(f^h(z_0),f^*(0,z_0)\bigr) & \le \omega(0,z_0) \\
    \iff p\bigl(f^h(z_0),f^*(0,z_0)\bigr) & \le p(0,z_0)=|z_0|,
  \end{align*}
  dove l'equivalenza fra le due disuguaglianze segue dal fatto che $\text{arctanh}$ è strettamente crescente. Per semplificare, scriviamo $f^h(z_0)=a, f^*(0,z_0)=b, |z_0|=r$. Vogliamo portare la disuguaglianza in forma euclidea. Abbiamo
  \begin{align*}
    p(a,b) & \le r \\
    & \iff \left|\frac{a-b}{1-\bar{b}a}\right| \le r \\
    & \iff (a-b)(\bar{a}-\bar{b}) \le r^2(1-\bar{b}a)(1-b\bar{a}) \\
    & \iff |a|^2-a\bar{b}-\bar{a}b+|b|^2 \le r^2-r^2a\bar{b}-r^2\bar{a}b+r^2|b|^2|a|^2 \\
    & \iff |a|^2(1-r^2|b|^2)-a\bar{b}(1-r^2)-\bar{a}b(1-r^2) \le r^2-|b|^2 \\
    & \iff |a|^2-a\cdot\frac{\bar{b}(1-r^2)}{1-r^2|b|^2}-\bar{a}\cdot\frac{b(1-r^2)}{1-r^2|b|^2} \le \frac{r^2-|b|^2}{1-r^2|b|^2} \\
    & \iff (a-\alpha)(\bar{a}-\bar{\alpha}) \le R^2 \\
    & \iff |a-\alpha| \le R,
  \end{align*}
  dove $\alpha=\dfrac{b(1-r^2)}{1-r^2|b|^2}$ e $R^2=\dfrac{r^2-|b|^2}{1-r^2|b|^2}+|b|^2\left(\dfrac{1-r^2}{1-r^2|b|^2}\right)^2$.
  Ricordando che $r=|z_0|$ e osservando che $b=f^*(0,z_0)=\frac{[f(0),f(z_0)]}{[0,z_0]}=\frac{[0,w_0]}{[0,z_0]}=\frac{w_0}{z_0}$, troviamo $\alpha=\dfrac{w_0(1-|z_0|^2)}{z_0(1-|w_0|^2)}$ e $R=\dfrac{|z_0|^2-|w_0|^2}{|z_0|(1-|w_0|^2)}$.
  Riprendendo infine la definizione di $a$, cioè $a=f^h(z_0)=\frac{f'(z_0)(1-|z_0|^2)}{1-|f(z_0)|^2}=\frac{f'(z_0)(1-|z_0|^2)}{1-|w_0|^2}$, otteniamo che
  $$\left|\frac{f'(z_0)(1-|z_0|^2)}{1-|w_0|^2}-\frac{w_0(1-|z_0|^2)}{z_0(1-|w_0|^2)}\right| \le \frac{|z_0|^2-|w_0|^2}{|z_0|(1-|w_0|^2)},$$
  che è equivalente alla tesi moltiplicando entrambi i membri per $\frac{1-|w_0|^2}{1-|z_0|^2}$.
\end{proof}


\subsection{Applicazioni dei lemmi di Schwarz-Pick multi-punto}
Vediamo ora alcune applicazioni dei risultati visti nella sezione precedente.

\begin{thm} \label{distortion}
  Dato $b \in [0,1)$, scriviamo $F_b(z)=\dfrac{z(z+b)}{1+b z}$. Consideriamo $f \in \normalfont{\text{Hol}}(\mathbb{D},\mathbb{D})$ tale che $f(0)=0$. Se $|f'(0)|<1$, allora per ogni $z \in \mathbb{D}$ si ha
  \begin{equation}
    \left|\frac{f^h(0)-f^h(z)}{1-f^h(0)f^h(z)}\right| \le \frac{2|z|}{1+|z|^2}
  \end{equation}
  e
  \begin{equation}
    F_{|f^h(0)|}^h(-|z|) \le |f^h(z)| \le F_{|f^h(0)|}^h(|z|).
  \end{equation}
\end{thm}

\begin{proof}
  Poiché $|f'(0)|<1$, per il lemma di Schwarz si ha $f \not\in \text{Aut}(\mathbb{D})$. Inoltre $f(0)=0$, perciò possiamo applicare il Corollario \ref{36}; si ha dunque
  $$\omega\bigl(f^h(0),f^h(z)\bigr) \le 2\omega(0,z).$$
  Applicando la tangente iperbolica, sfruttando l'uguaglianza $\tanh(2x)=\frac{2\tanh{x}}{1+\tanh^2{x}}$ e ricordando la definizione di $\omega$ si ha
  $$p\bigl(f^h(0),f^h(z)\bigr) \le \frac{2p(0,z)}{1+p^2(0,z)},$$
  da cui
  $$\left|\frac{f^h(0)-f^h(z)}{1-f^h(0)f^h(z)}\right| \le \frac{2|z|}{1+|z|^2}.$$

  Per dimostrare la seconda disuguaglianza, supponiamo dapprima che si abbia $f^h(0)=b \in [0,1)$. Possiamo ripetere i passaggi svolti nella dimostrazione del lemma di Dieudonné ponendo $a=f^h(z)$ e $r=\frac{2|z|}{1+|z|^2}$. Otteniamo la disuguaglianza $|f^h(z)-\alpha| \le R$, dove $\alpha=\dfrac{b(1-r^2)}{1-r^2b^2}$ e $R^2=\dfrac{r^2-b^2}{1-r^2b^2}+\alpha^2$. Sostituendo troviamo
  $$\alpha=\frac{b(1-|z|^2)^2}{(1+2b|z|+|z|^2)(1-2b|z|+|z|^2)},$$
  $$R=\frac{2|z|(|z|^2+1)(1-b^2)}{(1+2b|z|+|z|^2)(1-2b|z|+|z|^2)}.$$
  Consideriamo adesso $F_b^h(z)=\dfrac{bz^2+2z+b}{|z|^2+2b\,\mathfrak{Re}z+1}\left(\dfrac{|1+b z|}{1+b z}\right)^2$. Si ha
  $$F_b^h(|z|)=\dfrac{b|z|^2+2|z|+b}{|z|^2+2b|z|+1} \text{ e } F_b^h(-|z|)=\dfrac{b|z|^2-2|z|+b}{|z|^2-2b|z|+1}.$$
  Notiamo che $\alpha=\bigl(F_b^h(|z|)+F_b^h(-|z|)\bigr)/2$ e $R=\bigl(F_b^h(|z|)-F_b^h(-|z|)\bigr)/2$, perciò la disuguaglianza $|f^h(z)-\alpha| \le R$ ci dice che $f^h(z)$ appartiene al cerchio con diametro sull'asse reale passante per i punti $F_b^h(|z|)$ e $F_b^h(-|z|)$. Con semplici considerazioni geometriche otteniamo la seguente disuguaglianza:
  $$F_b^h(-|z|) \le \mathfrak{Re}f^h(z) \le |f^h(z)| \le F_b^h(|z|),$$
  la quale, ricordando che $b=f^h(0)$, ci dà
  $$F_{f^h(0)}^h(-|z|) \le |f^h(z)| \le F_{f^h(0)}^h(|z|).$$
  Per passare al caso generale consideriamo la funzione $g(z)=|f^h(0)|f(z)/f'(0)$. Osserviamo che $f(0)=0$ ci dice che $f'(0)=f^h(0)$, dunque $|g(z)|=|f(z)|$ e $|g'(z)|=|f'(z)|$, pertanto $|g^h(z)|=|f^h(z)|$; inoltre si ha anche $g(0)=0$, da cui $g^h(0)=g'(0)=|f^h(0)|$. Perciò applicando l'ultima disuguaglianza trovata alla funzione $g$ otteniamo proprio la seconda disuguaglianza della tesi.
\end{proof}

\begin{cor} \label{distorto}
  Sia $f \in \text{Hol}(\mathbb{D},\mathbb{D})$ tale che $f(0)=0$ e $f'(0) \in [0,1)$. Allora $\mathfrak{Re}f'(z)>0$ per $|z|<f^h(0)/\Bigl(1+\sqrt{1-\bigl(f^h(0)\bigr)^2}\Bigr)$.
\end{cor}

\begin{proof}
  Per $0 \le b<1$ e $z \in \mathbb{D}$ si ha $|z|^2-2b|z|+1>|z|^2-2|z|+1>0$, dunque abbiamo che il segno di $F_b^h(-|z|)$ coincide con quello di $b|z|^2-2|z|+b$. Quest'ultima quantità è minore di $0$ per $|z| \in \bigl((1-\sqrt{1-b^2})/b, (1+\sqrt{1-b^2})/b\bigr)$, zero agli estremi e maggiore di $0$ altrove.
  Prendendo $b=f'(0)=f^h(0)$, nella dimostrazione del Teorema \ref{distortion} abbiamo visto che $\mathfrak{Re}f^h(z) \ge F_{f^h(0)}^h(-|z|)$; per gli $z$ tali che $|z|<\Bigl(1-\sqrt{1-\bigl(f^h(0)\bigr)^2}\Bigr)/f^h(0)=f^h(0)/\Bigl(1+\sqrt{1-\bigl(f^h(0)\bigr)^2}\Bigr)$ si ha quindi $\mathfrak{Re}f^h(z)>0$.
  Ricordando che $f^h(z)=\frac{f'(z)(1-|z|^2)}{1-|f(z)|^2}$ e $f \in \text{Hol}(\mathbb{D},\mathbb{D})$, per tali $z$ si ha anche $\mathfrak{Re}f'(z)>0$.
\end{proof}

Vediamo ora il risultato che, come già anticipato, ci permetterà di dimostrare i teoremi successivi. L'enunciato originale si trova in \cite{GMG}, ma vedremo una formulazione che ci tornerà più utile, in particolare perché coinvolge la funzione $f^h$.

\begin{thm} \label{golusin}
  (disuguaglianza di Golusin, 1945) Sia $f \in \text{\normalfont{Hol}}(\mathbb{D},\mathbb{D})\setminus\text{\normalfont{Aut}}(\mathbb{D})$. Allora per ogni $z \in \mathbb{D}$ vale
  \begin{equation} \label{gol}
    |f^h(z)| \le \frac{|f^h(0)|+\frac{2|z|}{1+|z|^2}}{1+|f^h(0)|\frac{2|z|}{1+|z|^2}}.
  \end{equation}
\end{thm}

\begin{proof}
  Con passaggi analoghi a quelli della dimostrazione del Corollario \ref{quasigolusin} abbiamo che valgono le seguenti uguaglianze:
  \begin{gather*}
    \omega\bigl(|f^h(z)|,|f^h(0)|\bigr)=\frac{1}{2}\log\left(\frac{1+|f^h(z)|}{1-|f^h(z)|}\cdot\frac{1-|f^h(0)|}{1+|f^h(0)|}\right)\\
    \omega(z, 0)=\omega(|z|,0)=\frac{1}{2}\log\left(\frac{1+|z|}{1-|z|}\right).
  \end{gather*}
  Prendendo $w=0$ nella disuguaglianza \eqref{quasigol} otteniamo
  \begin{align}
    \nonumber \frac{1}{2}\log\left(\frac{1+|f^h(z)|}{1-|f^h(z)|}\cdot\frac{1-|f^h(0)|}{1+|f^h(0)|}\right) \le \log\left(\frac{1+|z|}{1-|z|}\right) \\
    \frac{1+|f^h(z)|}{1-|f^h(z)|} \le \frac{1+|f^h(0)|}{1-|f^h(0)|}\left(\frac{1+|z|}{1-|z|}\right)^2. \label{golprimo}
  \end{align}
  Adesso, dalla Proposizione \ref{24} sappiamo che $f^h(z),f^h(0) \in \mathbb{D}$, in particolare $|f^h(z)|,|f^h(0)|<1$, perciò è giustificato il seguente passaggio:
  \begin{align*}
    |f^h(z)| & \le \frac{\frac{1+|f^h(0)|}{1-|f^h(0)|}\left(\frac{1+|z|}{1-|z|}\right)^2-1}{\frac{1+|f^h(0)|}{1-|f^h(0)|}\left(\frac{1+|z|}{1-|z|}\right)^2+1} \\
    & =\frac{(1+|f^h(0)|)(1+2|z|+|z|^2)-(1-|f^h(0)|)(1-2|z|+|z|^2)}{(1+|f^h(0)|)(1+2|z|+|z|^2)+(1-|f^h(0)|)(1-2|z|+|z|^2)} \\
    & =\frac{2|f^h(0)|+2|f^h(0)||z|^2+4|z|}{2+2|z|^2+4|f^h(0)||z|}=\frac{|f^h(0)|+\frac{2|z|}{1+|z|^2}}{1+|f^h(0)|\frac{2|z|}{1+|z|^2}}.
  \end{align*}
\end{proof}


\subsubsection{Il teorema di Pick-Nevanlinna}
Dedichiamo una sottosezione al seguente risultato, dimostrato indipendentemente da Pick nel 1916 \cite{P} e Nevanlinna nel 1919; è un teorema di interpolazione interessante di per sé, inoltre vedremo un paio di esempi in cui le ipotesi vengono riformulate in termini del lemma di Schwarz-Pick, usando in un caso anche il rapporto iperbolico. Seguiamo la dimostrazione vista in \cite[Chapter 1, Theorem 2.2]{JBG}.
\marginpar{le fonti dicono 1916, ma andando a cercare l'articolo originale di Pick è del 1915. Quello di Nevanlinna è introvabile. Mah, mistero!}

\begin{thm}
  (Pick-Nevanlinna) Siano dati $n$ punti distinti $z_1, \dots, z_n \in \mathbb{D}$ e altri $n$ punti distinti (non necessariamente diversi dai primi) $w_1, \dots, w_n \in \mathbb{D}$. Consideriamo la forma quadratica
  $$A_n(t_1,\dots,t_n)=\sum_{i,j=1}^n\frac{1-w_i\bar{w}_j}{1-z_i\bar{z}_j}t_i\bar{t}_j.$$
  Allora esiste una funzione $f \in \normalfont{\text{Hol}}(\mathbb{D},\mathbb{D})$ tale che $f(z_i)=w_i$ per $j=1, \dots, n$ se e solo se $A_n$ è semidefinita positiva. In tal caso, si può trovare $f$ che sia un prodotto di Blaschke di grado al più $n$.
\end{thm}

\begin{proof}
  Procediamo per induzione su $n$. Il caso $n=1$ è banale per transitività di $\text{Aut}(\mathbb{D})$. Supponiamo adesso $n>1$. Poniamo
  $$z_i'=\frac{z_i-z_n}{1-\bar{z}_nz_i} \,\, \text{e} \,\, w_i'=\frac{w_i-w_n}{1-\bar{w}_nw_i} \,\, \text{per} \,\, 1 \le i \le n.$$
  Allora esiste $f$ olomorfa dal disco in sé che risolve l'interpolazione se e solo se la funzione
  $$g(z)=\frac{\Bigg(f\left(\dfrac{z+z_n}{1+\bar{z}_nz}\right)-w_n\Bigg)}{1-\bar{w}_n\Bigg(f\left(\dfrac{z+z_n}{1+\bar{z}_nz}\right)\Bigg)}$$
  appartiene a $\text{Hol}(\mathbb{D},\mathbb{D})$ e soddisfa $g(z_i')=w_i'$ per $1 \le i \le n$. Infatti, si tratta solo di comporre con i giusti automorfismi. La forma quadratica $A_n'$ definita con i punti $z_i',w_i'$ è legata alla forma $A_n$. Per mostrarlo, poniamo
  $$\frac{1-z_i'\bar{z}_j'}{1-z_i\bar{z}_j}=\frac{1-|z_n|^2}{(1-\bar{z}_nz_i)(1-z_n\bar{z}_j)}=\alpha_i\bar{\alpha}_j$$
  e
  $$\frac{1-w_i'\bar{w}_j'}{1-w_i\bar{w}_j}=\frac{1-|w_n|^2}{(1-\bar{w}_nw_i)(1-w_n\bar{w}_j)}=\beta_i\bar{\beta}_j,$$
  dove $\alpha_i=\frac{\sqrt{1-|z_n|^2}}{1-\bar{z}_nz_i}$ per $1 \le i \le n$ e analogamente per i $\beta_i$. Allora si ha
  \begin{align*}
    A_n'(t_1,\dots,t_n) & =\sum_{i,j=1}^n \frac{1-w_i'\bar{w}_j'}{1-z_i'\bar{z}_j'}t_i\bar{t}_j \\
    & =\sum_{i,j=1}^n \frac{1-w_i\bar{w}_j}{1-z_i\bar{z}_j}\left(\frac{\beta_i}{\alpha_i}t_i\right)\left(\frac{\bar{\beta}_j}{\bar{\alpha}_i}\bar{t}_j\right)=A_n\left(\frac{\beta_1}{\alpha_1}t_1,\dots,\frac{\beta_n}{\alpha_n}t_n\right).
  \end{align*}
  Poiché $z_i,w_i \in \mathbb{D}$ si ha $\alpha_i,\beta_i\not=0$; perciò $A_n$ è semidefinita positiva se e solo se lo è $A_n'$. Dato che $z_n'=w_n'=0$, a meno di cambiare $f$ con $g$ possiamo supporre senza perdita di generalità $z_n=w_n=0$. La condizione dell'enunciato diventa dunque $f(0)=0$ e $f(z_i)=w_i$ per $1 \le i \le n-1$.
  Tale funzione esiste se e solo se la funzione $h(z)=f(z)/z$, con $h(0)=f'(0)$, appartiene a $\text{Hol}(\mathbb{D},\mathbb{D})$ e soddisfa $h(z_i)=w_i/z_i$ per $1 \le i \le n-1$. Per il punto (ii) della proposizione \ref{blaschke-prop} con $w=0$, abbiamo anche che $f \in \mathcal{B}_d$ se e solo se $h \in \mathcal{B}_{d-1}$.
  Vogliamo dire che $A_n$ è semidefinita positiva se e solo se la forma quadratica $A_n''$, costruita usando i punti $z_i, w_i/z_i$, è semidefinita positiva. Dato che $z_n=w_n=0$, completando il quadrato abbiamo
  \begin{align*}
    A_n(t_1,\dots,t_n)&=|t_n|^2+2\cdot\mathfrak{Re}\sum_{i=1}^{n-1}\bar{t}_it_n+\sum_{i,j=1}^{n-1}\frac{1-w_i\bar{w}_j}{1-z_i\bar{z}_j}t_i\bar{t}_j \\
    &=\left|\sum_{i=1}^n t_i\right|^2+\sum_{i,j=1}^{n-1}\left(\frac{1-w_i\bar{w}_j}{1-z_i\bar{z}_j}-1\right)t_i\bar{t}_j \\
    &=\left|\sum_{i=1}^n t_i\right|^2+\sum_{i,j=1}^{n-1}\frac{1-(w_i/z_i)\overline{(w_j/z_j)}}{1-z_i\bar{z}_j}z_i\bar{z}_jt_i\bar{t}_j,
  \end{align*}
  quindi
  $$A_n(t_1,\dots,t_n)=\left|\sum_{i=1}^n t_i\right|^2+A_n''(z_1t_1,\dots,z_{n-1}t_{n-1}).$$
  Allora se $A_n''$ è semidefinita positiva lo è anche $A_n$, mentre per l'implicazione opposta basta prendere $\displaystyle t_n=-\sum_{i=1}^n t_i$ (ricordiamo che per ipotesi $z_i\not=z_n=0$ per $1 \le i \le n-1$).
\end{proof}

Vediamo che il caso $n=2$ è equivalente alla disuguaglianza del lemma di Schwarz-Pick. La forma quadratica $A_2$ è semidefinita positiva se e solo se i determinanti dei minori principali della matrice associata, che è hermitiana, sono non negativi. Per ipotesi $\frac{1-|w_1|^2}{1-|z_1|^2} \ge 0$, mentre per il determinante della matrice abbiamo
$$\frac{1-|w_1|^2}{1-|z_1|^2}\cdot\frac{1-|w_2|^2}{1-|z_2|^2}-\left|\frac{1-w_1\bar{w}_2}{1-z_1\bar{z}_2}\right|^2 \ge 0,$$
ovvero
$$\frac{|1-z_1\bar{z}_2|^2}{(1-|z_1|^2)(1-|z_2|^2)} \ge \frac{|1-w_1\bar{w}_2|^2}{(1-|w_1|^2)(1-|w_2|^2)};$$
riarrangiando i denominatori si ottiene
$$\frac{|1-z_1\bar{z}_2|^2}{|1-\bar{z}_2z_1|^2-|z_1-z_2|^2} \ge \frac{|1-w_1\bar{w}_2|^2}{|1-\bar{w}_2w_1|^2-|w_1-w_2|^2},$$
che è equivalente a
\begin{align*}
  \frac{1}{1-\left|\frac{z_1-z_2}{1-\bar{z}_2z_1}\right|^2} & \ge \frac{1}{1-\left|\frac{w_1-w_2}{1-\bar{w}_2w_1}\right|^2} \\
  & \Leftrightarrow \frac{1}{1-p^2(w_1,w_2)} \le \frac{1}{1-p^2(z_1,z_2)} \Leftrightarrow \omega(w_1,w_2) \le \omega(z_1,z_2).
\end{align*}
Per invarianza di $p$, e dunque anche di $\omega$, sotto l'azione di $\text{Aut}(\mathbb{D})$, si può porre $w_1=z_1=0$ e la funzione di interpolazione si trova immediatamente.

Andiamo adesso a ridimostrare il caso $n=3$ con una formulazione differente; otteniamo così una sorta di inverso del Teorema \ref{31}.

\begin{thm}
  Siano $z_1, z_2, z_3$ e $w_1, w_2, w_3$ due triple di punti distinti in $\mathbb{D}$. Allora esiste $f \in \normalfont{\text{Hol}}(\mathbb{D},\mathbb{D}) \setminus \normalfont{\text{Aut}}(\mathbb{D})$ tale che $f(z_i)=w_i$ per $i=1,2,3$ se e solo se valgono le seguenti condizioni:
  \begin{nlist}
    \item $\omega(w_i,w_j)<\omega(z_i,z_j)$ per $i,j=1,2,3$ e $i\not=j$;
    \item $\omega\left(\dfrac{[w_2,w_1]}{[z_2,z_1]},\dfrac{[w_3,w_1]}{[z_3,z_1]}\right) \le \omega(z_2,z_3)$.
  \end{nlist}
\end{thm}

\begin{proof}
  Supponiamo che esista siffatta $f$. Allora la condizione (i) segue dal lemma di Schwarz-Pick. La condizione (ii) invece si riscrive come $\omega\bigl(f^*(z_2,z_1),f^*(z_3,z_1)\bigr) \le \omega(z_2,z_3)$, che è l'enunciato del Teorema \ref{31}.

  Adesso dimostriamo l'altra freccia. Vediamola prima nel caso $z_1=w_1=0$. Allora per la condizione (i) abbiamo $\omega(0,w_i) < \omega(0,z_i)$, quindi $|w_i/z_i|<1$ per $i=2,3$. La condizione (ii) si riscrive invece come $\omega(w_2/z_2,w_3/z_3) \le \omega(z_2,z_3)$, cioè $p(w_2/z_2,w_3/z_3) \le p(z_2,z_3)$.
  Dunque, per il caso $n=2$ del teorema di Pick-Nevanlinna, esiste $g \in \text{Hol}(\mathbb{D},\mathbb{D})$ tale che $g(z_2)=w_2/z_2$ e $g(z_3)=w_3/z_3$. Allora basta prendere $f(z)=zg(z)$.

   Mostriamo che ci si può ridurre a questo caso. Consideriamo $h, g \in \text{Aut}(\mathbb{D})$ date da
   $$g(z)=\frac{z-z_1}{1-\bar{z}_1z} \,\, \text{e} \,\, h(z)=\frac{z-w_1}{1-\bar{w}_1z}.$$
   Allora esiste $f$ come quella richiesta dal Teorema se e solo se esiste $F \in \text{Hol}(\mathbb{D},\mathbb{D})$, con $F=h \circ f \circ g^{-1}$, tale che $F(0)=0$, $F\bigl(g(z_2)\bigr)=h(w_2)$ e $F\bigl(g(z_3)\bigr)=h(w_3)$.
   Questo corrisponde proprio al caso precedente, quindi tale $F$ esiste se e solo se
   $$\omega\bigl(h(w_i),h(w_j)\bigr) \le \omega\bigl(g(z_i),g(z_j)\bigr)$$
   per $i,j=1,2,3$ con $i\not=j$ e
   $$\omega\left(\frac{h(w_2)}{g(z_2)},\frac{h(w_3)}{g(z_3)}\right) \le \omega\bigl(g(z_2),g(z_3)\bigr).$$
   Poiché $\omega$ è invariante per azione di $\text{Aut}(\mathbb{D})$, la prima disuguaglianza è equivalente alla condizione (i). Sempre per questo motivo, sostituendo $h(z)=[z,w_1]$ e $g(z)=[z,z_1]$ otteniamo che la seconda è equivalente alla condizione (ii).
\end{proof}

L'argomento viene trattato più in dettaglio in \cite{BRW}, dove il caso $n$ generico viene studiato usando i rapporti iperbolici iterati. Nell'articolo i prodotti di Blaschke, e il loro legame con i rapporti iperbolici, vengono comparati ai polinomi e il loro legame con i rapporti incrementali nel caso euclideo. Anche l'algoritmo di interpolazione viene paragonato, formalmente, a quello di Newton per i polinomi.


\newpage

\section{Dalla disuguaglianza di Golusin al teorema di Burns-Krantz}

\subsection{Rigidità al bordo}
Ponendo $w=0$ in ref{quasigolusin} otteniamo la disuguaglianza di Golusin, dalla quale possiamo dimostrare una versione al bordo del lemma di Schwarz-Pick, seguendo la traccia data nel remark 5.6 di \cite{BKR}.

\begin{thm} \label{boundary_schwarz_pick}
  (lemma di Scharz-Pick al bordo) Sia $f:\mathbb{D} \longrightarrow \mathbb{D}$ una funzione olomorfa dal disco in sé tale che
  \begin{equation}
    f^h(z_n)=1+o(||z_n|-1|^2)
  \end{equation}
  per qualche successione $\{z_n\}_{n \in \mathbb{N}} \subset \mathbb{D}$ con $|z_n| \longrightarrow 1$. Allora $f \in \text{Aut}(\mathbb{D})$.
\end{thm}

\marginpar{segue da Golusin, cioè tutta una serie di risultati che texerò nei giorni a venire}

\begin{proof}
  Supponiamo per assurdo che $f$ non sia un automorfismo. Allora possiamo applicare ref{quasigolusin}, che per $w=0$ ci dà
\end{proof}


\subsection{Teorema di Burns-Krantz}
Per poter dimostrare il risultato finale sfruttando la versione al bordo del lemma di Schwarz-Pick, serve poter tradurre informazioni sull'andamento di $f$ vicino a $1$ in informazioni sull'andamento di $f^h$. La seguente proposizione ci permette proprio di fare queato passaggio.

\begin{prop} \label{o^3->o^2}
  Sia $f:\mathbb{D} \longrightarrow \mathbb{D}$ una funzione tale che
  \begin{equation} \label{o^3}
    f(z)=1+(z-1)+o((z-1)^3)
  \end{equation}
  per $z \longrightarrow 1$ non tangenzialmente. Allora
  \begin{equation} \label{o^2}
    f^h(z)=1+o((z-1)^2)
  \end{equation}
  per $z \longrightarrow 1$ non tangenzialmente.
\end{prop}

\begin{proof}

  Sia $S$ un settore di vertice $1$ e angolo d'apertura $2\alpha$, e $S'$ uno un po' più grande di vertice $1$ e angolo $2\beta$, $\beta>\alpha$. Per $z \in S$, sia $C(z)$ il cerchio di centro $z$ e raggio $r(z)=\text{dist}(z, \partial S')$ (la distanza di $z$ dal bordo di $S'$). Allora per la formula integrale di Cauchy
  \begin{align*}
    f'(z) & =\frac{1}{2\pi i} \int_{C(z)} \frac{f(w)}{(w-z)^2}\diff w \\
    & =\frac{1}{2\pi i} \int_{C(z)} \frac{w-1+(f(w)-w)}{(w-z)^2}\diff w \\
    & =\frac{1}{2\pi i} \int_{C(z)} \frac{1}{w-z}\diff w+\frac{1}{2\pi i} \int_{C(z)} \frac{z-1+f(w)-w}{(w-z)^2}\diff w \\
    & =1+\frac{1}{2\pi i} \int_{C(z)} \frac{f(w)-w}{(w-z)^2}\diff w=:1+I(z).
  \end{align*}

  \begin{center}
    \definecolor{qqffqq}{rgb}{0,1,0}
    \definecolor{qqqqff}{rgb}{0,0,1}
    \definecolor{uququq}{rgb}{0.25,0.25,0.25}
    \definecolor{ffqqqq}{rgb}{1,0,0}
    \begin{tikzpicture}[line cap=round,line join=round,>=triangle 45,x=3.0cm,y=3.0cm]
      \draw[->,color=black] (-1.12,0) -- (1.2,0);
      \foreach \x in {-1,1}
      \draw[shift={(\x,0)},color=black] (0pt,2pt) -- (0pt,-2pt);
      \draw[->,color=black] (0,-1.11) -- (0,1.11);
      \foreach \y in {-1,1}
      \draw[shift={(0,\y)},color=black] (2pt,0pt) -- (-2pt,0pt);
      \clip(-1.12,-1.11) rectangle (1.2,1.11);
      \draw [shift={(1,0)},color=ffqqqq,fill=ffqqqq,fill opacity=0.1] (0,0) -- (180:0.26) arc (180:244.77:0.26) -- cycle;
      \draw [shift={(1,0)},color=qqqqff,fill=qqqqff,fill opacity=0.1] (0,0) -- (-180:0.14) arc (-180:-127.09:0.14) -- cycle;
      \draw [shift={(1,0)},color=qqffqq,fill=qqffqq,fill opacity=0.1] (0,0) -- (115.23:0.33) arc (115.23:143.3:0.33) -- cycle;
      \draw(0,0) circle (3cm);
      \draw (1,0)-- (0.27,0.96);
      \draw (0.64,0.77)-- (1,0);
      \draw (0.27,-0.96)-- (1,0);
      \draw (1,0)-- (0.64,-0.77);
      \draw [shift={(1,0)},color=ffqqqq] (180:0.26) arc (180:244.77:0.26);
      \draw [shift={(1,0)},color=ffqqqq] (180:0.23) arc (180:244.77:0.23);
      \draw(0.5,0.37) circle (0.888cm);
      \draw (0.5,0.37)-- (0.77,0.49);
      \draw (0.27,0.54)-- (1,0);
      \draw [shift={(1,0)},color=qqffqq] (115.23:0.33) arc (115.23:143.3:0.33);
      \draw [shift={(1,0)},color=qqffqq] (115.23:0.3) arc (115.23:143.3:0.3);
      \draw [shift={(1,0)},color=qqffqq] (115.23:0.28) arc (115.23:143.3:0.28);
      \begin{scriptsize}
        \draw[color=ffqqqq] (0.84,-0.09) node {$\beta$};
        \fill [color=black] (0.5,0.37) circle (1.5pt);
        \draw[color=black] (0.48,0.32) node {$z$};
        \draw[color=black] (0.18,0.14) node {$C(z)$};
        \draw[color=black] (0.54,0.465) node {$r(z)$};
        \fill [color=black] (1,0) circle (1.5pt);
        \draw[color=black] (1.04,0.05) node {$1$};
        \fill [color=black] (0.26,0.545) circle (1.5pt);
        \draw[color=black] (0.20,0.58) node {$A$};
        \fill [color=black] (1,0) circle (1.5pt);
        \draw[color=qqqqff] (0.92,-0.03) node {$\alpha$};
        \draw[color=qqffqq] (0.83,0.16) node {$\gamma$};
      \end{scriptsize}
    \end{tikzpicture}
  \end{center}

  Dato $\epsilon>0$ fissato, per ipotesi esiste $\delta>0$ tale che $|f(w)-w|<\epsilon|1-w|^3$ per ogni $w \in S'$ con $|w-1|<\delta$. Se $|z-1|<\delta/2$, $r(z)<|z-1|<\delta/2$, dunque per ogni $w \in C(z)$ abbiamo effettivamente $|w-1| \le |w-z|+|z-1|=r(z)+|z-1|<\delta$. Per questi $z$ vale che
  \begin{align*}
    |I(z)| & \le \frac{\epsilon}{2\pi} \int_0^{2\pi} \frac{|1-(z+r(z)e^{i\theta})|^3}{|(z+r(z)e^{i\theta})-z|^2}r(z)\diff\theta \\
    & \le \frac{\epsilon}{r(z)}\max_{\theta \in [0,2\pi]} |1-(z+r(z)e^{i\theta})|^3 \\
    & =\frac{\epsilon}{r(z)}\max_{w \in C(z)}|1-w|^3.
  \end{align*}
  Il massimo è raggiunto per l'intersezione più lontana da $1$ tra la retta passante per $z$ e $1$ e la circonferenza $C(z)$ (il punto $A$ in figura), perciò
  \begin{align*}
    |I(z)| & \le \frac{\epsilon}{r(z)}(r(z)+|z-1|)^3 \\
    & =\epsilon r(z)^2\left(1+\frac{|z-1|}{r(z)}\right)^3 \\
    & =\epsilon r(z)^2(1+\csc\gamma)^3 \\
    & \le \epsilon r(z)^2(1+\csc(\beta-\alpha))^3 \\
    & \le \epsilon |z-1|^2(1+\csc(\beta-\alpha))^3,
  \end{align*}
  da cui otteniamo $f'(z)=1+o((z-1)^2)$ per $z \longrightarrow 1$ non tangenzialmente. Inoltre, per ipotesi
  \marginpar{$|z-1|$ e $1-|z|$ non tangenzialmente hanno gli stessi $o$-piccoli (chiedere conferma ad Abate): devo scrivere la dimostrazione?}
  $$\frac{1-|f(z)|}{1-|z|}=\frac{1-|z|+o((z-1)^3)}{1-|z|}=1+o((z-1)^2)$$
  per $z \longrightarrow 1$ non tangenzialmente (perché in tal caso $|z-1|$ e $1-|z|$ hanno gli stessi $o$-piccoli). \\
  %Accenno della dim: 1-|z| \le |z-1| segue dalla disuguaglianza triangolare, |z-1| \le (1-|z|)*costante si fa scrivendo |z-1|=r(z)*(|z-1|/r(z)), ripetere la dim di sopra per far apparire la costante csc(beta-alpha) e osservando che r(z) \le 1-|z| perché C(z) sta dentro \mathbb{D}
  Possiamo quindi concludere che
  $$f^h(z)=|f'(z)|\frac{1-|z|^2}{1-|f(z)|^2}=1+o((z-1)^2)$$
  per $z \longrightarrow 1$ non tangenzialmente.
  \marginpar{Gli ultimi passaggi con $o$-piccoli: devo spiegarli meglio? Forse non sono così ovvi}
\end{proof}

Siamo ora pronti a dimostrare il teorema 2.1 di \cite{BK}.

\begin{thm} \label{burns_krantz}
  (Burns-Krantz, 1994) Sia $f:\mathbb{D} \longrightarrow \mathbb{D}$ una funzione olomorfa dal disco in sé tale che
  \begin{equation} \label{O^4}
    f(z)=1+(z-1)+\mathcal{O}((z-1)^4)
  \end{equation}
  per $z \longrightarrow 1$. Allora $f$ è l'identità sul disco.
\end{thm}

\begin{proof}
  \marginpar{È comprensibile?}

  Chiaramente, se vale \eqref{O^4} per $z \longrightarrow 1$ vale anche \eqref{o^3}, in particolare per $z \longrightarrow 1$ non tangenzialmente.
  Dalla proposizione \ref{o^3->o^2} segue che anche \eqref{o^2} vale per $z \longrightarrow 1$ non tangenzialmente, quindi esiste una successione $z_n$ che soddisfa le ipotesi del teorema \ref{boundary_schwarz_pick} (usiamo di nuovo che, non tangenzialmente, $|z-1|$ e $1-|z|$ hanno gli stessi $o$-piccoli), da cui la tesi.
\end{proof}

Il termine $\mathcal{O}((z-1)^4)$ non è migliorabile, come mostra il seguente controesempio.

\begin{ex}
  $f(z)=\dfrac{1+3z^2}{3+z^2}$. Osserviamo che $f$ è una funzione olomorfa su $\mathbb{C} \setminus \{\pm i\sqrt{3}\}$, quindi in particolare è ben definita su $\mathbb{D}$. Verifichiamo che l'immagine è contenuta in $\mathbb{D}$:
  \begin{align*}
    |f(z)|^2<1 \\
    \dfrac{(1+3z^2)(1+3\bar{z}^2)}{(3+z^2)(3+\bar{z}^2)} < 1 \\
    (1+3z^2)(1+3\bar{z}^2) < (3+z^2)(3+\bar{z}^2) \\
    1+3z^2+3\bar{z}^2+9|z|^4 < 9+3z^2+3\bar{z}^2+|z|^4 \\
    1-|z|^4 < 9(1-|z|^4)
  \end{align*}
  e l'ultima disuguaglianza è verificata perché $z \in \mathbb{D} \implies |z|<1 \implies 1-|z|^4>0$. \\
  Ovviamente $f$ non può essere iniettiva su $\mathbb{D}$ perché $f(z)=f(-z)$, dunque non è un automorfismo. Adesso mostriamo che $f(z)-1-(z-1)$ è $\mathcal{O}((z-1)^3)$ ma non $\mathcal{O}((z-1)^4)$ per $z \longrightarrow 1$:
  \begin{align*}
    f(z)-z & =\frac{1+3z^2}{3+z^2}-z \\
    & =\frac{1+3z^2-3z-z^3}{3+z^2} \\
    & =\frac{(1-z)^3}{3+z^2}=:g(z).
  \end{align*}
  Poiché $\displaystyle \lim_{z \longrightarrow 1} g(z)/(z-1)^3=-1/4$, $g(z)$ è $\mathcal{O}((z-1)^3)$ ma non $\mathcal{O}((z-1)^4)$ per $z \longrightarrow 1$.
\end{ex}


\newpage

\begin{thebibliography}{widest entry}
%  \bibitem[BK]{BK} D. M. Burns, S. G. Krantz: Rigidity of holomorphic mappings and a new Schwarz lemma at the boundary. \textit{Journal of the American Mathematical Society}, \textbf{7} (1994), no. 3, 661--676
%  \bibitem[BKR]{BKR} F. Bracci, D. Kraus, O. Roth: A new Schwarz-Pick Lemma at the boundary and rigidity of holomorphic maps. Preprint, ArXiv:2003.02019v1 (2020)
%  \bibitem[BM]{BM} A. F. Beardon, D. Minda: A multi-point Schwarz-Pick lemma. \textit{Journal d'Analyse Mathématique}, \textbf{92} (2004), 81--104
%  \bibitem[BRW]{BRW} L. Baribeau, P. Rivard, E. Wegert: On Hyperbolic Divided Differences and the Nevanlinna-Pick Problem. \textit{ Computational Methods and Function Theory}, \textbf{9} (2009), no. 2, 391--405
%  \bibitem[D]{D} J. Dieudonné: Recherches sur quelques problèmes relatifs aux polynômes et aux fonctions bornées d'une variable complexe. \textit{Annales Scientifiques de l'École Normale Supérieure}, \textbf{48} (1931), 247--358
%  \bibitem[GMG]{GMG} G. M. Golusin: Some estimations of derivatives of bounded functions. \textit{Recueil Mathématique [Matematicheskiĭ Sbornik]}, \textbf{16(58)} (1945), no. 3, 295--306
%  \bibitem[JBG]{JBG} J. B. Garnett: \textbf{Bounded Analytic Functions (Revised First Edition)}. Springer, New York, 2007
%  \bibitem[N]{N} R. Nevanlinna: Über beschränkte Funktionen, die in gegebenen Punkten vorgeschriebene Werte annehmen. \textit{Annales Academiae Scientiarum Fennicae, Series A}, \textbf{13} (1919) no. 1
%  \bibitem[NN]{NN} R. Narasimhan, Y. Nievergelt: \textbf{Complex analysis in one variable (2nd edition)}. Springer, New York, 2001
%  \bibitem[P]{P} G. Pick: Über die Beschränkungen analytischer Funktionen, welche durch vorgegebene Funktionswerte bewirkt werden. \textit{ Mathematische Annalen}, \textbf{77} (1915), no. 1, 7--23
\end{thebibliography}
Da scrivere.

\addcontentsline{toc}{section}{Riferimenti bibliografici}

\section*{Ringraziamenti}
\addcontentsline{toc}{section}{Ringraziamenti}
Inizio ringraziando il mio relatore, il professor Marco Abate, per un'infinità di cose: l'argomento proposto, l'attenzione ai dettagli, tutti i consigli e gli insegnamenti su come scrivere matematica con un'esposizione chiara ed efficace, oltre che corretta, e su come presentarla ponendo l'attenzione sui passaggi importanti e le idee dietro le dimostrazioni.

Ciò che ho imparato durante la stesura di questa tesi mi sarà di grande aiuto negli anni a venire. \\

Ringrazio anche i compagni con cui ho passato questi tre anni, in particolare Maria Chiara, Lucrezia, Giorgio, Federico e infine Alessio, con il quale ho trascorso più tempo di tutti tra dispense, computer e videogiochi.

Abbiamo studiato insieme, ma soprattutto ci siamo divertiti. È anche grazie a voi se gli ultimi anni resteranno sempre tra i migliori della mia vita. \\

Per finire, ringrazio tutti i miei parenti: loro, a differenza mia, non hanno mai dubitato delle mie capacità.

Un grazie speciale ai miei genitori: mi hanno sempre sostenuto, ma hanno anche dovuto sopportarmi per mesi causa pandemia; mi conosco, e ancora non mi capacito di come siano riusciti nell'impresa. Vi ringrazio con tutto il cuore!


\end{document}
