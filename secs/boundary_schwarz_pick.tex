Dalla disuguaglianza di Golusin possiamo dimostrare una risultato di rigidità al Bordo, seguendo la traccia data nel Remark 5.6 di \cite{BKR}.

\begin{thm} \label{boundary_schwarz_pick}
  (Bracci-Kraus-Roth, 2020) Sia $f \in \text{\normalfont{Hol}}(\mathbb{D},\mathbb{D})$ tale che
  \begin{equation} \label{n_o^2}
    |f^h(z_n)|=1+o\bigl((|z_n|-1)^2\bigr)
  \end{equation}
  per qualche successione $\{z_n\}_{n \in \mathbb{N}} \subset \mathbb{D}$ con $|z_n| \longrightarrow 1$. Allora $f \in \text{\normalfont{Aut}}(\mathbb{D})$.
\end{thm}

\begin{proof}
  Supponiamo per assurdo che $f \not\in \text{Aut}(\mathbb{D})$. Possiamo applicare la disuguaglianza di Golusin \ref{golusin} nella forma \eqref{golprimo}, che riscriviamo come
  \begin{align*}
    \frac{\bigl(1-|f^h(0)|\bigr)\bigl(1+|f^h(z_n)|\bigr)}{\bigl(1-|f^h(z_n)|\bigr)\bigl(1+|f^h(0)|\bigr)} \le \frac{(1+|z_n|)^2}{(1-|z_n|)^2} \\
    \frac{1+|f^h(0)|}{\bigl(1-|f^h(0)|\bigr)\bigl(1+|f^h(z_n)|\bigr)}\bigl(1-|f^h(z_n)|\bigr) \ge \frac{(1-|z_n|)^2}{(1+|z_n|)^2}.
  \end{align*}
  Per ipotesi vale \eqref{n_o^2}, dunque
  \begin{align*}
    \frac{1+|f^h(0)|}{\bigl(1-|f^h(0)|\bigr)\bigl(1+|f^h(z_n)|\bigr)}o\bigl((|z_n|-1)^2\bigr) \ge \frac{(1-|z_n|)^2}{(1+|z_n|)^2} \\
    \frac{\bigl(1+|f^h(0)|\bigr)(1+|z_n|)^2}{\bigl(1-|f^h(0)|\bigr)\bigl(1+|f^h(z_n)|\bigr)}o(1) \ge 1.
  \end{align*}
  Se $f \not\in \text{Aut}(\mathbb{D})$, per il lemma di Schwarz-Pick si ha necessariamente $|f^h(0)|<1$, dunque $\displaystyle \lim_{n \longrightarrow +\infty} \frac{\bigl(1+|f^h(0)|\bigr)(1+|z_n|)^2}{\bigl(1-|f^h(0)|\bigr)\bigl(1+|f^h(z_n)|\bigr)}=\frac{2\bigl(1+|f^h(0)|\bigr)}{1-|f^h(0)|} < +\infty$ e otteniamo una contraddizione.
\end{proof}

Siamo ora pronti a dimostrare il Theorem 2.1 di \cite{BK}.
