Dalla parte di unicità del lemma di Schwarz-Pick possiamo dedurre il seguente risultato di rigidità: sia $f \in \text{Hol}(\mathbb{D},\mathbb{D})$ tale che
$$f(z)=z+o(z-z_0) \text{ per } z \longrightarrow z_0 \in \mathbb{D};$$
allora $f$ è l'identità. Lo si può dimostrare osservando che $f'(z_0)=1$ e usando quest'altro risultato più forte, che è una diretta conseguenza del lemma di Schwarz-Pick: se $f \in \text{Hol}(\mathbb{D},\mathbb{D})$ è tale che
$$|f^h(z)|=1+o(1) \text{ per } z \longrightarrow z_0 \in \mathbb{D},$$
allora $f \in \text{Aut}(\mathbb{D})$. A questo punto, per dimostrare il primo enunciato basta supporre $z_0=0$, a meno di comporre con opportuni automorfismi.
\marginpar{devo approfondire con i dettagli o va bene questa spiegazione?}

Questi sono risultati di rigidità piuttosto forti. Ad esempio, il primo può essere riassunto così: se una funzione olomorfa dal disco in sé dista dall'identità per termini di grado minore al primo (e ovviamente non potremmo chiedere di meglio), allora è forzata a essere l'identità. Il secondo è simile: se la derivata iperbolica di una funzione si comporta come quella di un automorfismo (cioè la funzione e la sua derivata si comportano come quelle di un automorfismo), allora la funzione è necessariamente un automorfismo.

Questi sono risultati che valgono per punti interni al disco; quello che andremo a fare in questa sezione e nella successiva è dimostrare i loro analoghi per punti sul bordo, andando a perdere due gradi nel resto delle ipotesi. Useremo la disuguaglianza di Golusin, seguendo la traccia data nel Remark 5.6 di \cite{BKR}.

\begin{thm} \label{boundary_schwarz_pick}
  (Bracci-Kraus-Roth, 2020) Sia $f \in \text{\normalfont{Hol}}(\mathbb{D},\mathbb{D})$ tale che
  \begin{equation} \label{n_o^2}
    |f^h(z_n)|=1+o\bigl((|z_n|-1)^2\bigr)
  \end{equation}
  per qualche successione $\{z_n\}_{n \in \mathbb{N}} \subset \mathbb{D}$ con $|z_n| \longrightarrow 1$. Allora $f \in \text{\normalfont{Aut}}(\mathbb{D})$.
\end{thm}

\begin{proof}
  Supponiamo per assurdo che $f \not\in \text{Aut}(\mathbb{D})$. Possiamo applicare la disuguaglianza di Golusin \ref{golusin} nella forma \eqref{golprimo}, che riscriviamo come
  $$\frac{1+|f^h(0)|}{\bigl(1-|f^h(0)|\bigr)\bigl(1+|f^h(z_n)|\bigr)}\bigl(1-|f^h(z_n)|\bigr) \ge \frac{(1-|z_n|)^2}{(1+|z_n|)^2}.$$
  Per ipotesi vale \eqref{n_o^2}, dunque
  $$\frac{1+|f^h(0)|}{\bigl(1-|f^h(0)|\bigr)\bigl(1+|f^h(z_n)|\bigr)}o\bigl((|z_n|-1)^2\bigr) \ge \frac{(1-|z_n|)^2}{(1+|z_n|)^2}$$
  da cui
  $$\frac{\bigl(1+|f^h(0)|\bigr)(1+|z_n|)^2}{\bigl(1-|f^h(0)|\bigr)\bigl(1+|f^h(z_n)|\bigr)}o(1) \ge 1.$$
  Se $f \not\in \text{Aut}(\mathbb{D})$, per il lemma di Schwarz-Pick si ha necessariamente $|f^h(0)|<1$, dunque $\displaystyle \lim_{n \longrightarrow +\infty} \frac{\bigl(1+|f^h(0)|\bigr)(1+|z_n|)^2}{\bigl(1-|f^h(0)|\bigr)\bigl(1+|f^h(z_n)|\bigr)}=\frac{2\bigl(1+|f^h(0)|\bigr)}{1-|f^h(0)|} < +\infty$ e otteniamo una contraddizione.
\end{proof}

Siamo ora pronti a dimostrare il Theorem 2.1 di \cite{BK}.
