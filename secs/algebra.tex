\begin{defn}
  Sia $f:\mathbb{C}^n \longrightarrow \mathbb{C}$ olomorfa, $V=\{z \in \mathbb{C}^n \mid f(z)=0\}$ è uno \textit{spazio analitico}. Se $f \in \mathbb{C}[z_1,\dots,z_n]$ (cioè è un polinomio), $V$ è detta \textit{varietà algebrica}.
\end{defn}

Idea: associare a $V \subset \mathbb{C}^n$ l'insieme $I(V)=\{g \in \mathcal{O}(\mathbb{C}^n) \mid g\restrict{V} \equiv 0\}$ che è un ideale dell'anello $\mathcal{O}(\mathbb{C}^n)$. Alle proprietà geometriche di $V$ corrisponderanno le proprietà algebriche di $I(V)$.

Sia $\mathcal{O}_0=\mathbb{C}\{z_1,\dots,z_n\}$ lo spazio dei germi in $0 \in \mathbb{C}^n$ do funzioni olomorfe (spazio delle serie di potenze convergenti in $0$).

\begin{prop}
  $\mathcal{O}_0$ è un anello \textit{locale} (cioè ha un unico ideale massimale).
\end{prop}

\begin{proof}
  $\mathfrak{m}_0=\{\underline{f} \in \mathcal{O}_0 \mid \underline{f}(0)=0\}$ è un ideale.
  Se $\underline{g} \not\in \mathfrak{m}_0$, $\underline{g}(0)\not=0 \implies 1/\underline{g} \in \mathcal{O}_0 \implies \underline{g}$ è un'unità (elemento invertibile) di $\mathcal{O}_0$. Quindi nessun ideale proprio può contenere elementi di $\mathcal{=}_0\setminus\mathfrak{m}_0$ $\implies$ $\mathfrak{m}_0$ è l'unico ideale massimale.
\end{proof}

\begin{defn}
  Sia $\displaystyle \underline{f}=\sum_{\alpha \in \mathbb{N}^n} a_{\alpha}z^{\alpha} \in \mathcal{O}_0$. L'\textsc{ordine} di $\underline{f}$ è $ord{\underline{f}}=\min\{|\alpha| \mid a_{\alpha}\not=0\}$.
\end{defn}

Chiaramente $\underline{f}$ è un'unità $\iff$ $ord{\underline{f}}=0$. Diremo che $\underline{f}$ è normalizzata (rispetto a $z_n$) se $a_{(0,\dots,0,ord{\underline{f}})}=1$, cioè se $\displaystyle \underline{f}=\sum_{|\alpha|=ord{\underline{f}}}a_{\alpha}z^{\alpha}+O(|\|z\|^{ord{\underline{f}}+1})=z_n^{ord{\underline{f}}}+O(z_1,\dots,z_{n-1})+O(|\|z\|^{ord{\underline{f}}+1})$.

\begin{exc}
  Per ogni $\underline{f} \in \mathcal{O}_0$ esiste $A \in GL(n,\mathbb{C})$ t.c. $\underline{f}\circ A$ sia normalizzata rispetto a $z_n$.
\end{exc}

\begin{defn}
  Sia $z=(z',z_n) \in \mathbb{C}^n$ con $z'=(z_1,\dots,z_{n-1})$, $\mathcal{O}_0'=\mathbb{C}\{z_1,\dots,z_{n-1}\}$. Un \textsc{polinomio di Weierstrass} è un polinomio monico $W \in \mathcal{O}_0'[z_n]$ della forma $W(z_n)=z_n^k+a_{k-1}(z')z_n^{k-1}+\dots+a_0(z')$ con $a_{k-1}(0')=\dots=a_0(0')=0$.
\end{defn}

\begin{thm}
  (di preparazione di Weierstrass) Sia $\underline{f} \in \mathcal{O}_0$ normalizzata di ordine $k \ge 0$. Allora esistono unici un'unità $\underline{u} \in \mathcal{O}_0\setminus\mathfrak{m}_0$ e un polinomio di Weierstrass $W$ di grado $k$ t.c. $\underline{f}=\underline{u}W$.
\end{thm}

\begin{proof}
  Per $k=0$ è ovvio ($\underline{f}=\underline{u}$ e $W \equiv 1$). Sia $k \ge 1 \implies \underline{f}(0)=0$.
  Essendo $\underline{f}$ normalizzata, $f(0',z_n)=z_n^k+O(z_n^{k+1})=z_n^k(1+O(z_n))$ ha in $z_n=0$ uno zero di ordine $k$ ed esiste un $r>0$ t.c. $\underline{f}(0',re^{i\theta})\not=0$ per ogni $\theta \in \mathbb{R}$ $\implies$ esiste $\delta>0$ t.c. $\underline{f}(z',re^{i\theta})\not=0$ per ogni $\|z'\|<\delta$ e $\theta \in \mathbb{R}$.
  Il principio dell'argomento ci dice che $\displaystyle z' \longmapsto \frac{1}{2\pi i} \int_{|\zeta|=r} \frac{\frac{\partial f}{\partial z_n}(z',\zeta)}{f(z',\zeta)}\diff\zeta$ conta il numero di zeri in $D(0,r)$ di $f(z',\cdot)$. Dipende con continuità da $z'$ e quindi è costante. Ponendo $z'=0'$, vediamo che la costante è $k$.
  Indichiamo con $\alpha_1(z'),\dots,\alpha_k(z') \in D(0,r)$ gli zeri di $f(z',\cdot)$ ripetuti con molteplicità. $\alpha_1,\dots,\alpha_k$ non sono funzioni ben definite su $\{\|z'\|<\delta\}$ ma per ogni $\varphi \in \mathcal{O}(\overline{D(0,r)})$ la funzione $\displaystyle J_{\varphi}(z')=\sum_{j=1}^k \varphi(\alpha_j(z'))$ è ben definita e olomorfa in $z'$ perché
  $\displaystyle J_{\varphi}(z')=\frac{1}{2\pi i}\int_{|\zeta|=r} \frac{\varphi(\zeta)\frac{\partial f}{\partial z_n}(z',\zeta)}{f(z',\zeta)}\diff\zeta$ per il teorema dei residui. Poniamo $\displaystyle W(z',z_n)=\prod_{j=1}^k (z_n-\alpha_j(z'))=z_n^k-J_1(z')z_n^{k-1}+\dots+(-1)^kT_k(z')$ con $J_1(z')=\alpha_1(z')+\dots+\alpha_k(z'),\dots, J_k(z')=\alpha_1(z')\cdots\alpha_k(z')$.
  $W \in \mathcal{O}_0'[z_n]$ monico; siccome $\alpha_1(0')=\dots=\alpha_k(0')=0$, $W$ è di Weierstrass. Poniamo $u=f/W$. Per ogni $z'$ fissato, $u(z',\cdot)$ è olomorfa fuori da $\alpha_j(z')$; ma per costruzione queste sono singolarità rimovibili $\implies$ per ogni $z'$ si ha $u(z',\cdot) \in \mathcal{O}(\overline{D(0,r)})$.
  Allora $\displaystyle u(z',z_n)=\frac{1}{2\pi i} \int_{|\zeta|=r} \frac{u(z',\zeta)}{\zeta-z_n}\diff\zeta$ e siccome $u$ è olomorfa in un intorno di $\{\|z'\|<\delta\}\times\{|z_n|=r\}$, allora $u \in \mathcal{O}(\{\|z'\|<\delta\}\times\{|z_n|=r\})$ con $u(0)=1$ $\implies$ $u \in \mathcal{O}_0\setminus\mathfrak{m}_0$ come voluto.

  Unicità: prendiamo $u_1W_1=u_2W_2$ $(\star)$. Ponendo $z'=0$ si ha $u_1(0',z_n)z_n^k=u_2(0',z_n)z_n^k \implies u_1(0',z_n)=u_2(0',z_n)$.
  Derivando $(\star)$ rispetto a $z_j$, cioè applicando l'operatore $\dfrac{\partial}{\partial z_j}$, e ponendo $z'=0$ otteniamo $\dfrac{\partial u_1}{\partial z_j}(0',z_n)z_n^k+u_1(0',z_n)\dfrac{\partial W_1}{\partial z_j}(0',z_n)=\dfrac{\partial u_2}{\partial z_j}(0',z_n)z_n^k+u_2(0',z_n)\dfrac{\partial W_2}{\partial z_j}(0',z_n)$.
  Abbiamo inoltre che $\deg_{z_n}{\dfrac{\partial W_h}{\partial z_j}} \le k-1 \implies \dfrac{\partial u_1}{\partial z_j}(0',z_n)=\dfrac{\partial u_2}{\partial z_j}(0',z_n)$. Continuando a derivare si ottiene $u_1 \equiv u_2$.
\end{proof}

\begin{thm}
  (di divisione di Weierstrass) Siano $f \in \mathcal{O}_0$ e $W \in \mathcal{O}_0'[z_n]$ polinomio di Weierstrass. Allora esistono unici $q \in \mathcal{O}_0, r \in \mathcal{O}_0'[z_n]$ con $\deg_{z_n}{r}<\deg_{z_n}{W}$ t.c. $f=qW+r$.
\end{thm}

\begin{proof}
  Sia $k=\deg_{z_n}{W} \ge 1$. Scegliamo $\delta>0,\rho>0$ t.c. $W$ non si annulla in $\{\|z'\|<\delta\}\times\{|z_n|=\rho\}$. Poniamo $\displaystyle q(z',z_n)=\frac{1}{2\pi i}\int_{|\zeta|=\rho} \frac{f(z',\zeta)}{W(z',\zeta)(\zeta-z_n)}\diff\zeta$ e $r=f-qW$.
  $q,r \in \mathcal{O}(B^{n-1}(0,\delta)\times D(0,\rho))$.
  Inoltre $r(z',z_n)=$ \\
  $\displaystyle=\frac{1}{2\pi i} \int_{|\zeta|=\rho}\left[f(z',\zeta)-\frac{W(z',z_n)f(z',\zeta)}{W(z',\zeta)}\right]\frac{\diff\zeta}{\zeta-z_n}=$ \\
  $\displaystyle=\frac{1}{2\pi i}\int_{|\zeta|=\rho} \frac{f(z',\zeta)}{W(z',\zeta)}\left[\frac{W(z',\zeta)-W(z',z_n)}{\zeta-z_n}\right]\diff\zeta=$ \\
  $\displaystyle=\frac{1}{2\pi i} \int_{|\zeta|=\rho} \frac{f(z',\zeta)}{W(z',\zeta)}\left[\frac{\zeta^k-z_n^k+\sum_{j=0}^{k-1} \alpha_j(z')(\zeta^j-z_n^j)}{\zeta-z_n}\right]\diff\zeta$.
  Il termine tra parentesi quadre è un polinomio in $z_n$ (e $\zeta$) di grado $\le k-1$, perciò $r \in \mathcal{O}_0'[z_n]$ di grado $\le k-1$.

  Unicità: $q_1W+r_1=q_2W+r_2 \iff r_1-r_2=(q_1-q_2)W$. Confrontando i gradi in $z_n$, $q_1-q_2 \equiv 0 \implies r_1-r_2 \equiv 0$.
\end{proof}

\begin{ftt}
  $R$ anello a fattorizzazione unica $\implies$ $R[x]$ a fattorizzazione unica.
\end{ftt}

\begin{thm}
  $\mathcal{O}_0$ è un anello a fattorizzazione unica.
\end{thm}

\begin{thm}
  Per induzione su $n$. Per $n=1$ $\underline{f}=z^k\underline{u}$ è una decomposizione unica. Supponiamo per ipotesi induttiva che $\mathcal{O}_0'$ sia a fattorizzazione unica. Sia $\underline{f} \in \mathcal{O}_0$; possiamo supporre $\underline{f}$ normalizzata di ordine $k \ge 1$. Per il teorema di preparazione di Weierstrass abbiamo $\underline{f}=uW$ con $u$ unità e $W$ polinomio di Weierstrass.

  \begin{lm} \label{lemma1}
    $\underline{f}$ è irriducibile in $\mathcal{O}_0$ $\iff$ $W$ è irriducibile in $\mathcal{O}_0'[z_n]$.
  \end{lm}

  Per ipotesi induttiva, $W=W_1\cdots W_r$ è la fattorizzazione in irriducibili in $\mathcal{O}_0'[z_n]$.

  \begin{lm} \label{lemma2}
    Se $p_1, p_2 \in \mathcal{O}_0'[z_n]$ t.c. $p_1p_2=W$ sia di Weierstrass, allora esiste $u \in \mathcal{O}_0'$ unità t.c. $up_1$ e $\frac{1}{u}p_2$ sono di Weierstrass.
  \end{lm}

  Per il lemma \ref{lemma2} possiamo scrivere $f=\tilde{u}W_1\cdots W_r$ con $W_1, \dots, W_r$ polinomi di Weierstrass irriducibili in $\mathcal{O}_0'[z_n]$, quindi per il lemma \ref{lemma1} sono irriducibili in $\mathcal{O}_0$, dunque abbiamo una decomposizione di $f$ in irriducibili.

  Unicità: sia $f=V_1\cdots V_l$ un'altra decomposizione in irriducibili.

  \begin{lm} \label{lemma3}
    $f=g_1g_2$ normalizzata $\implies$ $g_1,g_2$ normalizzati.
  \end{lm}

  Per il teorema di preparazione di Weierstrass, $f=u'W_1'\cdots W_l'$ con $W_1', \dots, W_l'$ di Weierstrass, irriducibili per il lemma \ref{lemma1}. Dall'Unicità del teorema di preparazione di Weierstrass otteniamo che $W_1\cdots W_r=W_1'\cdots W_l'$ in $\mathcal{O}_0'[z_n]$ che è a fattorizzazione unica, perciò $\{W_1,\dots,W_r\}=\{W_1',\dots,W_l'\}$ a meno di unità.
\end{thm}

\begin{oss}
  Sia $V=\{f=0\}$. La fattorizzazione unica ci dice che possiamo scrivere $f=uf_1\cdots f_k$ con $u$ unità e $f_1, \dots, f_k$ unici e irriducibili, perciò $V=\{f_1=0\}\cup\dots\cup\{f_k=0\}$ è la decomposizione in \textit{componenti irriducibili}.
\end{oss}

Usando il teorema di divisione di Weierstrass è possibile dimostrare il seguente teorema.

\begin{thm}
  $\mathcal{O}_0$ è noetheriano (cioè ogni ideale è finitamente generato).
\end{thm}

\begin{oss}
  Il teorema appena enunciato ci permette di dire che gli ideali $I(V)$, almeno localmente, si possono descrivere con un numero finito di elementi.
\end{oss}

Concludiamo queste dispense con un lemma che segue dal teorema di divisione di Weierstrass.

\begin{lm}
  Siano $f \in \mathcal{O}_0'[z_n]$, $W \in \mathcal{O}_0'[z_n]$ di Weierstrass, $g \in \mathcal{O}_0$ t.c. $f=gW$. Allora $g \in \mathcal{O}_0'[z_n]$.
\end{lm}
