Vediamo ora alcuni risultati analoghi a quelli che conosciamo per le funzioni olomorfe in una variabile. Abbiamo già usato uno di questi risultati, il principio di identità, nella sezione precedente. Inizialmente prendiamo come definizione di funzione olomorfa quella di analitica, poi tra le altre cose vedremo l'equivalenza con le altre definizioni date.

\begin{lm}
  (Abel) Data $\{c_{\alpha}\}_{\alpha \in \mathbb{N}^n} \subset \mathbb{C}$, supponiamo che esistano $\rho_1, \dots, \rho_n>0$ ($\underline{\rho}=(\rho_1, \dots, \rho_n)$), $M>0$ t.c. $|c_{\alpha}|\rho_1^{\alpha_1}\dots\rho_n^{\alpha_n} \le M$ per ogni $\alpha \in \mathbb{N}^n$.
  Allora $\displaystyle \sum_{\alpha \in \mathbb{N}^n} c_{\alpha}(z-z^0)^{\alpha}$ converge assolutamente in $P(z^0, \underline{\rho})$ e uniformemente in $\overline{P(z^0, \theta\underline{\rho})}$ per ogni $0<\theta<1$.
  Inoltre, la stessa convergenza vale per $\displaystyle \sum_{\alpha \in \mathbb{N}^n} c_{\alpha} \dfrac{\partial^{(\beta)}}{\partial z^\beta}(z-z^0)^\alpha$ per ogni $\beta \in \mathbb{N}^n$.
\end{lm}

\begin{cor} \label{coeff_anal_multi}
  $\mathcal{O}(\Omega) \subset C^\infty(\Omega)$ e $c_\alpha=\dfrac{1}{\alpha!}\dfrac{\partial^{(\alpha)} f}{\partial z^\alpha}(z_0)$.
\end{cor}

\begin{cor}
  $f \in \mathcal{O}(\Omega)$ $\implies$ $\bar{\partial}f \equiv 0$.
\end{cor}

\begin{proof}
  $\displaystyle f(z)=f(z^0)+\sum_{j=1}^n \dfrac{\partial f}{\partial z_j}(z^0)(z_j-z_j^0)+o(\|z-z^0\|)$. Il pezzo $\displaystyle f(z^0)+\sum_{j=1}^n \dfrac{\partial f}{\partial z_j}(z^0)(z_j-z_j^0)$ ha ovviamemte $\bar{\partial}(\dots)=0$ perché sono termini costanti o lineari nelle variabili $z_j$, mentre per un $o$-piccolo di termini di ordine $1$ qualunque derivata in $z^0$ vale $0$.
\end{proof}

\begin{cor} \label{olo_var}
  $f \in \mathcal{O}(\Omega)$ $\implies$ $f$ è olomorfa in ciascuna variabile.
\end{cor}

\begin{prop}
  (Principio di identità) Sia $\Omega \subseteq \mathbb{C}^n$ un dominio e $f \in \mathcal{O}(\Omega)$. Se $\{z \in \Omega \mid f(z)=0\}$ ha parte interna non vuota, $f \equiv 0$.
\end{prop}

\begin{proof}
  Per ogni $\alpha \in \mathbb{N}^n$, sia $E_\alpha=\left\{z \in \Omega \mid \dfrac{\partial^{(\alpha)} f}{\partial z^\alpha}(z)=0\right\}$ e $\displaystyle E=\bigcap_{\alpha \in \mathbb{N}^n} E_\alpha$.
  $E$ è ovviamente chiuso, ma è anche aperto per analiticità e per il corollario \ref{coeff_anal_multi}, ed è non vuoto per ipotesi, dunque essendo $\Omega$ connesso $E=\Omega$.
\end{proof}

\begin{prop}
  (Formula di Cauchy) Sia $\Omega \subset \mathbb{C}^n$ un dominio, $f \in \mathcal{O}(\Omega)$, $\overline{P(z^0,\underline{r})} \subset \Omega$. Allora per ogni $z \in P(z^0, \underline{r})$ $\displaystyle f(z)=$\\
  $\displaystyle =\dfrac{1}{(2\pi i)^n} \int_{|\zeta_1-z_1^0|=r}\dots\int_{|\zeta_n-z_n^0|=r} \frac{f(\zeta_1,\dots,\zeta_n)}{(\zeta_1-z_1)\dots(\zeta_n-z_n)}\diff\zeta_1\dots\diff\zeta_n=: \dfrac{1}{(2\pi i)^n} \int_{|\zeta-z^0|=\underline{r}} \dfrac{f(\zeta)}{(\zeta-z)}\diff \zeta$.
\end{prop}

\begin{proof}
  Facciamo per semplicità il caso $n=2$, il caso generico è analogo. Se $z_2$ è fissato, $z_1 \longmapsto f(z_1, z_2)$ è olomorfa in $D(z^0_1, r_1)$ per il corollario \ref{olo_var}. Allora Cauchy in una variabile ci dà $\displaystyle f(z_1,z_2)=\frac{1}{2\pi i} \int_{|\zeta_1-z^0_1|=r_1} \frac{f(\zeta_1, z_2)}{\zeta_1-z_1}\diff\zeta_1$.
  Se $\zeta_1 \in \partial D(z^0_1, r_1)$ è fissato, $z_2 \longmapsto f(\zeta_1,z_2)$ è olomorfa in $D(z^0_2, r_2)$ sempre per il corollario \ref{olo_var}.
  Allora di nuovo per Cauchy in una variabile abbiamo $\displaystyle f(z_1, z_2)=\frac{1}{(2\pi i)^2} \int_{|\zeta_1-z_1^0|=r_1} \frac{\diff \zeta_1}{\zeta_1-z_1}\int_{|\zeta_2-z_2^0|=r_2} \frac{f(\zeta_1,\zeta_2)}{\zeta_2-z_2}\diff\zeta_2$, da cui la tesi.
\end{proof}

\begin{oss}
  $\hat{\partial}P:=\{|\zeta-z^0|=\underline{r}\} \subset \partial P(z^0, \underline{r})$, in particolare è strettamente contenuto. È chiamato \textit{bordo di Šilov di $P(z^0, \underline{r})$}.
\end{oss}

\begin{cor}
  Sia $f \in \mathcal{O}(\Omega)$, $\overline{P(z^0, \underline{r})} \subset \Omega, \alpha \in \mathbb{N}^n$. Allora $\displaystyle \frac{\partial^{(\alpha)} f}{\partial z^\alpha}=\dfrac{\alpha!}{(2\pi i)^n} \int_{|\zeta-z^0|=\underline{r}} \frac{f(\zeta)}{(\zeta-z)^{\alpha+1}}\diff \zeta$ su $P(z^0, \underline{r})$.
\end{cor}

\begin{proof}
  Basta derivare la formula di Cauchy.
\end{proof}

\begin{prop} \label{c->olo_multi}
  Sia $f: \Omega \longrightarrow \mathbb{C}$ $C^0$ in ciascuna variabile, localmente limitata e t.c. per ogni $z^0 \in \Omega$ esiste $\bar{r}$ t.c. $\overline{P(z^0, \underline{r})} \subset \Omega$ e per ogni $z \in P(z^0, \underline{r})$ $\displaystyle f(z)=\frac{1}{(2\pi i)^n} \int_{|\zeta-z^0|=\underline{r}} \frac{f(\zeta)}{(\zeta-z)}\diff\zeta$. Allora $f \in \mathcal{O}(\Omega)$,
\end{prop}

\begin{proof}
  $\displaystyle \frac{1}{\zeta-z}=\sum_{\alpha \in \mathbb{N}^n} \frac{(z-z^0)^\alpha}{(\zeta-z^0)^{\alpha+1}} \implies$\\
  $\displaystyle \implies f(z)=\sum_{\alpha \in \mathbb{N}^n} \left[\int_{|\zeta-z^0|=\underline{r}} \frac{f(\zeta)}{(\zeta-z)^{\alpha+1}}\diff \zeta\right](z-z^0)^\alpha$.
\end{proof}

\begin{cor}
  (Osgood) Se $f:\Omega \longrightarrow \mathbb{C}$ è olomorfa in ciascuna variabile e localmente limitata, $f \in \mathcal{O}(\Omega)$.
\end{cor}

\begin{proof}
  Sia $\overline{P(z^0, \underline{r})} \subset \Omega$. $f$ olomorfa in ciascuna variabile $\implies$ $\displaystyle f(z)=\frac{1}{(2\pi i)^n} \int_{|\zeta_1-z_1^0|=r_1} \frac{\diff \zeta_1}{\zeta_1-z_1}\dots\int_{|z_n-z_n^0|=r_n}\frac{f(\zeta)}{\zeta_n-z_n}\diff\zeta_n$, ma essendo $f$ limitata questo ci dice che vale la formula di Cauchy, allora che $f$ è olomorfa segue dalla proposizione \ref{c->olo_multi}.
\end{proof}

\begin{ftt}
  Hartogs ha dimostrato che non serve l'ipotesi localmente limitata.
\end{ftt}

\begin{cor}
  Se $f \in C^1(\Omega)$ e $\bar{\partial}f\equiv 0$, $f \in \mathcal{O}(\Omega)$.
\end{cor}

\begin{proof}
  $f$ è olomorfa in ciascuna variabile.
\end{proof}

\begin{prop}
  (Disuguaglianze di Cauchy) $f \in \mathcal{O}(\Omega), \overline{P(z^0, \underline{r})} \subset \Omega, \alpha \in \mathbb{N}^n \implies \left|\dfrac{\partial^{(\alpha)} f}{\partial z^\alpha}(z^0)\right| \le \dfrac{\alpha!}{\underline{r}^\alpha}M(\underline{r})$ dove $\displaystyle M(\underline{r})=\sup_{|\zeta-z^0|=\underline{r}} |f(\zeta)|$.
\end{prop}

\begin{proof}
  $\displaystyle \dfrac{\partial^{(\alpha)} f}{\partial z^\alpha}(z^0)=\dfrac{\alpha!}{(2\pi i)^n} \int_0^{2\pi} \dots \int_0^{2\pi} \frac{f(z^0+\underline{r}e^{i\theta})}{(\underline{r}e^{i\theta})^{\alpha+1}}i^n\left(\prod_{j=1}^n r_je^{i\theta_j}\right)\diff\theta_1\dots\diff\theta_n=$\\
  $\displaystyle =\frac{\alpha!}{(2\pi)^n\underline{r}^\alpha}\int_0^{2\pi} \dots \int_0^{2\pi} \frac{f(z^0+\underline{r}e^{i\theta})}{e^{i(\alpha_1\theta_1+\dots+\alpha_n\theta_n)}}\diff\theta_1\dots\diff\theta_n \implies$ \\
  $\displaystyle \implies \left|\dfrac{\partial^{(\alpha)} f}{\partial z^\alpha}(z^0)\right| \le \frac{\alpha!}{(2\pi)^n\underline{r}^\alpha}\int_0^{2\pi} \dots \int_0^{2\pi} |f(z^0+\underline{r}e^{i\theta})|\diff\theta_1\dots\diff\theta_n \le \dfrac{\alpha!}{\underline{r}^\alpha}M(\underline{r})$.
\end{proof}

\begin{cor}
  (Liouville) Se $f \in \mathcal{O}(\mathbb{C}^n)$ è limitata, allora è costante.
\end{cor}

\begin{proof}
  Per ogni $\alpha \not=0$, $\left|\dfrac{\partial^{(\alpha)} f}{\partial z^\alpha}(z^0)\right| \le \dfrac{\alpha!}{\underline{r}^\alpha}M$ per ogni $\underline{r}$ $\implies$ tutte le derivate di $f$ sono $\equiv 0$.
\end{proof}

\begin{thm}
  (Applicazione aperta) $f \in \mathcal{O}(\Omega)$ non costante $\implies$ $f$ è aperta.
\end{thm}

\begin{proof}
  Basta far vedere che $f(\Omega)$ è aperto. Sia $z^0 \in \Omega$ e sia $U \subset \Omega$ un intorno convesso di $z^0$. Siccome $f$ non è costante, $f\restrict{U} \not\equiv f(z^0)$ (nel senso che non è costantemente uguale, altrimenti sarebbe costante per il principio di identità).
  Sia allora $\tilde{z}^0 \in U$ con $f(\tilde{z}^0)\not=f(z^0)$ e $D=\{\zeta \in \mathbb{C} \mid z^0+\zeta(\tilde{z}^0-z^0) \in U\} \subset \mathbb{C}$ aperto.
  Definiamo $g(\zeta)=f(z^0+\zeta(\tilde{z}^0-z^0))$; abbiamo $g \in \mathcal{O}(D)$ e non costante perché $g(0)\not=g(1)$, quindi $g(D)$ è aperto in $\mathbb{C}$ per il principio di identità e contiene $g(0)=f(z^0)$. Quindi $f(\Omega) \supseteq g(D)$, che è un intorno di $f(z^0)$.
\end{proof}

\begin{thm}
  (Principio del massimo) Sia $\Omega \subset\subset \mathbb{C}^n$ dominio limitato, $f \in \mathcal{O}(\Omega)$ non costante. Sia $\displaystyle M=\sup_{x \in \partial \Omega} \limsup_{z \longrightarrow x, z \in \Omega} |f(z)|$. Allora $|f(z)|<M$ per ogni $z \in \Omega$.
\end{thm}

\begin{proof}
  Se $M=+\infty$ è ovvio. Supponiamo $M<+\infty$. Sia $\varphi: \overline{\Omega} \longrightarrow \mathbb{R}^+\cup\{0\}$ data da $\varphi(x)=\begin{cases}
    |f(x)| & \mbox{se }x\in\Omega \\ \limsup_{z \longrightarrow x, z \in \Omega} |f(z)| & \mbox{se }x \in \partial\Omega
\end{cases}$. $\varphi$ è semicontinua superiormente su $\overline{\Omega}$ che è compatto $\implies$ $\varphi$ è limitata (in particolare, assume massimo).
Sia $D=f(\Omega) \subset \mathbb{C}$. $D$ è un aperto (connesso) limitato (perche $\varphi$ è limitata) di $\mathbb{C}$. Sia $\tau \in \partial D$, esiste $\{\zeta_{\nu}\} \subset D$ t.c. $\sigma_{\nu} \longrightarrow \tau$. Esiste $z_{\nu} \in \Omega$ t.c. $f(z_{\nu})=\zeta_{\nu}$.
A meno di sottosuccessioni, $z_{\nu} \longrightarrow z^0 \in \overline{\Omega}$. Se avessimo $z^0 \in \Omega$, allora $f(z^0) \in D$, ma $f(z^0)=\tau \in \partial D$, assurdo.
Quindi $z^0 \in \partial\Omega \implies |\tau| \le M \implies \partial D \subset \overline{D(0,M)} \implies \overline{D} \subset \overline{D(0,M)}$. Siccome $D$ è aperto, $D \subseteq D(0,M)$, che è la tesi.
\end{proof}

\begin{cor}
  Sia $f \in \mathcal{O}(\Omega)$. Se esiste $z^0 \in \Omega$ dominio t.c. $|f(z)| \le |f(z^0)|$ per ogni $z$ in un intorno di $z^0$, allora $f$ è costante.
\end{cor}

\begin{proof}
  Se $f$ non è costante, si applica il principio del massimo a un intorno di $z^0$ e si ottiene un assurdo.
\end{proof}

Dei seguenti tre teoremi non riportiamo le dimostrazioni, in quanto analoghe a quelle in una variabile.

\begin{thm}
  (Weierstrass) Sia $\{f_{\nu}\} \subseteq \mathcal{O}(\Omega)$ che converge uniformemente sui compatti a $f \in C^0(\Omega)$. Allora $f \in \mathcal{O}(\Omega)$ e per ogni $\alpha \in \mathbb{N}^n$ $\dfrac{\partial^{(\alpha)}f_{\nu}}{\partial z^\alpha} \longrightarrow \dfrac{\partial^{(\alpha)}f}{\partial z^\alpha}$.
\end{thm}

\begin{thm}
  (Montel) $\mathcal{F} \subset \mathcal{O}(\Omega)$ è relativamente compatta in $\mathcal{O}(\Omega)$ se (e solo se) è uniformemente limitata sui compatti, cioè per ogni $K \subset\subset \Omega$ compatto esiste $M_K$ t.c. $\|f\|_K<M_K$ per ogni $f \in \mathcal{F}$.
\end{thm}

\begin{thm}
  (Vitali) Sia $\{f_{\nu}\} \subset \mathcal{O}(\Omega)$ uniformemente limitata sui compatti e sia $A \subseteq \Omega$ un insieme con parte interna non vuota t.c. $\{f_{\nu}(z^0)\}$ converge in $\mathbb{C}$ per ogni $z^0 \in A$. Allora esiste $f \in \mathcal{O}(\Omega)$ t.c. $f_{\nu} \longrightarrow f$ uniformemente sui compatti.
\end{thm}
