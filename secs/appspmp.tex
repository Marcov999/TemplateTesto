Vediamo ora alcune applicazioni dei risultati visti nella sezione precedente.

Apriamo la sezione con il lemma di Dieudonné \cite[Chapitre III, paragraphe 8, équation (25)]{D}, per il quale l'approccio dell'articolo di Beardon e Minda semplifica la dimostrazione.

\begin{lm}
  (lemma di Dieudonné) Sia $f \in \text{Hol}(\mathbb{D},\mathbb{D})$ tale che $f(0)=0$ e sia $z_0 \in \mathbb{D}$. Allora
  \begin{equation} \label{dieu}
    |f'(z_0)-f(z_0)/z_0| \le \frac{|z_0|^2-|f(z_0)|^2}{|z_0|(1-|z_0|^2)}.
  \end{equation}
  In particolare,
  \begin{equation} \label{dieuprimo}
    |f'(z)| \le \begin{cases}
      1 & \mbox{se } |z| \le \sqrt{2}-1 \\
      \dfrac{(1+|z|^2)^2}{4|z|(1-|z|^2)} & \mbox{se } |z| \ge \sqrt{2}-1.
  \end{cases}
  \end{equation}
\end{lm}

\begin{proof}
  Per il Teorema \ref{31} con $z=v=z_0$ e $w=0$ abbiamo
  \begin{align*}
    \omega\bigl(f^h(z_0),f^*(0,z_0)\bigr) & \le \omega(0,z_0) \\
    \iff p\bigl(f^h(z_0),f^*(0,z_0)\bigr) & \le p(0,z_0)=|z_0|,
  \end{align*}
  dove l'equivalenza fra le due disuguaglianze segue dal fatto che $\text{arctanh}$ è strettamente crescente. Per semplificare, scriviamo $f^h(z_0)=a, f^*(0,z_0)=b$ e $|z_0|=r$. Vogliamo portare la disuguaglianza in forma euclidea. Abbiamo
  $$\left|\frac{a-b}{1-\bar{b}a}\right|=p(a,b) \le r,$$
  che si riscrive come
  \begin{align*}
    (a-b)(\bar{a}-\bar{b}) & \le r^2(1-\bar{b}a)(1-b\bar{a}) \\
    & \iff |a|^2(1-r^2|b|^2)-a\bar{b}(1-r^2)-\bar{a}b(1-r^2) \le r^2-|b|^2 \\
    & \iff |a|^2-a\cdot\frac{\bar{b}(1-r^2)}{1-r^2|b|^2}-\bar{a}\cdot\frac{b(1-r^2)}{1-r^2|b|^2} \le \frac{r^2-|b|^2}{1-r^2|b|^2};
  \end{align*}
  ponendo $\alpha=\dfrac{b(1-r^2)}{1-r^2|b|^2}$ e $R^2=\dfrac{r^2-|b|^2}{1-r^2|b|^2}+|\alpha|^2$, si ha
  $$|a|^2-a\bar{\alpha}-\bar{a}\alpha \le R^2-|\alpha|^2 \iff (a-\alpha)(\bar{a}-\bar{\alpha}) \le R^2\iff |a-\alpha| \le R.$$
  Ricordando che $r=|z_0|$ e osservando che $b=f^*(0,z_0)=\frac{[f(0),f(z_0)]}{[0,z_0]}=\frac{f(z_0)}{z_0}$, troviamo $\alpha=\dfrac{f(z_0)(1-|z_0|^2)}{z_0\bigl(1-|f(z_0)|^2\bigr)}$ e $R=\dfrac{|z_0|^2-|f(z_0)|^2}{|z_0|\bigl(1-|f(z_0)|^2\bigr)}$.
  Riprendendo infine la definizione di $a$, cioè $a=f^h(z_0)=\frac{f'(z_0)(1-|z_0|^2)}{1-|f(z_0)|^2}$, otteniamo che
  $$\left|\frac{f'(z_0)(1-|z_0|^2)}{1-|f(z_0)|^2}-\frac{f(z_0)(1-|z_0|^2)}{z_0\bigl(1-|f(z_0)|^2\bigr)}\right| \le \frac{|z_0|^2-|f(z_0)|^2}{|z_0|\bigl(1-|f(z_0)|^2\bigr)},$$
  che è equivalente alla \eqref{dieu} moltiplicando entrambi i membri per $\frac{1-|f(z_0)|^2}{1-|z_0|^2}$.

  Mostriamo ora la \eqref{dieuprimo}. Dalla \eqref{dieu} con $z$ al posto di $z_0$ si ha
  \begin{align*}
    |f'(z)| & \le |f'(z)-f(z)/z|+|f(z)/z| \\
    & \le \frac{|z|^2-|f(z)|^2}{|z|(1-|z|^2)}+\frac{|f(z)|}{|z|}=\frac{(|f(z)|+|z|^2)(1-|f(z)|)}{|z|(1-|z|^2)}.
  \end{align*}
  Adesso, per $|z|$ fissato un rapido conto mostra che il massimo si ottiene per $|f(z)|=\frac{1-|z|^2}{2}$; poiché per il lemma di Schwarz dev'essere $|f(z)| \le |z|$, questo caso può essere raggiunto solo se $|z| \ge \sqrt{2}-1$ e sostituendo otteniamo la seconda espressione della \eqref{dieuprimo}. Nell'altro caso il massimo si ottiene per $|f(z)|=|z|$ e sostituendo si trova che è proprio $1$.
\end{proof}

\begin{thm} \label{distortion}
  Dato $b \in [0,1)$, scriviamo $F_b(z)=\dfrac{z(z+b)}{1+b z}$. Consideriamo $f \in \normalfont{\text{Hol}}(\mathbb{D},\mathbb{D})$ tale che $f(0)=0$. Se $|f'(0)|<1$, allora per ogni $z \in \mathbb{D}$ si ha
  \begin{equation}
    \left|\frac{f^h(0)-f^h(z)}{1-\overline{f^h(0)}f^h(z)}\right| \le \frac{2|z|}{1+|z|^2}
  \end{equation}
  e
  \begin{equation}
    F_{|f^h(0)|}^h(-|z|) \le |f^h(z)| \le F_{|f^h(0)|}^h(|z|).
  \end{equation}
\end{thm}

\begin{proof}
  Poiché $|f'(0)|<1$, per il lemma di Schwarz si ha $f \not\in \text{Aut}(\mathbb{D})$. Inoltre $f(0)=0$, perciò possiamo applicare il Corollario \ref{36}; si ha dunque
  $$\omega\bigl(f^h(0),f^h(z)\bigr) \le 2\omega(0,z).$$
  Applicando la tangente iperbolica, sfruttando l'uguaglianza $\tanh(2x)=\frac{2\tanh{x}}{1+\tanh^2{x}}$ e ricordando la definizione di $\omega$ si ha
  $$p\bigl(f^h(0),f^h(z)\bigr) \le \frac{2p(0,z)}{1+p^2(0,z)},$$
  da cui
  $$\left|\frac{f^h(0)-f^h(z)}{1-\overline{f^h(0)}f^h(z)}\right| \le \frac{2|z|}{1+|z|^2}.$$

  Per dimostrare la seconda disuguaglianza, supponiamo dapprima che si abbia $f^h(0)=b \in [0,1)$. Possiamo ripetere i passaggi svolti nella dimostrazione del lemma di Dieudonné ponendo $a=f^h(z)$ e $r=\frac{2|z|}{1+|z|^2}$. Otteniamo la disuguaglianza $|f^h(z)-\alpha| \le R$, dove $\alpha=\dfrac{b(1-r^2)}{1-r^2b^2}$ e $R^2=\dfrac{r^2-b^2}{1-r^2b^2}+\alpha^2$. Sostituendo troviamo
  $$\alpha=\frac{b(1-|z|^2)^2}{(1+2b|z|+|z|^2)(1-2b|z|+|z|^2)},$$
  $$R=\frac{2|z|(|z|^2+1)(1-b^2)}{(1+2b|z|+|z|^2)(1-2b|z|+|z|^2)}.$$
  Consideriamo adesso $F_b^h(z)=\dfrac{bz^2+2z+b}{|z|^2+2b\,\mathfrak{Re}z+1}\left(\dfrac{|1+b z|}{1+b z}\right)^2$. Si ha
  $$F_b^h(|z|)=\dfrac{b|z|^2+2|z|+b}{|z|^2+2b|z|+1} \text{ e } F_b^h(-|z|)=\dfrac{b|z|^2-2|z|+b}{|z|^2-2b|z|+1}.$$
  Notiamo che $\alpha=\bigl(F_b^h(|z|)+F_b^h(-|z|)\bigr)/2$ e $R=\bigl(F_b^h(|z|)-F_b^h(-|z|)\bigr)/2$, perciò la disuguaglianza $|f^h(z)-\alpha| \le R$ ci dice che $f^h(z)$ appartiene al cerchio con diametro sull'asse reale passante per i punti $F_b^h(|z|)$ e $F_b^h(-|z|)$. Con semplici considerazioni geometriche otteniamo la seguente disuguaglianza:
  $$F_b^h(-|z|) \le \mathfrak{Re}f^h(z) \le |f^h(z)| \le F_b^h(|z|),$$
  la quale, ricordando che $b=f^h(0)$, ci dà
  $$F_{f^h(0)}^h(-|z|) \le |f^h(z)| \le F_{f^h(0)}^h(|z|).$$
  Per passare al caso generale consideriamo la funzione $g(z)=|f^h(0)|f(z)/f'(0)$. Osserviamo che $f(0)=0$ ci dice che $f'(0)=f^h(0)$, dunque $|g(z)|=|f(z)|$ e $|g'(z)|=|f'(z)|$, pertanto $|g^h(z)|=|f^h(z)|$; inoltre si ha anche $g(0)=0$, da cui $g^h(0)=g'(0)=|f^h(0)|$. Perciò applicando l'ultima disuguaglianza trovata alla funzione $g$ otteniamo proprio la seconda disuguaglianza della tesi.
\end{proof}

\begin{cor} \label{distorto}
  Sia $f \in \text{Hol}(\mathbb{D},\mathbb{D})$ tale che $f(0)=0$ e $f'(0) \in [0,1)$. Allora $\mathfrak{Re}f'(z)>0$ per $|z|<f^h(0)/\Bigl(1+\sqrt{1-\bigl(f^h(0)\bigr)^2}\Bigr)$.
\end{cor}

\begin{proof}
  Per $0 \le b<1$ e $z \in \mathbb{D}$ si ha $|z|^2-2b|z|+1>|z|^2-2|z|+1>0$, dunque abbiamo che il segno di $F_b^h(-|z|)$ coincide con quello di $b|z|^2-2|z|+b$. Quest'ultima quantità è minore di $0$ per $|z| \in \bigl((1-\sqrt{1-b^2})/b, (1+\sqrt{1-b^2})/b\bigr)$, zero agli estremi e maggiore di $0$ altrove.
  Prendendo $b=f'(0)=f^h(0)$, nella dimostrazione del Teorema \ref{distortion} abbiamo visto che $\mathfrak{Re}f^h(z) \ge F_{f^h(0)}^h(-|z|)$; per gli $z$ tali che $|z|<\Bigl(1-\sqrt{1-\bigl(f^h(0)\bigr)^2}\Bigr)/f^h(0)=f^h(0)/\Bigl(1+\sqrt{1-\bigl(f^h(0)\bigr)^2}\Bigr)$ si ha quindi $\mathfrak{Re}f^h(z)>0$.
  Ricordando che $f^h(z)=\frac{f'(z)(1-|z|^2)}{1-|f(z)|^2}$ e $f \in \text{Hol}(\mathbb{D},\mathbb{D})$, per tali $z$ si ha anche $\mathfrak{Re}f'(z)>0$.
\end{proof}

Vediamo ora il risultato che, come già anticipato, ci permetterà di dimostrare i teoremi successivi. L'enunciato originale si trova in \cite{GMG}, ma vedremo una formulazione che ci tornerà più utile, in particolare perché coinvolge la funzione $f^h$.

\begin{thm} \label{golusin}
  (disuguaglianza di Golusin, 1945) Sia $f \in \text{\normalfont{Hol}}(\mathbb{D},\mathbb{D})\setminus\text{\normalfont{Aut}}(\mathbb{D})$. Allora per ogni $z \in \mathbb{D}$ vale
  \begin{equation} \label{gol}
    |f^h(z)| \le \frac{|f^h(0)|+\frac{2|z|}{1+|z|^2}}{1+|f^h(0)|\frac{2|z|}{1+|z|^2}}.
  \end{equation}
\end{thm}

\begin{proof}
  Con passaggi analoghi a quelli della dimostrazione del Corollario \ref{quasigolusin} abbiamo che valgono le seguenti uguaglianze:
  \begin{gather*}
    \omega\bigl(|f^h(z)|,|f^h(0)|\bigr)=\frac{1}{2}\log\left(\frac{1+|f^h(z)|}{1-|f^h(z)|}\cdot\frac{1-|f^h(0)|}{1+|f^h(0)|}\right)\\
    \omega(z, 0)=\omega(|z|,0)=\frac{1}{2}\log\left(\frac{1+|z|}{1-|z|}\right).
  \end{gather*}
  Prendendo $w=0$ nella disuguaglianza \eqref{quasigol} otteniamo
  $$\frac{1}{2}\log\left(\frac{1+|f^h(z)|}{1-|f^h(z)|}\cdot\frac{1-|f^h(0)|}{1+|f^h(0)|}\right) \le \log\left(\frac{1+|z|}{1-|z|}\right),$$
  da cui
  \begin{equation}
    \frac{1+|f^h(z)|}{1-|f^h(z)|} \le \frac{1+|f^h(0)|}{1-|f^h(0)|}\left(\frac{1+|z|}{1-|z|}\right)^2. \label{golprimo}
  \end{equation}
  Adesso, dalla Proposizione \ref{24} sappiamo che $f^h(z),f^h(0) \in \mathbb{D}$, in particolare $|f^h(z)|,|f^h(0)|<1$, perciò è giustificato il seguente passaggio:
  \begin{align*}
    |f^h(z)| & \le \frac{\frac{1+|f^h(0)|}{1-|f^h(0)|}\left(\frac{1+|z|}{1-|z|}\right)^2-1}{\frac{1+|f^h(0)|}{1-|f^h(0)|}\left(\frac{1+|z|}{1-|z|}\right)^2+1} \\
    & =\frac{(1+|f^h(0)|)(1+2|z|+|z|^2)-(1-|f^h(0)|)(1-2|z|+|z|^2)}{(1+|f^h(0)|)(1+2|z|+|z|^2)+(1-|f^h(0)|)(1-2|z|+|z|^2)} \\
    & =\frac{2|f^h(0)|+2|f^h(0)||z|^2+4|z|}{2+2|z|^2+4|f^h(0)||z|}=\frac{|f^h(0)|+\frac{2|z|}{1+|z|^2}}{1+|f^h(0)|\frac{2|z|}{1+|z|^2}}.
  \end{align*}
\end{proof}
