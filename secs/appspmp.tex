Vediamo ora alcune applicazioni dei risultati visti nella sezione precedente.

\begin{thm} \label{distortion}
  Dato $b \in [0,1)$, scriviamo $F_b(z)=\dfrac{z(z+b)}{1+b z}$. Consideriamo $f \in \normalfont{\text{Hol}}(\mathbb{D},\mathbb{D})$ tale che $f(0)=0$. Se $|f'(0)|<1$, allora per ogni $z \in \mathbb{D}$ si ha
  \begin{equation}
    \left|\frac{f^h(0)-f^h(z)}{1-f^h(0)f^h(z)}\right| \le \frac{2|z|}{1+|z|^2}
  \end{equation}
  e
  \begin{equation}
    F_{|f^h(0)|}^h(-|z|) \le |f^h(z)| \le F_{|f^h(0)|}^h(|z|).
  \end{equation}
\end{thm}

\begin{proof}
  Poiché $|f'(0)|<1$, per il lemma di Schwarz si ha $f \not\in \text{Aut}(\mathbb{D})$. Inoltre $f(0)=0$, perciò possiamo applicare il Corollario \ref{36}; si ha dunque
  $$\omega\bigl(f^h(0),f^h(z)\bigr) \le 2\omega(0,z).$$
  Applicando la tangente iperbolica, sfruttando l'uguaglianza $\tanh(2x)=\frac{2\tanh{x}}{1+\tanh^2{x}}$ e ricordando la definizione di $\omega$ si ha
  $$p\bigl(f^h(0),f^h(z)\bigr) \le \frac{2p(0,z)}{1+p^2(0,z)},$$
  da cui
  $$\left|\frac{f^h(0)-f^h(z)}{1-f^h(0)f^h(z)}\right| \le \frac{2|z|}{1+|z|^2}.$$

  Per dimostrare la seconda disuguaglianza, supponiamo dapprima che si abbia $f^h(0)=b \in [0,1)$. Possiamo ripetere i passaggi svolti nella dimostrazione del lemma di Dieudonné ponendo $a=f^h(z)$ e $r=\frac{2|z|}{1+|z|^2}$. Otteniamo la disuguaglianza $|f^h(z)-\alpha| \le R$, dove $\alpha=\dfrac{b(1-r^2)}{1-r^2b^2}$ e $R^2=\dfrac{r^2-b^2}{1-r^2b^2}+\alpha^2$. Sostituendo troviamo
  $$\alpha=\frac{b(1-|z|^2)^2}{(1+2b|z|+|z|^2)(1-2b|z|+|z|^2)},$$
  $$R=\frac{2|z|(|z|^2+1)(1-b^2)}{(1+2b|z|+|z|^2)(1-2b|z|+|z|^2)}.$$
  Consideriamo adesso $F_b^h(z)=\dfrac{bz^2+2z+b}{|z|^2+2b\,\mathfrak{Re}z+1}\left(\dfrac{|1+b z|}{1+b z}\right)^2$. Si ha
  $$F_b^h(|z|)=\dfrac{b|z|^2+2|z|+b}{|z|^2+2b|z|+1} \text{ e } F_b^h(-|z|)=\dfrac{b|z|^2-2|z|+b}{|z|^2-2b|z|+1}.$$
  Notiamo che $\alpha=\bigl(F_b^h(|z|)+F_b^h(-|z|)\bigr)/2$ e $R=\bigl(F_b^h(|z|)-F_b^h(-|z|)\bigr)/2$, perciò la disuguaglianza $|f^h(z)-\alpha| \le R$ ci dice che $f^h(z)$ appartiene al cerchio con diametro sull'asse reale passante per i punti $F_b^h(|z|)$ e $F_b^h(-|z|)$. Con semplici considerazioni geometriche otteniamo la seguente disuguaglianza:
  $$F_b^h(-|z|) \le \mathfrak{Re}f^h(z) \le |f^h(z)| \le F_b^h(|z|),$$
  la quale, ricordando che $b=f^h(0)$, ci dà
  $$F_{f^h(0)}^h(-|z|) \le |f^h(z)| \le F_{f^h(0)}^h(|z|).$$
  Per passare al caso generale consideriamo la funzione $g(z)=|f^h(0)|f(z)/f'(0)$. Osserviamo che $f(0)=0$ ci dice che $f'(0)=f^h(0)$, dunque $|g(z)|=|f(z)|$ e $|g'(z)|=|f'(z)|$, pertanto $|g^h(z)|=|f^h(z)|$; inoltre si ha anche $g(0)=0$, da cui $g^h(0)=g'(0)=|f^h(0)|$. Perciò applicando l'ultima disuguaglianza trovata alla funzione $g$ otteniamo proprio la seconda disuguaglianza della tesi.
\end{proof}

\begin{cor} \label{distorto}
  Sia $f \in \text{Hol}(\mathbb{D},\mathbb{D})$ tale che $f(0)=0$ e $f'(0) \in [0,1)$. Allora $\mathfrak{Re}f'(z)>0$ per $|z|<f^h(0)/\Bigl(1+\sqrt{1-\bigl(f^h(0)\bigr)^2}\Bigr)$.
\end{cor}

\begin{proof}
  Per $0 \le b<1$ e $z \in \mathbb{D}$ si ha $|z|^2-2b|z|+1>|z|^2-2|z|+1>0$, dunque abbiamo che il segno di $F_b^h(-|z|)$ coincide con quello di $b|z|^2-2|z|+b$. Quest'ultima quantità è minore di $0$ per $|z| \in \bigl((1-\sqrt{1-b^2})/b, (1+\sqrt{1-b^2})/b\bigr)$, zero agli estremi e maggiore di $0$ altrove.
  Prendendo $b=f'(0)=f^h(0)$, nella dimostrazione del Teorema \ref{distortion} abbiamo visto che $\mathfrak{Re}f^h(z) \ge F_{f^h(0)}^h(-|z|)$; per gli $z$ tali che $|z|<\Bigl(1-\sqrt{1-\bigl(f^h(0)\bigr)^2}\Bigr)/f^h(0)=f^h(0)/\Bigl(1+\sqrt{1-\bigl(f^h(0)\bigr)^2}\Bigr)$ si ha quindi $\mathfrak{Re}f^h(z)>0$.
  Ricordando che $f^h(z)=\frac{f'(z)(1-|z|^2)}{1-|f(z)|^2}$ e $f \in \text{Hol}(\mathbb{D},\mathbb{D})$, per tali $z$ si ha anche $\mathfrak{Re}f'(z)>0$.
\end{proof}

Vediamo ora il risultato che, come già anticipato, ci permetterà di dimostrare i teoremi successivi. L'enunciato originale si trova in \cite{GMG}, ma vedremo una formulazione che ci tornerà più utile, in particolare perché coinvolge la funzione $f^h$.

\begin{thm} \label{golusin}
  (disuguaglianza di Golusin, 1945) Sia $f \in \text{\normalfont{Hol}}(\mathbb{D},\mathbb{D})\setminus\text{\normalfont{Aut}}(\mathbb{D})$. Allora per ogni $z \in \mathbb{D}$ vale
  \begin{equation} \label{gol}
    |f^h(z)| \le \frac{|f^h(0)|+\frac{2|z|}{1+|z|^2}}{1+|f^h(0)|\frac{2|z|}{1+|z|^2}}.
  \end{equation}
\end{thm}

\begin{proof}
  Con passaggi analoghi a quelli della dimostrazione del Corollario \ref{quasigolusin} abbiamo che valgono le seguenti uguaglianze:
  \begin{gather*}
    \omega\bigl(|f^h(z)|,|f^h(0)|\bigr)=\frac{1}{2}\log\left(\frac{1+|f^h(z)|}{1-|f^h(z)|}\cdot\frac{1-|f^h(0)|}{1+|f^h(0)|}\right)\\
    \omega(z, 0)=\omega(|z|,0)=\frac{1}{2}\log\left(\frac{1+|z|}{1-|z|}\right).
  \end{gather*}
  Prendendo $w=0$ nella disuguaglianza \eqref{quasigol} otteniamo
  \begin{align}
    \nonumber \frac{1}{2}\log\left(\frac{1+|f^h(z)|}{1-|f^h(z)|}\cdot\frac{1-|f^h(0)|}{1+|f^h(0)|}\right) \le \log\left(\frac{1+|z|}{1-|z|}\right) \\
    \frac{1+|f^h(z)|}{1-|f^h(z)|} \le \frac{1+|f^h(0)|}{1-|f^h(0)|}\left(\frac{1+|z|}{1-|z|}\right)^2. \label{golprimo}
  \end{align}
  Adesso, dalla Proposizione \ref{24} sappiamo che $f^h(z),f^h(0) \in \mathbb{D}$, in particolare $|f^h(z)|,|f^h(0)|<1$, perciò è giustificato il seguente passaggio:
  \begin{align*}
    |f^h(z)| & \le \frac{\frac{1+|f^h(0)|}{1-|f^h(0)|}\left(\frac{1+|z|}{1-|z|}\right)^2-1}{\frac{1+|f^h(0)|}{1-|f^h(0)|}\left(\frac{1+|z|}{1-|z|}\right)^2+1} \\
    & =\frac{(1+|f^h(0)|)(1+2|z|+|z|^2)-(1-|f^h(0)|)(1-2|z|+|z|^2)}{(1+|f^h(0)|)(1+2|z|+|z|^2)+(1-|f^h(0)|)(1-2|z|+|z|^2)} \\
    & =\frac{2|f^h(0)|+2|f^h(0)||z|^2+4|z|}{2+2|z|^2+4|f^h(0)||z|}=\frac{|f^h(0)|+\frac{2|z|}{1+|z|^2}}{1+|f^h(0)|\frac{2|z|}{1+|z|^2}}.
  \end{align*}
\end{proof}
