\begin{thm}
  Siano $f \in \mathcal{S}$ e $z,w \in \mathbb{D}$. Allora
  \begin{equation}\label{maggiorazione}
    \left|\frac{f(z)-f(w)}{z-w}\right| \le \frac{1-|wz|}{(1-|z|)^2(1-|w|)^2}.
  \end{equation}
\end{thm}

\begin{proof}
  Useremo la maggiorazione del teorema di distorsione \eqref{distorsion}. Sia $\gamma(t)=w+t(z-w)$ il segmento che congiunge $w$ a $z$, allora
  \begin{align*}
    |f(z)-f(w)|&=\left|\int_{\gamma}f'(\zeta)\diff\zeta\right| \le \int_{\gamma} |f'(\zeta)||\diff\zeta| \le \\
    & \le \int_{\gamma} \frac{1+|\zeta|}{(1-|\zeta|)^3}|\diff\zeta|= \\
    &=\int_0^1 \frac{1+|w+t(z-w)|}{\big(1-|w+t(z-w)|\big)^2}|z-w|\diff t.
  \end{align*}
  Adesso, osserviamo che $|w+t(z-w)| \le |w|+t(|z|-|w|)$ e che la funzione $\frac{1+x}{(1-x)^3}$ è crescente per $0 \le x \le 1$, dunque
  \begin{gather*}
    \int_0^1 \frac{1+|w+t(z-w)|}{\big(1-|w+t(z-w)|\big)^2}|z-w|\diff t \le \int_0^1 \frac{1+|w|+t(|z|-|w|)}{\Big(1-\big(|w|+t(|z|-|w|)\big)\Big)^2}|z-w|\diff t,
  \end{gather*}
  ed effettuando il cambio di variabile $s=|w|+t(|z|-|w|)$ troviamo
  \begin{align*}
    \int_0^1 \frac{1+|w|+t(|z|-|w|)}{\Big(1-\big(|w|+t(|z|-|w|)\big)\Big)^2}|z-w|\diff t &=\int_{|w|}^{|z|} \frac{1+s}{(1-s)^3}\frac{|z-w|}{|z|-|w|}\diff s=\\
    &=\frac{|z-w|(1-|zw|)}{(1-|z|)^2(1-|w|)^2}.
  \end{align*}
  Dobbiamo prestare attenzione, perché questo dimostra la tesi per $|z|>|w|$, e per simmetria per $|z|\not=|w|$, ma a questo punto mandando al limite $|z| \longrightarrow |w|$ la otteniamo anche per $|z|=|w|$.
\end{proof}

Una cosa che lascia leggermente insoddisfatti di questa dimostrazione è che noi volevamo generalizzare le disuguaglianze di Koebe, ma ci siamo comunque dovuti appoggiare ad una di esse. Questa non è necessariamente una cosa negativa e rimane comunque un approccio valido, ma proporremo adesso un'altra dimostrazione che si basa sulla stima dei coefficienti $a_n$ di $f$, approccio che vedremo porterà a qualche risultato sulla minorazione passando da $\Sigma$.

Per le funzioni in $\mathcal{S}$ è stato congetturato da Bieberbach (inserire citazione) e poi dimostrato da De Brange (inserire citazione) che vale $|a_n| \le n$. Vediamo come questo ci dà immediatamente la maggiorazione:
\begin{align*}
  \left|\frac{f(z)-f(w)}{z-w}\right|&=\left|\frac{\sum_{n=1}^{+\infty}a_n(z^n-w^n)}{z-w}\right|=\left|\sum_{n=1}^{+\infty}\sum_{j=0}^{n-1}a_nz^jw^{n-1-j}\right| \le \\
  &\le \sum_{n=1}^{+\infty}\sum_{j=0}^{n-1}|a_n||z|^j|w|^{n-1-j} \le \sum_{n=1}^{+\infty}\sum_{j=0}^{n-1}n|z|^j|w|^{n-1-j}= \\
  &=\frac{1-|zw|}{(1-|z|)^2(1-|w|)^2}.
\end{align*}
