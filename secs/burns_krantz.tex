\begin{thm} \label{burns_krantz}
  (Burns-Krantz, 1994) Sia $f:\mathbb{D} \longrightarrow \mathbb{D}$ una funzione olomorfa dal disco in sé tale che
  \begin{equation} \label{O^4}
    f(z)=1+(z-1)+\mathcal{O}((z-1)^4)
  \end{equation}
  per $z \longrightarrow 1$. Allora $f$ è l'identità sul disco.
\end{thm}

\begin{proof}
  Chiaramente, se vale \eqref{O^4} per $z \longrightarrow 1$ vale anche \eqref{o^3}, in particolare per $z \longrightarrow 1$ non tangenzialmente.
  Dalla proposizione \ref{o^3->o^2} segue che anche \eqref{o^2} vale per $z \longrightarrow 1$ non tangenzialmente, quindi esiste una successione $z_n$ che soddisfa le ipotesi del teorema \ref{boundary_schwarz_pick} (usiamo di nuovo che, non tangenzialmente, $|z-1|$ e $1-|z|$ hanno gli stessi $o$-piccoli), dunque $f$ è un automorfismo.
  Allora per la proposizione \ref{aut} esistono $\theta \in \mathbb{R}, a \in \mathbb{D}$ tali che $f(z)=e^{i\theta}[z,a]$. Poiché vale \eqref{O^4}, dev'essere $f''(1)=0$. Un semplice conto mostra che $f''(z)=\dfrac{e^{i\theta}\bar{a}(1-|a|^2)}{(1-\bar{a}z)^3}$.
  Siccome $e^{i\theta}\not=0, |a|<1$, deve necessariamente essere $\bar{a}=0$, perciò $f(z)=e^{i\theta}z$. Il fatto che $f(z)=z$ segue da, sempre per \eqref{O^4}, $\displaystyle \lim_{z \longrightarrow 1} f(z)=1$.
\end{proof}

\begin{ex}
  $f(z)=\dfrac{1+3z^2}{3+z^2}$. Osserviamo che $f$ è una funzione olomorfa su $\mathbb{C} \setminus \{\pm i\sqrt{3}\}$, quindi in particolare è ben definita su $\mathbb{D}$. Verifichiamo che l'immagine è contenuta in $\mathbb{D}$:
  \begin{align*}
    |f(z)|^2<1 \\
    \dfrac{(1+3z^2)(1+3\bar{z}^2)}{(3+z^2)(3+\bar{z}^2)} < 1 \\
    (1+3z^2)(1+3\bar{z}^2) < (3+z^2)(3+\bar{z}^2) \\
    1+3z^2+3\bar{z}^2+9|z|^4 < 9+3z^2+3\bar{z}^2+|z|^4 \\
    1-|z|^4 < 9(1-|z|^4)
  \end{align*}
  e l'ultima disuguaglianza è verificata perché $z \in \mathbb{D} \implies |z|<1 \implies 1-|z|^4>0$. \\
  Ovviamente $f$ non può essere iniettiva su $\mathbb{D}$ perché $f(z)=f(-z)$, dunque non è un automorfismo. Adesso mostriamo che $f(z)-1-(z-1)$ è $\mathcal{O}((z-1)^3)$ ma non $\mathcal{O}((z-1)^4)$ per $z \longrightarrow 1$:
  \begin{align*}
    f(z)-z & =\frac{1+3z^2}{3+z^2}-z \\
    & =\frac{1+3z^2-3z-z^3}{3+z^2} \\
    & =\frac{(1-z)^3}{3+z^2}=:g(z).
  \end{align*}
  Poiché $\displaystyle \lim_{z \longrightarrow 1} g(z)/(z-1)^3=-1/4$, $g(z)$ è $\mathcal{O}((z-1)^3)$ ma non $\mathcal{O}((z-1)^4)$ per $z \longrightarrow 1$. Dunque il termine $\mathcal{O}((z-1)^4)$ nel teorema \ref{burns_krantz} non è migliorabile.
\end{ex}
