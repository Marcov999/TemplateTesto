Purtroppo, come già preannunciato non ci sono molte speranze di dimostrare la congettura nel caso generale passando da $\Sigma$. Il motivo è che la disuguaglianza nelle ipotesi della proposizione \ref{passaggio} non è vera in generale. L'esistenza implicita di un controesempio ci è data dal seguente risultato.

\begin{thm}
  (\cite[Chapter IV.3, Theorem 2]{Gol1}) Siano $\zeta, \eta \in \mathbb{C}\setminus\overline{\mathbb{D}}$ fissati e sia $F \in \Sigma$ che massimizza la quantità
  $$I_F=\mathfrak{Re}\left(-\log\frac{f(\zeta)-f(\eta)}{\zeta-\eta}\right)=\log\left|\frac{f(\zeta)-f(\eta)}{\zeta-\eta}\right|^{-1};$$
  allora l'immagine di $F$ è tutto il piano meno un tratto di iperbole.
\end{thm}

È chiaro che massimizzare $I_F$ ci dà la stima cercata in $\Sigma$. Supponiamo per assurdo che tale massimo sia $\log\left(1-\frac{1}{|\zeta\eta|}\right)^{-1}$ come voluto; è facile vedere che, fissati $\zeta$ e $\eta$, esiste $\sigma \in \partial\mathbb{D}$ tale che, detta $F(z)=z+\sigma^2/z$, allora $I_F$ assume tale valore. Ma questa $F$ è ottenuta da un'opportuna funzione di Koebe, dunque la sua immagine è tutto il piano meno un segmento, contraddizione. \\

Questo non esclude che la disuguaglianza sia vera in $\mathcal{S}$, ma se fosse vera si evidenzierebbe una certa asimmetria tra $\mathcal{S}$ e $\Sigma$ per queste disuguaglianze a due punti. Ciò non deve stupire in quanto, per passare queste disuguaglianze da un insieme all'altro, compaiono dei termini che inevitabilmente rendono la disuguaglianza in arrivo più debole, cioè non c'è equivalenza. Questo si può vedere ad esempio con la maggiorazione: se fosse vera la controparte in $\Sigma$ che implica la disuguaglianza in $\mathcal{S}$, avremmo $\left|\frac{F(\zeta)-F(\eta)}{\zeta-\eta}\right| \le 1-\frac{1}{|\zeta\eta|}$, ma è facile vedere (con lo stesso esempio di prima, derivato dalle Koebe) che questa disuguaglianza non può essere vera.
