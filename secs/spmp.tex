Adesso possiamo procedere a dimostrare la serie di disuguaglianze di \cite{BM}, che coinvolgono la distanza di Poincaré $\omega$ e le funzioni olomorfe dal disco in sé che non sono automorfismi.

\begin{prop} \label{24}
  Siano $f \in \text{\normalfont{Hol}}(\mathbb{D},\mathbb{D})\setminus\text{\normalfont{Aut}}(\mathbb{D})$ e $v \in \mathbb{D}$. Allora per ogni $z \in \mathbb{D}$ si ha che $f^*(z,v) \in \mathbb{D}$ e la funzione $z \longmapsto f^*(z,v)$ è olomorfa.
\end{prop}

\begin{proof}
  Per quanto riguarda l'olomorfia, dalla definizione sappiamo che l'unico punto che potrebbe dar problemi è $v$; abbiamo però visto che la funzione ammette limite finito per $z \longrightarrow v$, perciò $v$ è una singolarità rimovibile. Per il lemma di Schwarz-Pick, $|f^*(z,w)| \le 1$; inoltre, vale l'uguaglianza in qualche punto solo se $f$ è un automorfismo. Dunque le ipotesi su $f$ assicurano che vale la disuguaglianza stretta sempre, cioè $f^*(z,v) \in \mathbb{D}$ per ogni $z \in \mathbb{D}$.
\end{proof}

\begin{thm} \label{31}
  (Beardon-Minda, 2004) Sia $f \in \text{\normalfont{Hol}}(\mathbb{D},\mathbb{D})\setminus\text{\normalfont{Aut}}(\mathbb{D})$. Allora per ogni $z, w, v \in \mathbb{D}$ vale
  \begin{equation} \label{3.1}
    \omega\bigl(f^*(z,v),f^*(w,v)\bigr) \le \omega(z,w).
  \end{equation}
\end{thm}

\begin{proof}
  Poiché $f$ non è un automorfismo, per la Proposizione \ref{24} la funzione $z \longmapsto f^*(z,v)$ è olomorfa dal disco unitario in sé; perciò il membro sinistro della disuguaglianza \eqref{3.1} è ben definito e la tesi segue dal lemma di Schwarz-Pick e dall'osservazione \ref{oss1}, punto (ii).
\end{proof}

\begin{cor} \label{32}
  Sia $f \in \text{\normalfont{Hol}}(\mathbb{D},\mathbb{D})\setminus\text{\normalfont{Aut}}(\mathbb{D})$. Allora per ogni $z, w, v \in \mathbb{D}$ vale
  \begin{equation}
    \omega\bigl(0, f^*(z,v)\bigr) \le \omega\bigl(0,f^*(w,v)\bigr)+\omega(z,w).
  \end{equation}
\end{cor}

\begin{proof}
  Si ha
  \begin{align*}
    \omega\bigl(0,f^*(z,v)\bigr) & \le \omega\bigl(0,f^*(w,v)\bigr)+\omega\bigl(f^*(w,v),f^*(z,v)\bigr) \\
    & \le \omega\bigl(0,f^*(w,v)\bigr)+\omega(z,w),
  \end{align*}
  dove la prima è la disuguaglianza triangolare per la distanza $\omega$ e la seconda segue dal Teorema \ref{31}.
\end{proof}

\begin{cor} \label{33}
  Sia $f \in \text{\normalfont{Hol}}(\mathbb{D},\mathbb{D})\setminus\text{\normalfont{Aut}}(\mathbb{D})$. Allora per ogni $z, w, v, u \in \mathbb{D}$ vale
  \begin{equation}
    \omega\bigl(0, f^*(z,v)\bigr) \le \omega\bigl(0, f^*(u,w)\bigr)+\omega(z,w)+\omega(v,u).
  \end{equation}
\end{cor}

\begin{proof}
  Si ha
  \begin{align*}
    \omega\bigl(0,f^*(z,v)\bigr) & \le \omega\bigl(0,f^*(w,v)\bigr)+\omega(z,w) \\
    & =\omega\bigl(0,|f^*(w,v)|\bigr)+\omega(z,w) \\
    & =\omega\bigl(0,|f^*(v,w)|\bigr)+\omega(z,w) \\
    & =\omega\bigl(0,f^*(v,w)\bigr)+\omega(z,w) \\
    & \le \omega\bigl(0,f^*(u,w)\bigr)+\omega(z,w)+\omega(v,u),
  \end{align*}
  dove le due disuguaglianze seguono dal Corollario \ref{32}.
\end{proof}

I due enunciati seguenti non ci serviranno nel seguito, ma vengono riportati per completezza. Prima di enunciare il primo, è necessario dare una definizione.

\begin{defn}
  Una \textit{geodetica} per $\omega$ è una curva $\sigma: \mathbb{R} \longrightarrow \mathbb{D}$ tale che per ogni $t_1,t_2 \in \mathbb{R}$ si ha $\omega\bigl(\sigma(t_1),\sigma(t_2)\bigr)=|t_1-t_2|$.
\end{defn}

\begin{cor} \label{35}
  Sia $f \in \normalfont{\text{Hol}}(\mathbb{D},\mathbb{D})$ e siano $z, w \in \mathbb{D}$. Sia $\sigma$ una geodetica con $\sigma(t_1)=z, \sigma(t_2)=v$ e sia $w=\sigma(t)$ con $t_1<t<t_2$. Allora
  \begin{equation} \label{geod}
    2\omega\bigl(f(z),f(v)\bigr) \le \log\Bigl(\cosh\bigl(2\omega(z,v)\bigr)+|f^h(w)|\sinh\bigl(2\omega(z,v)\bigr)\Bigr).
  \end{equation}
\end{cor}

\begin{proof}
  Osserviamo che se $f \in \text{Aut}(\mathbb{D})$, allora per il lemma di Schwarz-Pick $|f^h(w)|=1$ e il membro destro della disuguaglianza \eqref{geod} è esattamente $2\omega(z,v)$. In questo caso, per il lemma di Schwarz-Pick si ha proprio l'uguaglianza.

  Supponiamo ora $f \not\in \text{Aut}(\mathbb{D})$, allora possiamo applicare il Corollario \ref{33} con $u=v$ per ottenere
  \begin{gather*}
    \omega\bigl(0,f^*(z,v)\bigr) \le \omega\bigl(0,f^h(w)\bigr)+\omega(z,v) \\
    p\bigl(0,f^*(z,v)\bigr) \le \tanh\Bigl(\omega\bigl(0,f^h(w)\bigr)+\omega(z,v)\Bigr) \\
    \frac{p\bigl(f(z),f(v)\bigr)}{p(z,v)} \le \frac{|f^h(w)|+p(z,v)}{1+|f^h(w)|p(z,v)},
  \end{gather*}
  dove abbiamo usato $p=\tanh\omega$, $f^*(z,v)=\frac{[f(z),f(v)]}{[z,v]}$, $\tanh(a+b)=\frac{\tanh{a}+\tanh{b}}{1+\tanh{a}\tanh{b}}$ e $p(0,\zeta)=|\zeta|$. Riscriviamo come
  \begin{gather*}
    \frac{p\bigl(f(z),f(v)\bigr)}{p(z,v)}-p(z,v) \le |f^h(w)|\Bigl(1-p\bigl(f(z),f(v)\bigr)\Bigr) \\
    \frac{p\bigl(f(z),f(v)\bigr)-p^2(z,v)}{p(z,v)\Bigl(1-p\bigl(f(z),f(v)\bigr)\Bigr)} \le |f^h(w)| \\
    \frac{2\Bigl(p\bigl(f(z),f(v)\bigr)-p^2(z,v)\Bigr)}{\bigl(1-p^2(z,v)\bigr)\Bigl(1-p\bigl(f(z),f(v)\bigr)\Bigr)} \le |f^h(w)|\cdot \frac{2p(z,v)}{1-p^2(z,v)}.
  \end{gather*}
  Adesso usiamo le seguenti uguaglianze:
  $$\frac{1+p^2}{1-p^2}=\cosh(2\omega), \quad \frac{2p}{1-p^2}=\sinh(2\omega).$$
  Sommando appunto la quantità $\dfrac{1+p^2(z,v)}{1-p^2(z,v)}$ all'ultima disuguaglianza ottenuta, il membro destro diventa $\cosh\bigl(2\omega(z,v)\bigr)+|f^h(w)|\sinh\bigl(2\omega(z,v)\bigr)$. Ci basta dunque mostrare che il membro sinistro è uguale a $\exp\Bigl(2\omega\bigl(f(z),f(v)\bigr)\Bigr)$, cioè $\dfrac{1+p\bigl(f(z),f(v)\bigr)}{1-p\bigl(f(z),f(v)\bigr)}$. Si ha infatti
  \begin{align*}
    & \frac{2\Bigl(p\bigl(f(z),f(v)\bigr)-p^2(z,v)\Bigr)}{\bigl(1-p^2(z,v)\bigr)\Bigl(1-p\bigl(f(z),f(v)\bigr)\Bigr)}+\frac{1+p^2(z,v)}{1-p^2(z,v)}= \\
    & = \frac{p\bigl(f(z),f(v)\bigr)-p^2(z,v)+1-p^2(z,v)p\bigl(f(z),f(v)\bigr)}{\bigl(1-p^2(z,v)\bigr)\Bigl(1-p\bigl(f(z),f(v)\bigr)\Bigr)} \\
    & = \frac{\bigl(1-p^2(z,v)\bigr)\Bigl(1+p\bigl(f(z),f(v)\bigr)\Bigr)}{\bigl(1-p^2(z,v)\bigr)\Bigl(1-p\bigl(f(z),f(v)\bigr)\Bigr)}=\frac{1+p\bigl(f(z),f(v)\bigr)}{1-p\bigl(f(z),f(v)\bigr)}.
  \end{align*}
\end{proof}

\begin{cor} \label{36}
  Sia $f \in \text{\normalfont{Hol}}(\mathbb{D},\mathbb{D})\setminus\text{\normalfont{Aut}}(\mathbb{D})$ tale che $f(0)=0$. Allora
  \begin{equation}
    \omega\bigl(f^h(0),f^h(z)\bigr) \le 2\omega(0,z).
  \end{equation}
  Inoltre, $2$ è la migliore costante possibile.
\end{cor}

\begin{proof}
  Notiamo che $f(0)=0 \implies f^*(z,0)=f^*(0,z)$, dunque si ha
  \begin{align*}
    \omega\bigl(f^h(0),f^h(z)\bigr) & = \omega\bigl(f^*(0,0),f^*(z,z)\bigr) \\
    & \le \omega\bigl(f^*(0,0),f^*(z,0)\bigr)+\omega\bigl(f^*(0,z),f^*(z,z)\bigr) \\
    & \le 2\omega(0,z)
  \end{align*}
  dove la prima è la disuguaglianza triangolare per $\omega$ e la seconda segue applicando il Teorema \ref{31}.

  Per dire che $2$ è la migliore costante possibile, basta prendere $f(z)=z^2$ e $z \in \mathbb{D}$ con $|z|=1/3$ per ottenere l'uguaglianza.
\end{proof}

Il prossimo risultato è quello che ci permetterà di dimostrare la disuguaglianza di Golusin.

\begin{cor} \label{quasigolusin}
  Sia $f \in \text{\normalfont{Hol}}(\mathbb{D},\mathbb{D})\setminus\text{\normalfont{Aut}}(\mathbb{D})$. Allora per ogni $z, w \in \mathbb{D}$ vale
  \begin{equation} \label{quasigol}
    \omega\bigl(|f^h(z)|, |f^h(w)|\bigr) \le 2\omega(z,w).
  \end{equation}
\end{cor}

\begin{proof}
  Siano $z, w \in \mathbb{D}$; senza perdita di generalità possiamo supporre $|f^h(z)| \ge |f^h(w)|$. Allora
  \begin{align*}
    \omega\bigl(|f^h(z)|, |f^h(w)|\bigr) & =\frac{1}{2}\log\left(\frac{1+\frac{|f^h(z)|-|f^h(w)|}{1-|f^h(w)||f^h(z)|}}{1-\frac{|f^h(z)|-|f^h(w)|}{1-|f^h(w)||f^h(z)|}}\right) \\
    & =\frac{1}{2}\log\left(\frac{1-|f^h(w)||f^h(z)|+|f^h(z)|-|f^h(w)|}{1-|f^h(w)||f^h(z)|+|f^h(w)|-|f^h(z)|}\right) \\
    & =\frac{1}{2}\log\left(\frac{1+|f^h(z)|}{1-|f^h(z)|}\cdot\frac{1-|f^h(w)|}{1+|f^h(w)|}\right) \\
    & =\frac{1}{2}\log\left(\frac{1+|f^h(z)|}{1-|f^h(z)|}\right)-\frac{1}{2}\log\left(\frac{1+|f^h(w)|}{1-|f^h(w)|}\right) \\
    & =\omega\bigl(0,|f^h(z)|\bigr)-\omega\bigl(0,|f^h(w)|\bigr) \\
    & =\omega\bigl(0,f^h(z)\bigr)-\omega\bigl(0,f^h(w)\bigr) \le 2\omega(z,w),
  \end{align*}
  dove l'ultima disuguaglianza segue dal Corollario \ref{33} prendendo $u=w$ e $v=z$.
\end{proof}

La sezione si conclude con due lemmi sulle funzioni olomorfe dal disco in sé, per i quali l'approccio dal punto di vista dell'articolo di Beardon e Minda semplifica le dimostrazioni.

\marginpar{aggiungere delle reference, vedasi [9] e [18] tra le reference di \cite{BM}}

Notazione: dati $E \subset \mathbb{D}$ e $z \in \mathbb{D}$, scriviamo $zE=\{zw \mid w \in E\}$. Inoltre, dati $\gamma \subset \mathbb{D}$ e $r>0$, scriviamo $\Sigma(\gamma,r)=\{w \mid \omega(z,w)<r, z \in \gamma\}$.

\begin{lm}
  (lemma di Rogosinski) Sia $f \in \normalfont{\text{Hol}}(\mathbb{D},\mathbb{D})\setminus\normalfont{\text{Aut}}(\mathbb{D},\mathbb{D})$ tale che $f(0)=0$ e $f'(0) \in \mathbb{R}$. Allora per ogni $z \in \mathbb{D}$ si ha $f(z) \in z\Sigma\bigl((-1,1),\omega(0,z)\bigr)$.
\end{lm}

\begin{proof}
  Sia $g$ definita come nella dimostrazione del lemma di Schwarz, il quale ci dice anche che $g \in \text{Hol}(\mathbb{D},\mathbb{D})$. La tesi è vera per $z=0$, supponiamo dunque $z\not=0$. Per il lemma di Schwarz-Pick si ha
  \begin{gather*}
    \omega\bigl(g(0),g(z)\bigr) \le \omega(0,z) \\
    \omega\bigl(f'(0),f(z)/z\bigr) \le \omega(0,z).
  \end{gather*}
  Per il lemma di Schwarz dev'essere $f'(0) \in (-1,1)$. Abbiamo quindi che $f(z)/z \in \Sigma\bigl((-1,1),\omega(0,z)\bigr) \implies f(z) \in z\Sigma\bigl((-1,1),\omega(0,z)\bigr)$, come voluto.
\end{proof}

\begin{lm}
  (lemma di Dieudonné) Siano $z_0,w_0 \in \mathbb{D}$ con $|w_0| \le |z_0|$ e sia $f \in \text{Hol}(\mathbb{D},\mathbb{D})$ tale che $f(0)=0$ e $f(z_0)=w_0$. Allora
  \begin{equation}
    |f'(z_0)-w_0/z_0| \le \frac{|z_0|^2-|w_0|^2}{|z_0|(1-|z_0|^2)}.
  \end{equation}
\end{lm}

\begin{proof}
  Per il Teorema \ref{31} con $z=v=z_0$ e $w=0$ abbiamo
  \begin{align*}
    \omega\bigl(f^h(z_0),f^*(0,z_0)\bigr) & \le \omega(0,z_0) \\
    \iff p\bigl(f^h(z_0),f^*(0,z_0)\bigr) & \le p(0,z_0)=|z_0|,
  \end{align*}
  dove l'equivalenza fra le due disuguaglianze segue dal fatto che $\text{arctanh}$ è strettamente crescente. Per semplificare, scriviamo $f^h(z_0)=a, f^*(0,z_0)=b, |z_0|=r$. Vogliamo portare la disuguaglianza in forma euclidea. Abbiamo
  \begin{align*}
    p(a,b) & \le r \\
    & \iff \left|\frac{a-b}{1-\bar{b}a}\right| \le r \\
    & \iff (a-b)(\bar{a}-\bar{b}) \le r^2(1-\bar{b}a)(1-b\bar{a}) \\
    & \iff |a|^2-a\bar{b}-\bar{a}b+|b|^2 \le r^2-r^2a\bar{b}-r^2\bar{a}b+r^2|b|^2|a|^2 \\
    & \iff |a|^2(1-r^2|b|^2)-a\bar{b}(1-r^2)-\bar{a}b(1-r^2) \le r^2-|b|^2 \\
    & \iff |a|^2-a\cdot\frac{\bar{b}(1-r^2)}{1-r^2|b|^2}-\bar{a}\cdot\frac{b(1-r^2)}{1-r^2|b|^2} \le \frac{r^2-|b|^2}{1-r^2|b|^2} \\
    & \iff (a-\alpha)(\bar{a}-\bar{\alpha}) \le R^2 \\
    & \iff |a-\alpha| \le R,
  \end{align*}
  dove $\alpha=\dfrac{b(1-r^2)}{1-r^2|b|^2}$ e $R^2=\dfrac{r^2-|b|^2}{1-r^2|b|^2}+|b|^2\left(\dfrac{1-r^2}{1-r^2|b|^2}\right)^2$.
  Ricordando che $r=|z_0|$ e osservando che $b=f^*(0,z_0)=\frac{[f(0),f(z_0)]}{[0,z_0]}=\frac{[0,w_0]}{[0,z_0]}=\frac{w_0}{z_0}$, troviamo $\alpha=\dfrac{w_0(1-|z_0|^2)}{z_0(1-|w_0|^2)}$ e $R=\dfrac{|z_0|^2-|w_0|^2}{|z_0|(1-|w_0|^2)}$.
  Riprendendo infine la definizione di $a$, cioè $a=f^h(z_0)=\frac{f'(z_0)(1-|z_0|^2)}{1-|f(z_0)|^2}=\frac{f'(z_0)(1-|z_0|^2)}{1-|w_0|^2}$, otteniamo che
  $$\left|\frac{f'(z_0)(1-|z_0|^2)}{1-|w_0|^2}-\frac{w_0(1-|z_0|^2)}{z_0(1-|w_0|^2)}\right| \le \frac{|z_0|^2-|w_0|^2}{|z_0|(1-|w_0|^2)},$$
  che è equivalente alla tesi moltiplicando entrambi i membri per $\frac{1-|w_0|^2}{1-|z_0|^2}$.
\end{proof}
