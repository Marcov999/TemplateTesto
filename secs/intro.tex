A volte può capitare che un problema delle olimpiadi di matematica possa essere risolto più facilmente mediante l'utilizzo di argomenti più tecnici o avanzati. Bisogna stare molto attenti però nel presentare una soluzione che faccia uso di certi strumenti di teoria in quanto i correttori, soprattutto a partire dal livello nazionale in su, sono molto severi e tendono a una votazione del tipo tutto o niente: la mancata verifica di un'ipotesi o un'applicazione non corretta della tesi rischiano di portare quella che sarebbe benissimo una soluzione perfetta a prendere pochissimi punti, se non nessuno. Tuttavia, se uno sa come sfruttarli, risultati e teoremi avanzati possono regalare soluzioni quasi gratuite o svelare l'idea che si cela dietro una soluzione completamente elementare, aprendo la via alla stessa, che potrebbe risultare più facile da scrivere e sarebbe sicuramente meno rischiosa.

In queste dispense voglio occuparmi di analisi, dunque di quegli argomenti che trovano la loro applicazione nei problemi delle olimpiadi di algebra e teoria dei numeri. Ci sarebbero un sacco di cose da dire, quindi è poco probabile che riesca a coprire tutto ciò che può essere d'aiuto a un concorrente delle olimpiadi, ma cercherò di trattare la maggior parte degli argomenti più utili. \\

Buona lettura!
