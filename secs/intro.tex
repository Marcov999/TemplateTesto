L'obiettivo di questo scritto è dimostrare un teorema del 1994, il teorema di Burns-Krantz (teorema 2.1 di \cite{BK}), attraverso risultati elementari. L'enunciato del teorema riguarda le funzioni olomorfe sul disco unitario con un certo andamento vicino a $1$: se la funzione dista dall'identità al più per un $\mathcal{O}((z-1)^4)$, allora è proprio l'identità.

La dimostrazione originale del teorema non è lunga, ma un po' tecnica. In un recente articolo di Bracci, Kraus e Roth (\cite{BKR}) si trova una dimostrazione alternativa del teorema di Burns-Krantz. Come spiegato nel remark 2.2 dell'articolo, è possibile passare dalle ipotesi del teorema di Burns-Krantz a quelle del teorema 2.1 di \cite{BKR} (si veda la proposizione 8.1 dello stesso articolo), dal quale poi è facile concludere. Il teorema 2.1 è sostanzialmente una versione al bordo del lemma di Schwarz-Pick. \\

Bracci, Kraus e Roth dimostrano il teorema 2.1 usando risultati più generali visti nell'articolo, ma complicati. Tuttavia, nel remark 5.6 danno una traccia per una dimostrazione più elementare. L'idea è sfruttare una disuguaglianza dovuta a Golusin e vengono indicati vari articoli in cui è stata ridimostrata.

In particolare, l'articolo di Beardon e Minda del 2004 (\cite{BM}) dimostra una serie di disuguaglianze, delle quali il corollario 3.7 ha a sua volta come corollario la disuguaglianza di Golusin. Queste disuguaglianze coinvolgono la distanza iperbolica sul disco unitario. In effetti, i vari risultati più generali di \cite{BK} riguardano pseudometriche sul disco e lo stesso teorema 2.1 può essere riformulato in termini della distanza iperbolica, come spiegato nell'articolo.
