Questi appunti sono basati sul corso Elementi di Analisi Complessa tenuto dal professor Marco Abate nel secondo semestre dell'anno accademico 2019/2020. Sono dati per buoni (si vedano i prerequisiti del corso) analisi in più variabili, topologia, concetto di gruppo fondamentale e le basi di analisi complessa in più variabili, che si vedono nei corsi Analisi 2 e Geometria 2. Verranno omesse o soltanto hintate le dimostrazioni più semplici, ma si consiglia comunque di provare a svolgerle per conto proprio. Ogni tanto sarà commesso qualche abuso di notazione, facendo comunque in modo che il significato sia reso chiaro dal contesto. Inoltre, la notazione verrà alleggerita man mano, per evitare inutili ripetizioni e appesantimenti nella lettura. Si ricorda anche che questi appunti sono scritti non sempre subito dopo le lezioni, non sempre con appunti completi, ecc\dots. Spesso saranno rivisti, verranno aggiunte cose che mancavano perché c'era poco tempo (o voglia\dots), potrebbero mancare argomenti più o meno marginali\dots insomma, non è un libro di testo per il corso, ma vuole essere un valido supporto per aiutare gli studenti che seguono il corso. Spero di essere riuscito in questo intento.
