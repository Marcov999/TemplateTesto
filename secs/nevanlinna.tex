Dedichiamo una sottosezione al seguente risultato, dimostrato indipendentemente da Pick nel 1916 e Nevanlinna nel 1919; è un teorema di interpolazione interessante di per sé, e vedremo un paio di esempi in cui le ipotesi vengono riformulate in termini del lemma di Schwarz-Pick, usando in un caso anche il rapporto iperbolico.

\begin{thm}
  (Pick-Nevanlinna, \cite[Chapter 1, Theorem 2.2]{JBG}) Siano dati $n$ punti distinti $z_1, \dots, z_n \in \mathbb{D}$ e altri $n$ punti distinti (non necessariamente diversi dai primi) $w_1, \dots, w_n \in \mathbb{D}$. Consideriamo la forma quadratica
  $$A_n(t_1,\dots,t_n)=\sum_{i,j=1}^n\frac{1-w_i\bar{w}_j}{1-z_i\bar{z}_j}t_i\bar{t}_j.$$
  Allora esiste una funzione $f \in \normalfont{\text{Hol}}(\mathbb{D},\mathbb{D})$ tale che $f(z_i)=w_i$ per $j=1, \dots, n$ se e solo se $A_n$ è definita positiva. In tal caso, si può trovare $f$ che sia un prodotto di Blaschke di grado al più $n$.
\end{thm}

\begin{proof}
  Procediamo per induzione su $n$. Il caso $n=1$ è banale per transitività di $\text{Aut}(\mathbb{D})$. Supponiamo adesso $n>1$. Poniamo
  $$z_i'=\frac{z_i-z_n}{1-\bar{z}_nz_i} \,\, \text{e} \,\, w_i'=\frac{w_i-w_n}{1-\bar{w}_nw_i} \,\, \text{per} \,\, 1 \le i \le n.$$
  Allora esiste $f$ olomorfa dal disco in sé che risolve l'interpolazione se e solo se la funzione
  $$g(z)=\frac{\Bigg(f\left(\dfrac{z+z_n}{1+\bar{z}_nz}\right)-w_n\Bigg)}{1-\bar{w}_n\Bigg(f\left(\dfrac{z+z_n}{1+\bar{z}_nz}\right)\Bigg)}$$
  appartiene a $\text{Hol}(\mathbb{D},\mathbb{D})$ e soddisfa $g(z_i')=w_i'$ per $1 \le i \le n$. Infatti, si tratta solo di comporre con i giusti automorfismi. La forma quadratica $A_n'$ definita con i punti $z_i',w_i'$ è legata alla forma $A_n$. Per mostrarlo, poniamo
  $$\frac{1-z_i'\bar{z}_j'}{1-z_i\bar{z}_j}=\frac{1-|z_n|^2}{(1-\bar{z}_nz_i)(1-z_n\bar{z}_j)}=\alpha_i\bar{\alpha}_j$$
  e
  $$\frac{1-w_i'\bar{w}_j'}{1-w_i\bar{w}_j}=\frac{1-|w_n|^2}{(1-\bar{w}_nw_i)(1-w_n\bar{w}_j)}=\beta_i\bar{\beta}_j,$$
  dove $\alpha_i=\frac{\sqrt{1-|z_n|^2}}{1-\bar{z}_nz_i}$ per $1 \le i \le n$ e analogamente per i $\beta_i$. Allora si ha
  \begin{align*}
    A_n'(t_1,\dots,t_n) & =\sum_{i,j=1}^n \frac{1-w_i'\bar{w}_j'}{1-z_i'\bar{z}_j'}t_i\bar{t}_j \\
    & =\sum_{i,j=1}^n \frac{1-w_i\bar{w}_j}{1-z_i\bar{z}_j}\left(\frac{\beta_i}{\alpha_i}t_i\right)\left(\frac{\bar{\beta}_j}{\bar{\alpha}_i}\bar{t}_j\right)=A_n\left(\frac{\beta_1}{\alpha_1}t_1,\dots,\frac{\beta_n}{\alpha_n}t_n\right).
  \end{align*}
  Poiché $z_i,w_i \in \mathbb{D}$ si ha $\alpha_i,\beta_i\not=0$; perciò $A_n$ è definita positiva se e solo se lo è $A_n'$. Dato che $z_n'=w_n'=0$, a meno di cambiare $f$ con $g$ possiamo supporre senza perdita di generalità $z_n=w_n=0$. La condizione dell'enunciato diventa dunque $f(0)=0$ e $f(z_i)=w_i$ per $1 \le i \le n-1$.
  Tale funzione esiste se e solo se la funzione $h(z)=f(z)/z$, con $h(0)=f'(0)$, appartiene a $\text{Hol}(\mathbb{D},\mathbb{D})$ e soddisfa $h(z_i)=w_i/z_i$ per $1 \le i \le n-1$. Per il punto (ii) della proposizione \ref{blaschke-prop} con $w=0$, abbiamo anche che $f \in \mathcal{B}_d$ se e solo se $h \in \mathcal{B}_{d-1}$. 
\end{proof}

Vediamo il caso $n=2$.

\begin{proof}
  La condizione è sempre verificata per $k=1$, mentre per $k=2$ si riscrive come
  \begin{gather*}
    \frac{1-|w_1|^2}{1-|z_1|^2}\cdot\frac{1-|w_2|^2}{1-|z_2|^2}-\frac{1-w_1\bar{w}_2}{1-z_1\bar{z}_2}\cdot\frac{1-\bar{w}_1w_2}{1-\bar{z}_1z_2} \ge 0 \\
    \frac{(1-|w_1|^2)(1-|w_2|^2)}{(1-|z_1|^2)(1-|z_2|^2)} \ge \frac{|1-w_1\bar{w}_2|^2}{|1-z_1\bar{z}_2|^2} \\
    \frac{|1-z_1\bar{z}_2|^2}{(1-|z_1|^2)(1-|z_2|^2)} \ge \frac{|1-w_1\bar{w}_2|^2}{(1-|w_1|^2)(1-|w_2|^2)} \\
    \frac{|1-z_1\bar{z}_2|^2}{1-|z_1|^2-|z_2|^2+|z_1|^2|z_2|^2} \ge \frac{|1-w_1\bar{w}_2|^2}{1-|w_1|^2-|w_2|^2+|w_1|^2|w_2|^2} \\
    \frac{|1-z_1\bar{z}_2|^2}{|1-\bar{z}_2z_1|^2-|z_1-z_2|^2} \ge \frac{|1-w_1\bar{w}_2|^2}{|1-\bar{w}_2w_1|^2-|w_1-w_2|^2} \\
    \frac{1}{1-\left|\frac{z_1-z_2}{1-\bar{z}_2z_1}\right|^2} \ge \frac{1}{1-\left|\frac{w_1-w_2}{1-\bar{w}_2w_1}\right|^2} \\
    \frac{1}{1-p^2(w_1,w_2)} \le \frac{1}{1-p^2(z_1,z_2)} \\
    p(w_1,w_2) \le p(z_1,z_2).
  \end{gather*}
  Ricordiamo adesso che $p$ è invariante per azione di $\text{Aut}(\mathbb{D})$; quindi, a meno di comporre a sinistra e a destra con opportuni automorfismi olomorfi di $\mathbb{D}$, possiamo supporre senza perdita di generalità $z_1=w_1=0$. La condizione diventa dunque $p(0,w_2) \le p(0,z_2)$, da cui $|w_2| \le |z_2|$; perciò basta prendere la funzione $f(z)=w_2z/z_2$.
\end{proof}

Andiamo adesso a dimostrare il Teorema di Pick-Nevanlinna nel caso $n=3$, con una formulazione differente.

\begin{thm}
  Siano $z_1, z_2, z_3$ e $w_1, w_2, w_3$ due triple di punti distinti in $\mathbb{D}$. Allora esiste $f \in \normalfont{\text{Hol}}(\mathbb{D},\mathbb{D}) \setminus \normalfont{\text{Aut}}(\mathbb{D})$ tale che $f(z_i)=w_i$ per $i=1,2,3$ se e solo se valgono le seguenti condizioni:
  \begin{nlist}
    \item $\omega(w_i,w_j)<\omega(z_i,z_j)$ per $i,j=1,2,3$ e $i\not=j$;
    \item $\omega\left(\dfrac{[w_2,w_1]}{[z_2,z_1]},\dfrac{[w_3,w_1]}{[z_3,z_1]}\right) \le \omega(z_2,z_3)$.
  \end{nlist}
\end{thm}

\begin{proof}
  Supponiamo che esista siffatta $f$. Allora la condizione (i) segue dal lemma di Schwarz-Pick. La condizione (ii) invece si riscrive come $\omega\bigl(f^*(z_2,z_1),f^*(z_3,z_1)\bigr) \le \omega(z_2,z_3)$, che è l'enunciato del Teorema \ref{31}.

  Adesso dimostriamo l'altra freccia. Vediamola prima nel caso $z_1=w_1=0$. Allora per la condizione (i) abbiamo $\omega(0,w_i) < \omega(0,z_i)$, quindi $|w_i/z_i|<1$ per $i=2,3$. La condizione (ii) si riscrive invece come $\omega(w_2/z_2,w_3/z_3) \le \omega(z_2,z_3)$, cioè $p(w_2/z_2,w_3/z_3) \le p(z_2,z_3)$.
  Dunque, per il caso $n=2$ del Teorema di Pick-Nevanlinna, esiste $g \in \text{Hol}(\mathbb{D},\mathbb{D})$ tale che $g(z_2)=w_2/z_2$ e $g(z_3)=w_3/z_3$. Allora basta prendere $f(z)=zg(z)$.

   Mostriamo che ci si può ridurre a questo caso. Consideriamo $h, g \in \text{Aut}(\mathbb{D})$ date da
   $$g(z)=\frac{z-z_1}{1-\bar{z}_1z} \text{ e } h(z)=\frac{z-w_1}{1-\bar{w}_1z}.$$
   Allora esiste $f$ come quella richiesta dal Teorema se e solo se esiste $F \in \text{Hol}(\mathbb{D},\mathbb{D})$, con $F=h \circ f \circ g^{-1}$, tale che $F(0)=0$, $F\bigl(g(z_2)\bigr)=h(w_2)$ e $F\bigl(g(z_3)\bigr)=h(w_3)$.
   Questo corrisponde proprio al caso precedente, quindi tale $F$ esiste se e solo se
   $$\omega\bigl(h(w_i),h(w_j)\bigr) \le \omega\bigl(g(z_i),g(z_j)\bigr)$$
   per $i,j=1,2,3$ con $i\not=j$ e
   $$\omega\left(\frac{h(w_2)}{g(z_2)},\frac{h(w_3)}{g(z_3)}\right) \le \omega\bigl(g(z_2),g(z_3)\bigr).$$
   Poiché $p$, e di conseguenza $\omega$, è invariante per azione di $\text{Aut}(\mathbb{D})$, la prima disuguaglianza è equivalente alla condizione (i). Sempre per questo motivo, sostituendo $h(z)=[z,w_1]$ e $g(z)=[z,z_1]$ otteniamo che la seconda è equivalente alla condizione (ii).
\end{proof}
