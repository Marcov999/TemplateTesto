Dedichiamo una sottosezione al seguente risultato, dimostrato indipendentemente da Pick nel 1915 \cite{P} e Nevanlinna nel 1919 \cite{N}; è un teorema di interpolazione interessante di per sé, inoltre vedremo un paio di esempi in cui le ipotesi vengono riformulate in termini del lemma di Schwarz-Pick, usando in un caso anche il rapporto iperbolico. Seguiamo la dimostrazione vista in \cite[Chapter 1, Theorem 2.2]{JBG}.

\begin{thm}
  (Pick-Nevanlinna) Siano dati $n$ punti distinti $z_1, \dots, z_n \in \mathbb{D}$ e altri $n$ punti distinti (non necessariamente diversi dai primi) $w_1, \dots, w_n \in \mathbb{D}$. Consideriamo la forma quadratica
  $$A_n(t_1,\dots,t_n)=\sum_{i,j=1}^n\frac{1-w_i\bar{w}_j}{1-z_i\bar{z}_j}t_i\bar{t}_j.$$
  Allora esiste una funzione $f \in \normalfont{\text{Hol}}(\mathbb{D},\mathbb{D})$ tale che $f(z_i)=w_i$ per $j=1, \dots, n$ se e solo se $A_n$ è semidefinita positiva. In tal caso, si può trovare $f$ che sia un prodotto di Blaschke di grado al più $n$.
\end{thm}

\begin{proof}
  Procediamo per induzione su $n$. Il caso $n=1$ è banale per transitività di $\text{Aut}(\mathbb{D})$. Supponiamo adesso $n>1$. Poniamo
  $$z_i'=\frac{z_i-z_n}{1-\bar{z}_nz_i} \,\, \text{e} \,\, w_i'=\frac{w_i-w_n}{1-\bar{w}_nw_i} \,\, \text{per} \,\, 1 \le i \le n.$$
  Allora esiste $f$ olomorfa dal disco in sé che risolve l'interpolazione se e solo se la funzione
  $$g(z)=\frac{\Bigg(f\left(\dfrac{z+z_n}{1+\bar{z}_nz}\right)-w_n\Bigg)}{1-\bar{w}_n\Bigg(f\left(\dfrac{z+z_n}{1+\bar{z}_nz}\right)\Bigg)}$$
  appartiene a $\text{Hol}(\mathbb{D},\mathbb{D})$ e soddisfa $g(z_i')=w_i'$ per $1 \le i \le n$. Infatti, si tratta solo di comporre con i giusti automorfismi. La forma quadratica $A_n'$ definita con i punti $z_i',w_i'$ è legata alla forma $A_n$. Per mostrarlo, poniamo
  $$\frac{1-z_i'\bar{z}_j'}{1-z_i\bar{z}_j}=\frac{1-|z_n|^2}{(1-\bar{z}_nz_i)(1-z_n\bar{z}_j)}=\alpha_i\bar{\alpha}_j$$
  e
  $$\frac{1-w_i'\bar{w}_j'}{1-w_i\bar{w}_j}=\frac{1-|w_n|^2}{(1-\bar{w}_nw_i)(1-w_n\bar{w}_j)}=\beta_i\bar{\beta}_j,$$
  dove $\alpha_i=\frac{\sqrt{1-|z_n|^2}}{1-\bar{z}_nz_i}$ per $1 \le i \le n$ e analogamente per i $\beta_i$. Allora si ha
  \begin{align*}
    A_n'(t_1,\dots,t_n) & =\sum_{i,j=1}^n \frac{1-w_i'\bar{w}_j'}{1-z_i'\bar{z}_j'}t_i\bar{t}_j \\
    & =\sum_{i,j=1}^n \frac{1-w_i\bar{w}_j}{1-z_i\bar{z}_j}\left(\frac{\beta_i}{\alpha_i}t_i\right)\left(\frac{\bar{\beta}_j}{\bar{\alpha}_i}\bar{t}_j\right)=A_n\left(\frac{\beta_1}{\alpha_1}t_1,\dots,\frac{\beta_n}{\alpha_n}t_n\right).
  \end{align*}
  Poiché $z_i,w_i \in \mathbb{D}$ si ha $\alpha_i,\beta_i\not=0$; perciò $A_n$ è semidefinita positiva se e solo se lo è $A_n'$. Dato che $z_n'=w_n'=0$, a meno di cambiare $f$ con $g$ possiamo supporre senza perdita di generalità $z_n=w_n=0$. La condizione dell'enunciato diventa dunque $f(0)=0$ e $f(z_i)=w_i$ per $1 \le i \le n-1$.
  Tale funzione esiste se e solo se la funzione $h(z)=f(z)/z$, con $h(0)=f'(0)$, appartiene a $\text{Hol}(\mathbb{D},\mathbb{D})$ e soddisfa $h(z_i)=w_i/z_i$ per $1 \le i \le n-1$. Per il punto (ii) della proposizione \ref{blaschke-prop} con $w=0$, abbiamo anche che $f \in \mathcal{B}_d$ se e solo se $h \in \mathcal{B}_{d-1}$.
  Vogliamo dire che $A_n$ è semidefinita positiva se e solo se la forma quadratica $A_n''$, costruita usando i punti $z_i, w_i/z_i$, è semidefinita positiva. Dato che $z_n=w_n=0$, completando il quadrato abbiamo
  \begin{align*}
    A_n(t_1,\dots,t_n)&=|t_n|^2+2\cdot\mathfrak{Re}\sum_{i=1}^{n-1}\bar{t}_it_n+\sum_{i,j=1}^{n-1}\frac{1-w_i\bar{w}_j}{1-z_i\bar{z}_j}t_i\bar{t}_j \\
    &=\left|\sum_{i=1}^n t_i\right|^2+\sum_{i,j=1}^{n-1}\left(\frac{1-w_i\bar{w}_j}{1-z_i\bar{z}_j}-1\right)t_i\bar{t}_j \\
    &=\left|\sum_{i=1}^n t_i\right|^2+\sum_{i,j=1}^{n-1}\frac{1-(w_i/z_i)\overline{(w_j/z_j)}}{1-z_i\bar{z}_j}z_i\bar{z}_jt_i\bar{t}_j,
  \end{align*}
  quindi
  $$A_n(t_1,\dots,t_n)=\left|\sum_{i=1}^n t_i\right|^2+A_n''(z_1t_1,\dots,z_{n-1}t_{n-1}).$$
  Allora se $A_n''$ è semidefinita positiva lo è anche $A_n$, mentre per l'implicazione opposta basta prendere $\displaystyle t_n=-\sum_{i=1}^n t_i$ (ricordiamo che per ipotesi $z_i\not=z_n=0$ per $1 \le i \le n-1$).
\end{proof}

Vediamo che il caso $n=2$ è equivalente alla disuguaglianza del lemma di Schwarz-Pick. Una forma quadratica associata a una matrice hermitiana è definita positiva se e solo se i determinanti delle sottomatrici principali sono positivi. Per ipotesi $\frac{1-|w_1|^2}{1-|z_1|^2} > 0$, quindi in particolare $A_2$ è semidefinita positiva se e solo se il determinante è non negativo. Abbiamo
$$\frac{1-|w_1|^2}{1-|z_1|^2}\cdot\frac{1-|w_2|^2}{1-|z_2|^2}-\left|\frac{1-w_1\bar{w}_2}{1-z_1\bar{z}_2}\right|^2 \ge 0,$$
ovvero
$$\frac{|1-z_1\bar{z}_2|^2}{(1-|z_1|^2)(1-|z_2|^2)} \ge \frac{|1-w_1\bar{w}_2|^2}{(1-|w_1|^2)(1-|w_2|^2)};$$
riarrangiando i denominatori si ottiene
$$\frac{|1-z_1\bar{z}_2|^2}{|1-\bar{z}_2z_1|^2-|z_1-z_2|^2} \ge \frac{|1-w_1\bar{w}_2|^2}{|1-\bar{w}_2w_1|^2-|w_1-w_2|^2},$$
che è equivalente a
\begin{align*}
  \frac{1}{1-\left|\frac{z_1-z_2}{1-\bar{z}_2z_1}\right|^2} & \ge \frac{1}{1-\left|\frac{w_1-w_2}{1-\bar{w}_2w_1}\right|^2} \\
  & \Leftrightarrow \frac{1}{1-p^2(w_1,w_2)} \le \frac{1}{1-p^2(z_1,z_2)} \Leftrightarrow \omega(w_1,w_2) \le \omega(z_1,z_2).
\end{align*}
Per invarianza di $\omega$ sotto l'azione di $\text{Aut}(\mathbb{D})$, si può porre $w_1=z_1=0$ e la funzione di interpolazione si trova immediatamente.

Andiamo adesso a ridimostrare il caso $n=3$ con una formulazione differente; otteniamo così una sorta di inverso del Teorema \ref{31}.

\begin{thm}
  Siano $z_1, z_2, z_3$ e $w_1, w_2, w_3$ due triple di punti distinti in $\mathbb{D}$. Allora esiste $f \in \normalfont{\text{Hol}}(\mathbb{D},\mathbb{D}) \setminus \normalfont{\text{Aut}}(\mathbb{D})$ tale che $f(z_i)=w_i$ per $i=1,2,3$ se e solo se valgono le seguenti condizioni:
  \begin{nlist}
    \item $\omega(w_i,w_j)<\omega(z_i,z_j)$ per $i,j=1,2,3$ e $i\not=j$;
    \item $\omega\left(\dfrac{[w_2,w_1]}{[z_2,z_1]},\dfrac{[w_3,w_1]}{[z_3,z_1]}\right) \le \omega(z_2,z_3)$.
  \end{nlist}
\end{thm}

\begin{proof}
  Supponiamo che esista siffatta $f$. Allora la condizione (i) segue dal lemma di Schwarz-Pick. La condizione (ii) invece si riscrive come $\omega\bigl(f^*(z_2,z_1),f^*(z_3,z_1)\bigr) \le \omega(z_2,z_3)$, che è l'enunciato del Teorema \ref{31}.

  Adesso dimostriamo l'altra freccia. Vediamola prima nel caso $z_1=w_1=0$. Allora per la condizione (i) abbiamo $\omega(0,w_i) < \omega(0,z_i)$, quindi $|w_i/z_i|<1$ per $i=2,3$. La condizione (ii) si riscrive invece come $\omega(w_2/z_2,w_3/z_3) \le \omega(z_2,z_3)$, cioè $p(w_2/z_2,w_3/z_3) \le p(z_2,z_3)$.
  Dunque, per il caso $n=2$ del teorema di Pick-Nevanlinna, esiste $g \in \text{Hol}(\mathbb{D},\mathbb{D})$ tale che $g(z_2)=w_2/z_2$ e $g(z_3)=w_3/z_3$. Allora basta prendere $f(z)=zg(z)$.

   Mostriamo che ci si può ridurre a questo caso. Consideriamo $h, g \in \text{Aut}(\mathbb{D})$ date da
   $$g(z)=\frac{z-z_1}{1-\bar{z}_1z} \,\, \text{e} \,\, h(z)=\frac{z-w_1}{1-\bar{w}_1z}.$$
   Allora esiste $f$ come quella richiesta dal Teorema se e solo se esiste $F \in \text{Hol}(\mathbb{D},\mathbb{D})$, con $F=h \circ f \circ g^{-1}$, tale che $F(0)=0$, $F\bigl(g(z_2)\bigr)=h(w_2)$ e $F\bigl(g(z_3)\bigr)=h(w_3)$.
   Questo corrisponde proprio al caso precedente, quindi tale $F$ esiste se e solo se
   $$\omega\bigl(h(w_i),h(w_j)\bigr) \le \omega\bigl(g(z_i),g(z_j)\bigr)$$
   per $i,j=1,2,3$ con $i\not=j$ e
   $$\omega\left(\frac{h(w_2)}{g(z_2)},\frac{h(w_3)}{g(z_3)}\right) \le \omega\bigl(g(z_2),g(z_3)\bigr).$$
   Poiché $\omega$ è invariante per azione di $\text{Aut}(\mathbb{D})$, la prima disuguaglianza è equivalente alla condizione (i). Sempre per questo motivo, sostituendo $h(z)=[z,w_1]$ e $g(z)=[z,z_1]$ otteniamo che la seconda è equivalente alla condizione (ii).
\end{proof}

L'argomento viene trattato più in dettaglio in \cite{BRW}, dove il caso $n$ generico viene studiato usando i rapporti iperbolici iterati. Nell'articolo i prodotti di Blaschke, e il loro legame con i rapporti iperbolici, vengono comparati ai polinomi e il loro legame con i rapporti incrementali nel caso euclideo. Anche l'algoritmo di interpolazione è, formalmente, l'analogo di quello di Newton per i polinomi.
