\begin{thebibliography}{widest entry}
  \bibitem[B]{B} L. Bieberbach: Über die Koeffizienten derjenigen Potenzreihen, welche eine schlichte Abbildung des Einheitskreises vermitteln. \textit{Sitzungsberichte der Königlich Preussischen Akademie der Wissenschaften zu Berlin}, \textbf{1} (1916), no. 3, 940--955
%  \bibitem[BKR]{BKR} F. Bracci, D. Kraus, O. Roth: A new Schwarz-Pick Lemma at the boundary and rigidity of holomorphic maps. Preprint, ArXiv:2003.02019v1 (2020)
%  \bibitem[BM]{BM} A. F. Beardon, D. Minda: A multi-point Schwarz-Pick lemma. \textit{Journal d'Analyse Mathématique}, \textbf{92} (2004), 81--104
%  \bibitem[BRW]{BRW} L. Baribeau, P. Rivard, E. Wegert: On Hyperbolic Divided Differences and the Nevanlinna-Pick Problem. \textit{ Computational Methods and Function Theory}, \textbf{9} (2009), no. 2, 391--405
%  \bibitem[D]{D} J. Dieudonné: Recherches sur quelques problèmes relatifs aux polynômes et aux fonctions bornées d'une variable complexe. \textit{Annales Scientifiques de l'École Normale Supérieure}, \textbf{48} (1931), 247--358
%  \bibitem[GMG]{GMG} G. M. Golusin: Some estimations of derivatives of bounded functions. \textit{Recueil Mathématique [Matematicheskiĭ Sbornik]}, \textbf{16(58)} (1945), no. 3, 295--306
  \bibitem[D]{D} P. L. Duren: \textbf{Univalent Functions}. Springer-Verlag, New York, 1983
  \bibitem[dB]{dB} L. de Branges: A proof of the Bieberbach conjecture. \textit{Acta Mathematica}, \textbf{154} (1985) no. 1-2, 137--152
  \bibitem[Gol]{Gol} G. M. Goluzin: \textbf{Geometric theory of functions of a complex variable}. American Mathematical Society, Providence, 1969
  \bibitem[Gr]{Gr} T. H. Gronwall: Some remarks on conformal representation. \textit{Annals of Mathematics. Second Series}, \textbf{16} (1914), no. 1-4, 72--76
\end{thebibliography}
