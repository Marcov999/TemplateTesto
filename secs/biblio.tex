\begin{thebibliography}{widest entry}
  \bibitem[B]{B} L. Bieberbach: Über die Koeffizienten derjenigen Potenzreihen, welche eine schlichte Abbildung des Einheitskreises vermitteln. \textit{Sitzungsberichte der Königlich Preussischen Akademie der Wissenschaften zu Berlin}, \textbf{1} (1916), no. 3, 940--955
  \bibitem[D]{D} P. L. Duren: \textbf{Univalent Functions}. Springer-Verlag, New York, 1983
  \bibitem[dB]{dB} L. de Branges: A proof of the Bieberbach conjecture. \textit{Acta Mathematica}, \textbf{154} (1985) no. 1-2, 137--152
  \bibitem[Gol1]{Gol1} G. M. Goluzin: \textbf{Geometric theory of functions of a complex variable}. American Mathematical Society, Providence, 1969
  \bibitem[Gol2]{Gol2} G. M. Goluzin: Method of variations in the theory of conform representation. \textit{Recueil Mathématique [Matematicheskiĭ Sbornik]}, \textbf{21(63)} (1947), 83--117
  \bibitem[Gr]{Gr} T. H. Gronwall: Some remarks on conformal representation. \textit{Annals of Mathematics. Second Series}, \textbf{16} (1914), no. 1-4, 72--76
  \bibitem[L]{L} K. Löwner: Über Extremumsätze bei der konformen Abbildung des Äußeren des Einheitskreises. \textit{Mathematische Zeitschrift}, \textbf{3} (1919), no. 1, 65--77
\end{thebibliography}
