\begin{defn}
  Sia $\gamma:[0, 1] \longrightarrow \mathbb{C}$ un cammino continuo.
  Se esistono $0=t_0<t_1<\dots<t_r=1$, intorni $U_0, \dots, U_j, \dots, U_r$ di $\gamma(t_j)$ e $f_j:U_j \longrightarrow \mathbb{C}$ olomorfe t.c. $f_j\restrict{U_j \cap U_{j+1}} \equiv f_{j+1}\restrict{U_j \cap U_{j+1}}$ diremo che \textsc{$f_0$ si prolunga olomorficamente lungo $\gamma$}.
\end{defn}

\begin{ex}
  $\gamma(t)=e^{2\pi i t}, \gamma(0)=\gamma(1)=1$. $z=|z|e^{2\pi i \theta}, \theta \in \mathbb{R}$. $U_0=D(1, 1/2), f_0:U_0: \longrightarrow \mathbb{C}, f_0(z)=z^{1/2}=|z|^{1/2}e^{2\pi i(\theta/2)}$ ($\theta \in (-\pi, \pi)$).
  $f_0 \in \mathcal{O}(U_0)$. È possibile prolungare olomorficamente $f_0$ lungo $\gamma$ con $f(\gamma(t))=e^{2\pi i(t/2)}$ $\implies$ $f(\gamma(1))=e^{2\pi i/2}=e^{\pi i}=-1$. $f(\gamma(0))=1$.
\end{ex}

\begin{defn}
  Sia $a \in \mathbb{C}$ e consideriamo le coppie $(U, f)$ dove $U \subseteq \mathbb{C}$ è un intorno aperto di $a$ e $f \in \mathcal{O}(U)$. Definiamo la seguente relazione di equivalenza: $(U, f) \sim (V, g)$ se esiste $W \subseteq U \cap V$ intorno aperto di $a$ t.c. $f\restrict{W}=g\restrict{W}$. \\
  $\mathcal{O}_a:=\faktor{\{(U, f)\}}{\sim}$ è detta \textsc{spiga dei germi di funzioni olomorfe in $a$}. \\
  $\underline{f_a} \in \mathcal{O}_a$ si dice \textsc{germe} di funzione olomorfa. \\
  $(U,f) \in \underline{f_a}$ si dice \textsc{rappresentante} di $\underline{f_a}$. \\
  $\displaystyle \mathcal{O}:=\bigcup_{a \in \mathbb{C}} \mathcal{O}_a$ si dice \textsc{fascio dei germi} di funzioni olomorfe. \\
  Dato $\Omega \subseteq \mathbb{C}$ aperto, definiamo anche $\displaystyle \mathcal{O}_{\Omega}:=\bigcup_{a \in \Omega} \mathcal{O}_a$.
\end{defn}

\begin{exc}
  $\sim$ appena definita è una relazione di equivalenza.
\end{exc}

\begin{exc}
  $\mathcal{O}_a$ è una $\mathbb{C}$-algebra ($\underline{f_a}+\underline{g}_a$ è il germe rappresentato da $(U \cap V, (f+g)\restrict{U \cap V})$ dove $(U,f) \in \underline{f_a}$ e $(V, g) \in \underline{g}_a$).
\end{exc}

\begin{oss}
  Possiamo definire per ogni $k \ge 0$ $\underline{f_a}^{(k)}(a) \in \mathbb{C}$ ponendo $\underline{f_a}^{(k)}(a)=f^{(k)}(a)$ con $(U, f) \in \underline{f_a}$.
\end{oss}

\begin{defn}
  Definiamo $p$ come la \textit{proiezione}
  \begin{align*}
    p: \mathcal{O} &\longrightarrow \mathbb{C}\\
    \underline{f_z} &\longmapsto z
  \end{align*}
  Vale che $p(\mathcal{O}_a)=\{a\}$. Vogliamo rendere $p$ "quasi" un rivestimento (vedremo che, per i soliti esempi stupidi, non può essere un rivestimento).
\end{defn}

Vogliamo definire una topologia su $\mathcal{O}$. Definiamo un sistema fondamentale di intorni.

\begin{defn}
  Gli intorni del sistema fondamentale sono i seguenti: dati $U \subseteq \mathbb{C}$ aperto, $f \in \mathcal{O}(U)$ l'intorno associato è $N(U, f)=\{\underline{f_z} \mid z \in U, (U, f) \in \underline{f_z}\}$.
\end{defn}

\begin{exc}
  Esiste un'unica topologia su $\mathcal{O}$ t.c. $\{N(U, f)\}$ siano un sistema fondamentale di intorni.
\end{exc}

\begin{oss}
  $p\restrict{N(U, f)}:N(U, f) \longrightarrow U$ è una bigezione.
\end{oss}

\begin{prop}
  $\mathcal{O}$ è uno spazio di Hausdorff.
\end{prop}

\begin{proof}
  Siano $\underline{f_a} \not \underline{g_b}$. Se $a \not= b$, esistono $(U, f) \in \underline{f_a}, (V, g) \in \underline{g_b}$ con $U \cap V=\emptyset$ $\implies$ $N(U, f) \cap N(V, g)=\emptyset$.
  Se $a=b$, siano $(U, f) \in \underline{f_a}, (V, g) \in \underline{g_a}$, $D \subset U \cap V$ disco aperto di centro $a$. Vogliamo $N(D, f) \cap N(D, g)=\emptyset$.
  Per assurdo, sia $\underline{h}_z \in N(D, f) \cap N(D, g)$ $\implies$ $z \in D$ e $\underline{h_z}=\underline{f_z}$ e $\underline{h_z}=\underline{g_z}$ $\implies$ $\underline{f_z}=\underline{g_z}$ $\implies$ esiste un aperto $W \subseteq D$ intorno di $z$ t.c. $f\restrict{W}=g\restrict{W}$ e per il principio di identità si avrebbe $f \equiv g$ su $D$ $\implies$ $\underline{f_a}=\underline{g_a}$, assurdo.
\end{proof}

\begin{prop}
  $p: \mathcal{O} \longrightarrow \mathbb{C}$ è continua, aperta e omeomorfismo locale.
\end{prop}

\begin{proof}
  Sia $V \subseteq \mathbb{C}$, $\displaystyle p^{-1}(V)=\bigcup\{N(W, f) \mid W \subseteq V \text{ aperto}, f \in \mathcal{O}(W)\}$ è aperto. $p(N(U, f))=U$ $\implies$ $p$ è aperta.
  $p\restrict{N(U, f)}$ è invertibile: $p^{-1}(z)=\underline{f_z}$ $\implies$ $p\restrict{N(U, f)}$ è un omeomorfismo $\implies$ $p$ è un omeomorfismo locale.
\end{proof}

\begin{defn}
  Una \textit{sezione} di $\mathcal{O}$ su un $\Omega \subset \mathbb{C}$ aperto è una $\underline{f}:\Omega \longrightarrow \mathcal{O}$ continua t.c. $p \circ \underline{f}=\id_{\Omega}$, cioè $\underline{f}(z) \in \mathcal{O}_z$ per ogni $z \in \Omega$.
\end{defn}

\begin{exc}
  L'insieme delle sezioni di $\mathcal{O}$ su $\Omega$ è in corrispondenza biunivoca con lo spazio $\mathcal{O}(\Omega)$ delle funzioni olomorfe su $\Omega$.
\end{exc}

\begin{defn}
  Siano $a \in \mathbb{C}, \underline{f_a} \in \mathcal{O}_a$. Sia $\gamma:[0, 1] \longrightarrow \mathbb{C}$ una curva continua con $\gamma(0)=a$.
  Un \textsc{prolungamento analitico di $\underline{f_a}$ lungo $\gamma$} è un sollevamento $\tilde{\gamma}:[0, 1] \longrightarrow \mathcal{O}$ di $\gamma$ (cioè $p \circ \tilde{\gamma}=\gamma$) t.c. $\tilde{\gamma}(0)=\underline{f_a}$.
\end{defn}

\begin{oss}
  $p$ non è un rivestimento perché non tutte le curve possono essere sollevate. Vediamo un esempio.
\end{oss}

\begin{ex}
  $a=1, \underline{f_a}=(\mathbb{C}^*, 1/z), \gamma(t)=1-t$. Non esiste alcun sollevamento di $\gamma$ che parte da $\underline{f_a}$.
\end{ex}

\begin{defn}
  Sia $\diff:\mathcal{O} \longrightarrow \mathcal{O}$ così definita: dato $\underline{f_a} \in \mathcal{O}_a$, $\diff\underline{f_a}$ è il germe in $a$ rappresentato dalla derivata di un rappresentante di $\underline{f_a}$, cioè se $(U, f) \in \underline{f_a}$, $\diff\underline{f_a}$ è rappresentato da $(U, f')$.
\end{defn}

\begin{lm} \label{primitiva}
  Sia $D \subseteq \mathbb{C}$ un disco aperto. Allora ogni $f \in \mathcal{O}(D)$ ha una primitiva in $D$, e due primitive differiscono per una costante additiva.
\end{lm}

\begin{proof}
  Se $a \in D$ è il centro, $\displaystyle f(z)=\sum_{n=0}^{+\infty} c_n(z-a)^n$. Una primitiva è data da $\displaystyle F(z)=\sum_{n=0}^{+\infty} \frac{c_n}{n+1}(z-a)^{n+1}$. È chiaro che due primitive differiscono per una costante additiva.
\end{proof}

\begin{prop}
  $\diff:\mathcal{O} \longrightarrow \mathcal{O}$ è un rivestimento.
\end{prop}

\begin{proof}
  Dati $\underline{f_a} \in \mathcal{O}_a$, $(U, f) \in \underline{f_a}$, $D \subseteq U$ un disco centrato in $a$, poniamo $\mathcal{D}=N(D, f)$, intorno aperto di $\underline{f_a}$. Sia $F$ una primitiva di $f$ su $D$ che esiste per il lemma \ref{primitiva}, per ogni $c \in \mathbb{C}$ poniamo $\mathcal{D}_c=N(D, F+c)$. Vogliamo dimostrare che:
  \begin{nlist}
    \item $\displaystyle \diff^{-1}(\mathcal{D})=\bigcup_{c \in \mathbb{C}} \mathcal{D}_c$;
    \item $\diff\restrict{\mathcal{D}_c}:\mathcal{D}_c \longrightarrow \mathcal{D}$ è un omeomorfismo;
    \item $c_1\not=c_2 \implies \mathcal{D}_{c_1} \cap \mathbb{D}_{c_2} \emptyset$.
  \end{nlist}
  (i), (ii), (iii) $\implies$ $\diff$ è un rivestimento. Procediamo con la dimostrazione.
  \begin{nlist}
    \item Sia $z \in D$ e $\underline{f_z} \in \mathcal{D}$. Sia $\underline{g_z} \in \mathcal{O}_z$ t.c. $\diff\underline{g_z}=\underline{f_z}$ $\implies$ esiste $(W, g) \in \underline{g_z}$ t.c. $g'=f$;
    possiamo supporre che $W \subseteq D$, il disco, quindi sempre per il lemma \ref{primitiva} esiste $c \in \mathbb{C}$ t.c. $g\restrict{W}=F\restrict{W}+c$ $\implies$ $\underline{g_z} \in \mathcal{D}_c$. È banale vedere che $\underline{g_z} \in \mathcal{D}_c \implies \diff\underline{g_z} \in \mathcal{D}$.
    \item È ovvio che $\diff(\mathcal{D}_c)=\mathcal{D}$ (per definizione di $\diff$ e $\mathcal{D}_c$). Questo più il punto (i) ci danno che $\diff$ è continua e aperta: infatti,\ gli insiemi della forma $\mathcal{D}$ formano un sistema fondamentale di intorni e la loro preimmagine, unione di aperti, è aperta; anche gli insiemi $\mathcal{D}_c$ sono un sistema fondamentale di intorni (ogni funzione olomorfa è la primitiva della sua derivata) e la loro immagine, come abbiamo visto, è un aperto.
    $\diff\restrict{\mathcal{D}_c}: \mathcal{D}_c \longrightarrow \mathcal{D}$ è, come visto sopra, suriettiva, ma anche iniettiva perché $\displaystyle \mathcal{D}_c=\bigcup_{z \in D} \mathcal{D}_c \cap \mathcal{O}_z, \mathcal{D}=\bigcup_{z \in D} \mathcal{D} \cap \mathcal{O}_z$,
     ma per ogni $z \in D$, $\mathcal{D}_c\cap \mathcal{O}_z$ e $\mathcal{D}\cap \mathcal{O}_z$ contengono un unico germe e $\diff(\mathcal{O}_z) \subseteq \mathcal{O}_z$, da cui appunto segue l'iniettività ($z\not=z' \implies \mathcal{O}_z\cap\mathcal{O}_{z'}=\emptyset$).
     \item Se $\underline{F_z} \in \mathcal{D}_{c_1}\cap \mathcal{D}_{c_2}$ $\implies$ $\underline{F_z}$ è rappresentanto sia da $(D, F+c_1)$ che da $(D, F+c_2)$ $\implies$ $F+c_1\equiv F+c_2$ vicino a $z$ $\implies$ $c_1=c_2$.
  \end{nlist}
\end{proof}

\begin{thm}
  Sia $\Omega \subseteq \mathbb{C}$ un aperto semplicemente connesso. Allora ogni $f \in \mathcal{O}(\Omega)$ ammette una primitiva.
\end{thm}

\begin{proof}
  Sia $\varphi:\Omega \longrightarrow \mathcal{O}$ la sezione corrispondente a $f$. Sia $\Phi$ un sollevamento di $\varphi$, cioè $d \circ \Phi=\varphi$ (che esiste per la teoria generali dei rivestimenti).
  \begin{center}
    \begin{tikzcd}
      & \mathcal{O} \arrow[d, "\diff"]\\
      \Omega \arrow[ru, "\Phi"] \arrow[r, "\varphi"] & \mathcal{O}
    \end{tikzcd}
  \end{center}
  Anche $\Phi$ è una sezione di $\mathcal{O}$: infatti, siccome $p \circ \diff=p$, $p \circ \Phi=p\circ\diff\circ\Phi=p\circ\varphi=\id_{\Omega}$ $\implies$ la $F \in \mathcal{O}(\Omega)$ associata a $\Phi$ è una primitiva di $f$.
\end{proof}

Concludiamo il paragrafo definendo logaritmo e radice $n$-esima su insiemi semplicemente connessi.

\begin{cor}
  Sia $\Omega \subseteq \mathbb{C}$ aperto semplicemente connesso, $f \in \text{Hol}(\Omega, \mathbb{C}^*)$. Allora esiste $g \in \mathcal{O}(\Omega)$ t.c. $f=\exp(g)$. Inoltre $g$ è unica a meno di costanti additive della forma $2k\pi i$ con $k \in \mathbb{Z}$.
\end{cor}

\begin{proof}
  Sia $g_0$ una primitiva di $f'/f$. $\dfrac{\diff}{\diff z}(fe^{-g_0})=f'e^{-g_0}+f(-e^{-g_0}g_0')=f'e^{-g_0}+f(-e^{-g_0}f'/f)=e^{-g_0}(f'-f')=0$ $\implies$ $f \cdot e^{-g_0}=\text{costante diversa da zero}=e^{c_0}$ $\implies$ $f\equiv e^{c_0+g_0}$.
  Per l'unicità a meno di costanti additive, $\exp(g_1)=\exp(g_2) \implies \exp(g_1-g_2)\equiv 1 \implies$ $g_1-g_2$ è continua a valori in $2\pi i \mathbb{Z}$ discreto $\implies$ $g_1-g_2=2k\pi i$ con $k \in \mathbb{Z}$ costante.
\end{proof}

\begin{cor}
  Sia $\Omega \subseteq \mathbb{C}$ aperto semplicemente connesso, $f \in \text{Hol}(\Omega, \mathbb{C}^*)$, $n \in \mathbb{Z}^*$. Allora esiste $h \in \mathcal{O}(\Omega)$ t.c. $f=h^n$. Inoltre $h$ è unica a meno di costanti moltiplicative della forma $e^{2\pi ik/n}$ con $k \in \mathbb{Z}$.
\end{cor}

\begin{proof}
  Sia $g \in \mathcal{O}(\Omega)$ t.c. $f=\exp(g)$, allora $h=\exp(g/n)$ soddisfa le condizioni richieste. Poi, $h_1^n=h_2^n \iff (h_1/h_2)^n\equiv 1 \implies h_1=e^{2\pi i k/n}h_2$ con $k \in \mathbb{Z}$ costante.
\end{proof}
