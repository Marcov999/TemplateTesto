Come già anticipato, per la maggior parte dei risultati in questa sezione passeremo dalle funzioni in $\Sigma$; per fissare la notazione, data $f \in \mathcal{S}$, indichiamo con $F$ la funzione in $\Sigma$ data da $F(z)=1/f(1/z)+a_2$. Vediamo subito un risultato parziale che sarà un po' il fulcro di tutti i nostri tentativi di dimostrare qualcosa passando per $\Sigma$.

\begin{prop} \label{passaggio}
  Sia $f \in \mathcal{S}$ tale che per la $F \in \Sigma$ corrispondente e per ogni $\zeta,\eta \in \mathbb{C}\setminus\overline{\mathbb{D}}$ si ha
  $$\left|\frac{F(\zeta)-F(\eta)}{\zeta-\eta}\right| \ge 1-\frac{1}{|\zeta\eta|};$$
  allora vale la minorazione per $f$.
\end{prop}

\begin{proof}
  Dati $z,w \in \mathbb{D}$, poniamo $\zeta=1/z, \eta=1/w$. Ricordando la definizione di $F$ abbiamo dunque che
  \begin{gather*}
    \left|\frac{1/f(z)-1/f(w)}{1/z-1/w}\right| \ge 1-|zw| \\
    \left|\frac{f(z)-f(w)}{z-w}\right| \ge \left|\frac{f(w)f(z)}{zw}\right| (1-|zw|),
  \end{gather*}
  e a questo punto la tesi segue applicando la minorazione in \eqref{growth}.
\end{proof}

Quello che otterremo in questa sezione saranno delle stime dal basso per $\left|\dfrac{F(\zeta)-F(\eta)}{\zeta-\eta}\right|$ che, ripetendo il ragionamento appena fatto, ci daranno dei risultati parziali per la minorazione in \eqref{congettura}.
In generale, se $N: \mathbb{D}\times \mathbb{D} \longrightarrow [0,+\infty)$ è tale che $\left|\dfrac{F(\zeta)-F(\eta)}{\zeta-\eta}\right| \ge N(1/\zeta,1/\eta)$, allora vale la minorazione con $N(z,w)$ al posto di $1-|zw|$. Vedremo che, sotto alcune ipotesi aggiuntive, riusciremo a dimostrare la disuguaglianza voluta. \\

Definiamo adesso dei coefficienti che chiameremo impropriamente coefficienti di Grunsky: data $F \in \Sigma$, scriviamo
$$\log{\frac{F(z)-F(w)}{z-w}}=-\sum_{n,k=1}^{+\infty} \gamma_{nk}z^{-n}w^{-k}$$
(i veri coefficienti di Grunsky sarebbero $\beta_{nk}=n\gamma_{nk}$).

Enunciamo delle disuguaglianze molto utili sui coefficienti di Grunsky.
\begin{thm}
  (disuguaglianze di Grunsky, \cite[Chapter 4.3, Inequalities (10) and (12)]{D}) Per ogni scelta di $\lambda_n, \mu_n \in \mathbb{C}$ vale
  \begin{equation}\label{grunskystrong}
    \sum_{k=1}^{+\infty} k\left|\sum_{n=1}^N\gamma_{nk}\lambda_n\right|^2 \le \sum_{n=1}^N \frac{|\lambda_n|^2}{n}
  \end{equation}
  e
  \begin{equation}\label{grunskyweak}
    \left|\sum_{n,k=1}^N \gamma_{nk}\lambda_n\mu_k\right|^2 \le \sum_{n=1}^N \frac{|\lambda_n|^2}{n}\sum_{n=1}^N \frac{|\mu_n|^2}{n}.
  \end{equation}
\end{thm}
Vediamo come una diretta applicazione di \eqref{grunskyweak} con $\lambda_n=z^{-n},\mu_k=w^{-k}$ ci permette già di dire qualcosa (mandando $N \longrightarrow +\infty$):
\begin{align*}
  \left|\log{\frac{F(z)-F(w)}{z-w}}\right|^2 &=\left|-\sum_{n,k=1}^{+\infty} \gamma_{nk}z^{-n}w^{-k}\right|^2 \le \left(\sum_{n,k=1}^{+\infty} |\gamma_{nk}||z|^{-n}|w|^{-k}\right)^2 \\
  &\le \sum_{n=1}^{+\infty} \frac{1}{n|z|^{2n}}\sum_{n=1}^{+\infty} \frac{1}{n|w|^{2n}}=\log\left(1-\frac{1}{|z|^2}\right)\log\left(1-\frac{1}{|w|^2}\right).
\end{align*}
Si ha allora
$$-\log{\left|\frac{F(z)-F(w)}{z-w}\right|} \le \left|\log{\frac{F(z)-F(w)}{z-w}}\right| \le \sqrt{\log\left(1-\frac{1}{|z|^2}\right)\log\left(1-\frac{1}{|w|^2}\right)},$$
da cui
$$\left|\frac{F(z)-F(w)}{z-w}\right| \ge \exp\left(-\sqrt{\log\left(1-\frac{1}{|z|^2}\right)\log\left(1-\frac{1}{|w|^2}\right)}\right).$$
Questo ci dà un'espressione per $N(z,w)$ non proprio bella da vedere, ma che per adesso è il miglior risultato generale ottenuto:
\begin{equation} \label{best}
  N(z,w)=\exp\left(-\sqrt{\log\left(1-|z|^2\right)\log\left(1-|w|^2\right)}\right).
\end{equation}
Volendo la si può sostituire con un'espressione più semplice da maneggiare: $\sqrt{(1-|z|^2)(1-|w|^2)}$; tuttavia, è una stima peggiore e il caso $w=0$ non interpola bene. Il problema principale di queste due espressioni per $N(z,w)$ è che, fissato $w$, tendono a $0$ per $|z| \longrightarrow 1$, cosa che non accade con $1-|zw|$ e che comunque non vale per il rapporto incrementale, quindi in quel caso non sarebbe una buona stima. Entrambe ci permettono comunque di dedurre che la minorazione in \eqref{congettura} è vera per $|z|=|w|$.

Una generalizzazione in più variabili per l'espressione \eqref{best}, o meglio per la stima del modulo del logaritmo, può essere trovata in \cite[Chapter IV.2, Theorem 3]{Gol}. \\

Un altro caso particolare è quello in cui $\arg{z}=\arg{w}$. Infatti, nell'articolo in cui Golusin applica metodi variazionali per ottenere questo tipo di stime ottiene il risultato (inserire citazione); questo ci permette di dire che la minorazione congetturata è vera in questo caso.

Continuando ad indagare questi risultati parziali, possiamo dimostrare la minorazione anche per $F \in \Sigma$ con coefficienti non negativi. Questo grazie a un risultato dovuto a Gronwall, direttamente dall'articolo in cui dimostra il suo famoso teorema dell'area.
\begin{prop}
  (inserire citazione) Sia $F \in \Sigma$ e per $|z|>1$ scriviamo $F(z)=z+\displaystyle \sum_{n=1}^{+\infty} a_nz^{-n}$. Se $a_n \ge 0$ per ogni $n$, allora $\displaystyle \sum_{n=1}^{+\infty} na_n \le 1$.
\end{prop}
Vediamo come questo ci dà la minorazione:
\begin{align*}
  \left|\frac{F(z)-F(w)}{z-w}\right|&=\left|\frac{z-w+\sum_{n=1}^{+\infty}a_n(z^{-n}-w^{-n})}{z-w}\right|=\\
  &=\left|1-\sum_{n=1}^{+\infty} \frac{a_n}{z^nw^n}\sum_{h+j=n-1}z^hw^j\right| \ge \\
  &\ge 1-\sum_{n=1}^{+\infty} \frac{a_n}{|zw|}\sum_{h+j=n-1}|z|^{h-n+1}|w|^{j-n+1} \ge \\
  & \ge 1-\frac{1}{|zw|}\sum_{n=1}^{+\infty} a_nn \ge 1-\frac{1}{|zw|}.
\end{align*}

Come ultimo risultato parziale, vediamo una dimostrazione per le funzioni in $\mathcal{S}$ convesse (cioè con immagine convessa), che è indipendente dalla stima per le funzioni in $\Sigma$.
\begin{prop}
  Sia $f \in \mathcal{S}$ tale che $\Ima{f}$ è convessa. Allora vale la minorazione in \eqref{congettura}.
\end{prop}
\begin{proof}
  Supponiamo $|w|<|z|$ e sia $\Gamma(t)=f(w)+t\big(f(z)-f(w)\big) \subset \Ima{f}$ il segmente che collega le immagini dei due punti, per ipotesi tutto contenuto nell'immagine. Cambiando di variabile e applicando la minorazione in \eqref{distortion} troviamo
  \begin{align*}
    |f(z)-f(w)|&=\int_{\Gamma} |\diff\eta|=\int_{f^{-1}(\Gamma)} |f'(\zeta)||\diff\zeta| \ge \\
    &\ge \int_{f^{-1}(\Gamma)} \frac{1-|\zeta|}{(1+|\zeta|)^3}|\diff\zeta| \ge \\
    &\ge \int_\gamma \frac{1-|\zeta|}{(1+|\zeta|)^3}|\diff\zeta|,
  \end{align*}
  dove $\gamma(t)=w+t(z-w)$ (è di questo passaggio che non sono tanto sicuro, perché non ho ben capito come fa Lowner a passare dall'uguaglianza alla disuguaglianza e se mi sono sbagliato potrebbe essere che quello che fa lui non va bene per il cambio di variabile che ci serve). Osserviamo che $(1-x)/(1+x)^3$ è decrescente, quindi possiamo concludere in maniera analoga a quanto già fatto per la maggiorazione per ottenere la tesi.
\end{proof}
