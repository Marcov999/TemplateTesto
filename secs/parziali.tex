Come già anticipato, passeremo da $\Sigma$; per fissare la notazione, data $f \in \mathcal{S}$, indichiamo con $F$ la funzione in $\Sigma$ data da $F(z)=1/f(1/z)+a_2$. Vediamo subito un risultato parziale che sarà un po' il fulcro di tutti i nostri tentativi di dimostrare qualcosa passando per $\Sigma$.

\begin{prop}
  Sia $f \in \mathcal{S}$ tale che per la $F \in \Sigma$ corrispondente e per ogni $\zeta,\eta \in \mathbb{C}\setminus\overline{\mathbb{D}}$ si ha
  $$\left|\frac{F(\zeta)-F(\eta)}{\zeta-\eta}\right| \ge 1-\frac{1}{|\zeta\eta|};$$
  allora vale la minorazione per $f$.
\end{prop}

\begin{proof}
  Dati $z,w \in \mathbb{D}$, poniamo $\zeta=1/z, \eta=1/w$. Ricordando la definizione di $F$ abbiamo dunque che
  \begin{gather*}
    \left|\frac{1/f(z)-1/f(w)}{1/z-1/w}\right| \ge 1-|zw| \\
    \left|\frac{f(z)-f(w)}{z-w}\right| \ge \left|\frac{f(w)f(z)}{zw}\right| (1-|zw|),
  \end{gather*}
  e a questo punto la tesi segue applicando la minorazione in \eqref{growth}.
\end{proof}

Quello che otterremo in questa sezione saranno delle stime dal basso per $\left|\dfrac{F(\zeta)-F(\eta)}{\zeta-\eta}\right|$ che, ripetendo il ragionamento appena fatto, ci daranno dei risultati parziali per la minorazione in \eqref{congettura}. Vedremo che, sotto alcune ipotesi aggiuntive, riusciremo a dimostrare la disuguaglianza voluta. \\

Partiamo definendo dei coefficienti che chiameremo impropriamente coefficienti di Grunsky: data $F \in \Sigma$, scriviamo
$$\log{\frac{F(z)-F(w)}{z-w}}=-\sum_{n,k=1}^{+\infty} \gamma_{nk}z^{-n}w^{-k}$$
(i veri coefficienti di Grunsky sarebbero $\beta_{nk}=n\gamma_{nk}$).

Per cominciare, enunciamo delle disuguaglianze molto utili sui coefficienti di Grunsky.
\begin{thm}
  (Grunsky, inserire citazione) Per ogni scelta di $\lambda_n, \mu_n \in \mathbb{C}$ vale
  \begin{equation}\label{grunskystrong}
    \sum_{k=1}^{+\infty} k\left|\sum_{n=1}^N\gamma_{nk}\lambda_n\right|^2 \le \sum_{n=1}^N \frac{|\lambda_n|^2}{n}
  \end{equation}
  e
  \begin{equation}\label{grunskyweak}
    \left|\sum_{n,k=1}^N \gamma_{nk}\lambda_n\mu_k\right|^2 \le \sum_{n=1}^N \frac{|\lambda_n|^2}{n}\sum_{n=1}^N \frac{|\mu_n|^2}{n}.
  \end{equation}
\end{thm}
Vediamo come una diretta applicazione di \eqref{grunskyweak} con $\lambda_n=z^{-n},\mu_k=w^{-k}$ ci permette già di dire qualcosa (mandando $N \longrightarrow +\infty$):
\begin{align*}
  \left|\log{\frac{F(z)-F(w)}{z-w}}\right|^2 &\le \left(\sum_{n,k=1}^{+\infty} |\gamma_{nk}||z|^{-n}|w|^{-k}\right)^2 \le \\
  &\le \sum_{n=1}^{+\infty} \frac{1}{n|z|^{2n}}\sum_{n=1}^{+\infty} \frac{1}{n|w|^{2n}}=\log\left(1-\frac{1}{|z|^2}\right)\log\left(1-\frac{1}{|w|^2}\right).
\end{align*}
Si ha allora
$$-\log{\left|\frac{F(z)-F(w)}{z-w}\right|} \le \left|\log{\frac{F(z)-F(w)}{z-w}}\right| \le \sqrt{\log\left(1-\frac{1}{|z|^2}\right)\log\left(1-\frac{1}{|w|^2}\right)},$$
da cui
$$\left|\frac{F(z)-F(w)}{z-w}\right| \ge \exp\left(-\sqrt{\log\left(1-\frac{1}{|z|^2}\right)\log\left(1-\frac{1}{|w|^2}\right)}\right).$$
Questo ci dà un'espressione per $N(z,w)$ non proprio bella da vedere, ma che per adesso è il miglior risultato generale ottenuto: \\
$N(z,w)=\exp\left(-\sqrt{\log\left(1-|z|^2\right)\log\left(1-|w|^2\right)}\right)$. Volendo la si può sostituire con un'espressione leggermente peggiore come stima, ma più semplice da maneggiare e senza palesi aspetti negativi per la disuguaglianza che non fossero già presenti: $\sqrt{(1-|z|^2)(1-|w|^2)}$. Il problema principale di queste due espressioni per $N(z,w)$ è che, fissato $w$, tendono a $0$ per $|z| \longrightarrow 1$, cosa che non accade con $1-|zw|$. Entrambe ci permettono comunque di dedurre che la minorazione in \eqref{congettura} è vera per $|z|=|w|$. \\

Altre cose da fare: citare la "generalizzazione" teorema 3 di IV.2 in Golusin; il suo articolo per poter dire arg(z)=arg(w) ok; la cosa di Gronwall per le F con coefficienti>=0; la "dimostrazione" (speriamo giusta) per le f convesse (cioè con immagine convessa) (anche se F è convessa va bene, si veda Golusin pag 134 espressione (15)).

Nella prossima sezione: controesempio (?) in Sigma, teorema 2 di IV.3 in Golusin.
