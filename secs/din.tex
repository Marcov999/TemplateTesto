Vogliamo adesso cercare di studiare qual è la "dinamica" delle funzioni olomorfe. Lo faremo nei casi del disco e del semipiano superiore.

\begin{prop}
  Sia $\gamma \in \text{Aut}(\mathbb{D}), \gamma \not=\id_{\mathbb{D}}$. Allora o
  \begin{nlist}
    \item $\gamma$ ha un unico punto fisso in $\mathbb{D}$ (si parla in questo caso di automorfismo \textit{ellittico}) o
    \item $\gamma$ non ha punti fissi in $\mathbb{D}$ e ha un unico punto fisso in $\partial\mathbb{D}$ (\textit{parabolico}) o
    \item $\gamma$ non ha punti fissi in $\mathbb{D}$ e ha due punti fissi distinti in $\partial\mathbb{D}$ (\textit{iperbolico}).
  \end{nlist}
\end{prop}

\begin{proof}
  $\gamma(z_0)=z_0 \iff e^{i\theta}(z_0-a)=(1-\bar{a}z_0)z_0 \iff \bar{a}z_0^2+(e^{i\theta}-1)z_0-e^{i\theta}a=0$, equazione di secondo grado con radici $z_1, z_2$ (può essere che $z_1=z_2$) t.c.
  $z_1 \cdot z_2=-e^{i\theta}\dfrac{a}{\bar{a}} \in \partial\mathbb{D} \implies |z_1||z_2|=1$.
  Se $z_1 \not=z_2$, o $z_1 \in \mathbb{D}$ e $z_2 \in \mathbb{C} \setminus \{\overline{\mathbb{D}}\}$ (caso ellittico) e $z_1, z_2 \in \partial\mathbb{D}$ (caso iperbolico). Se $z_1=z_2$, $|z_1|=|z_2|=1$ (caso iperbolico).
\end{proof}

\begin{oss}
  Se $f \in \text{Hol}(\mathbb{D}, \mathbb{D})$ è t.c. $f(z_1)=z_1$ e $f(z_2)=z_2$ con $z_1, z_2 \in \mathbb{D}, z_1 \not=z_2$, allora $f=\id_{\mathbb{D}}$.
  Infatti, possiamo supporre $z_1=0$ $\implies$ $f(0)=0$ e $f(z_2)=z_2$, quindi siamo nel caso del lemma di Schwarza in cui vale l'uguaglianza, per cui $f(z)=e^{i\theta}z$, ma $f(z_2)=z_2$ $\implies$ $e^{i\theta}=1$.
\end{oss}

\begin{ex}
  Esempio di automorfismo ellittico: la rotazione intorno a $0$ $\gamma_{0, \theta}(z)=e^{i\theta}z$. Più in generale, se $a \in \mathbb{D}$, $\gamma_{a, 0}(z)=\dfrac{z-a}{1-\bar{a}z}$, allora $\gamma_{a, 0}^{-1} \circ \gamma_{0, \theta} \circ \gamma_{a, 0}$ è ellittico con punto fisso $a$.
  Queste sono dette \textit{rotazioni non euclidee} e caratterizzano tutti gli automorfismi ellittici (lo si può vedere coniugando opportunamente con $\gamma_{a, 0}$ o $\gamma_{a, 0}^{-1}$).
\end{ex}

\begin{defn}
  Il \textsc{semipiano superiore} è $\mathbb{H}^+=\{w \in \mathbb{C} \mid \mathfrak{Im}w>0\}$. La \textsc{trasformata di Cayley} è $\Psi:\mathbb{D} \longrightarrow \mathbb{H}^+$ t.c. $\Psi(z)=i\dfrac{1+z}{1-z}$.
\end{defn}

Notiamo che possiamo vedere $\mathbb{H}^+ \subset \hat{\mathbb{C}}$ e in questo caso $\partial\mathbb{H}^+=\mathbb{R}\cup\{\infty\}$. $\Psi^{-1}(w)=\dfrac{w-i}{w+i}$. $\Psi(0)=1, \Psi(1)=\infty$. \\
$\mathfrak{Im}\Psi(z)=\mathfrak{Im}\left(i\dfrac{1+z}{1-z}\right)=\mathfrak{Re}\left(\dfrac{1+z}{1-z}\right)=\dfrac{1}{|1-z|^2}\mathfrak{Re}((1+z)(1-\bar{z}))=\dfrac{1-|z|^2}{|1-z|^2}$ che è $>0$ $\iff$ $z \in \mathbb{D}$ e $=0$ $\iff$ $z \in \partial\mathbb{D}\setminus\{1\}$.

$\Psi$ è un biolomorfismo fra $\mathbb{D}$ e $\mathbb{H}^+$ che si estende continua a $\partial\mathbb{D} \longrightarrow \partial\mathbb{H}^+$. Se abbiamo $f: \mathbb{D} \longrightarrow \mathbb{D}$, abbiamo anche $F=\Psi \circ f \circ \Psi^{-1}:\mathbb{H}^+ \longrightarrow \mathbb{H}^+$ e viceversa.

\begin{cor}
  $\gamma \in \text{Aut}(\mathbb{H}^+) \iff \gamma(w)=\dfrac{aw+b}{cw+s}$ con $ad-bc=1$ e $a, b, c, d \in \mathbb{R}$. Si ha allora che $\text{Aut}(\mathbb{H}^+) \cong \faktor{SL(2, \mathbb{R})}{\{\pm I_2\}}=PSL(2, \mathbb{R})$ (questo è detto \textit{gruppo speciale lineare proiettivo}).
\end{cor}

\begin{proof}
  $\gamma \in \text{Aut}(\mathbb{H}^+) \iff \Psi^{-1} \circ \gamma \circ \Psi \in \text{Aut}(\mathbb{D}) \iff (\Psi^{-1} \circ \gamma \circ \Psi)(z)=e^{i\theta}\dfrac{z-a}{1-\bar{a}z}$.
  Ponendo $\Psi(z)=w$, l'uguaglianza sopra equivale a $\gamma(w)=\Psi\left(e^{i\theta}\dfrac{z-a}{1-\bar{a}z}\right)=\Psi\left(e^{i\theta}\dfrac{\Psi^{-1}(w)-a}{1-\bar{a}\Psi^{-1}(w)}\right)$. Facendo il conto si trova l'enunciato.
\end{proof}

\begin{exc}
  $\gamma \in \text{Aut}(\mathbb{H}^+)$ è t.c. $\gamma(i)=i \iff \gamma(w)=\dfrac{w\cos{\theta}-\sin{\theta}}{w\sin{\theta}+\cos{\theta}}$.
\end{exc}

\begin{ex}
  Sia $\gamma \in \text{Aut}(\mathbb{H}^+)$, $\gamma(\infty)=\infty \iff \gamma(w)=\alpha w+\beta$ con $\alpha, \beta \in \mathbb{R}, \alpha>0$.
  Se lo vogliamo parabolico non deve avere altri punti fissi in $\mathbb{C}$ e questo è possibile se e solo se $\alpha w+\beta=w$ non ha altre soluzioni $\iff$ $\alpha=1, \beta \not=0$, cioè $\gamma(w)=w+\beta$.
  È una traslazione di $\mathbb{H}^+$ parallela al suo bordo.
\end{ex}

\begin{exc}
  Sia $\tau \in \partial\mathbb{D}$. Dimostrare che tutti gli automorfismi $\gamma$ parabolici di $\mathbb{D}$ con $\gamma(\tau)=\tau$ sono della forma $\gamma(z)=\sigma_0\dfrac{z+z_0}{1+\bar{z}_0z}$ con $z_0=\dfrac{ic}{2-ic}\tau$ e $\sigma_0=\dfrac{2-ic}{2+ic}$ con $c \in \mathbb{R}$.
  Hint: a meno di una rotazione, $\tau=1$.
\end{exc}

\begin{ex}
  $\gamma \in \text{Aut}(\mathbb{H}^+)$ è iperbolico con $\gamma(\infty)=\infty$ e $\gamma(0)=0$ $\iff$ $\gamma(w)=\alpha w$ con $\alpha>0$.
\end{ex}

Passiamo ora alla \textsc{dinamica di funzioni iterate}. Abbiamo uno spazio generico $X$ e una funzione $f:X \longrightarrow X$. Le sue \textit{iterate} sono $f^2=f \circ f$ e, induttivamente, $f^{k+1}=f \circ f^{k-1}$. Vogliamo capire il comportamento asintotico di $\{f^k\}$ (in relazione alla struttura presente su $X$), per esempio, capire cosa succede all'\textit{orbita} $O^+(x)=\{f^k(x)\}$ con $x \in X$.

\begin{ex}
  $\gamma(w)=\alpha w \implies \gamma^2(w)=\alpha(\alpha w)=\alpha^2 w \implies \gamma^k(w)=\alpha^k w$.
  Quindi: se $0<\alpha<1$, $\gamma^k(w) \longrightarrow 0$ per $k \longrightarrow +\infty$ $\implies$ $\gamma^k \longrightarrow 0$ (costante) uniformemente sui compatti; se $\alpha>1$, $\gamma^k(w) \longrightarrow \infty$ per $k \longrightarrow +\infty$ $\implies$ $\gamma^k \longrightarrow \infty$ (costante) uniformemente sui compatti.
\end{ex}

\begin{ex}
  $\gamma(w)=w+\beta \implies \gamma^k(w)=w+k\beta \implies \gamma^k(w) \longrightarrow \infty$ per $k \longrightarrow +\infty$ $\implies$ $\gamma^k \longrightarrow \infty$ (costante) uniformemente sui compatti.
\end{ex}

\begin{oss}
  \begin{center}
    \begin{tikzcd}
      X \arrow[r, "f"] & X\\
      Y \arrow[u, "\Psi", "\cong" right] \arrow[r, "F"] & Y \arrow[u, "\cong", "\Psi" right]
    \end{tikzcd}
  \end{center}
  $\Psi$ bigezione (omeomorfismo/biolomorfismo/eccetera), $F=\Psi^{-1} \circ f \circ \Psi$, cioè \textit{$F$ è coniugata a $f$}. Allora $f^k=\Psi^{-1} \circ f^k \circ \Psi$, cioè $F^k$ è coniugata a $f^k$ per ogni $k$. In particolare, la "dinamica di $F$" è "uguale" alla "dinamica di $f$".
\end{oss}

\begin{cor}
  Sia $\gamma \in \text{Aut}(\mathbb{D})$ parabolico o iperbolico, allora $\gamma^k$ converge uniformemente sui compatti a una funzione costantemente uguale a un punto fisso di $\gamma$ sul bordo.
\end{cor}

\begin{proof}
  A meno di coniugio possiamo supporre $\text{Fix}(\gamma)=\{1\}$ nel caso parabolico e $\{1, -1\}$ nel caso iperbolico. Coniughiamo con $\Psi$ e usiamo gli esempi.
\end{proof}

Sia $\gamma \in \text{Aut}(\mathbb{D})$ ellittico, a meno di coniugio $\gamma(0)=0 \implies \gamma(z)=e^{2\pi i \theta}z \implies \gamma^k(z)=e^{2k\pi i \theta}z$.
Se $\theta \in \mathbb{Q}$, esiste $k_0$ t.c. $k_0\theta \in \mathbb{Z} \implies \gamma^{k_0}(z) \equiv z \iff \gamma^{k_0}=\id_{\mathbb{D}}$.

\begin{exc}
  Se $\theta \not\in \mathbb{Q}$, $\gamma^k(z) \not=z$ per ogni $z \not=0$ e $k \in \mathbb{N}^*$, da cui si ha anche che $\gamma^k(z)\not=\gamma^h(z)$ per ogni $z \not=0$ e $h \not=k$.
\end{exc}

\begin{exc}
  Se $\theta \not\in \mathbb{Q}$, $\{\gamma^k(z_0) \mid k \in \mathbb{N}\}$ è densa nella circonferenza $\{|z|=|z_0|\}$.
\end{exc}

Vogliamo adesso studiare la dinamica di una $f \in \text{Hol}(\mathbb{D}, \mathbb{D})$ qualunque.

\begin{defn}
  Sia $f \in \text{Hol}(\Omega, \Omega)$, un \textit{punto limite} di $\{f^k\}$ è $g \in \text{Hol}(\Omega, \mathbb{C})$ t.c. è il limite di una sottosuccessione $\{f^{k_{\nu}}\}$, cioè $f^{k_{\nu}} \longrightarrow g$.
\end{defn}

\begin{lm} \label{pli}
  Sia $\Omega \subseteq \mathbb{C}$ dominio, $f \in \text{Hol}(\Omega, \Omega)$. Se $\id_{\Omega}$ è un punto limite di $\{f^k\}$, allora $f \in \text{Aut}(\Omega)$.
\end{lm}

\begin{proof}
  $f^{k_{\nu}} \longrightarrow \id_{\Omega}$ $\implies$ $f$ è iniettiva (se $z_1 \not=z_2$ sono t.c. $f(z_1)=f(z_2)$, allora $f^{k_{\nu}}(z_1)=f^{k_{\nu}}(z_2)$, ma la prima tende a $z_1$ e la seconda a $z_2$, che sono diversi, assurdo).
  Se $z_0 \in \Omega$, $z_0=\id_{\Omega}(z_0)$. Per il primo teorema di Hurwitz, $\id_{\Omega}(z_0) \in f^{k_{\nu}}(\Omega)$ per $\nu \gg 1$, ma $f^{k_{\nu}}(\Omega) \subseteq f(\Omega)$ $\implies$ $f$ è suriettiva.
\end{proof}

\begin{prop} \label{lim_aut}
  Sia $\Omega \subset \subset \mathbb{C}$ un dominio limitato, $f \in \text{Hol}(\Omega, \Omega)$. Sia $h \in \text{Hol}(\Omega, \mathbb{C})$ un punto limite di $\{f^k\}$ (che esiste per il teorema di Montel). Allora o
  \begin{nlist}
    \item $h \equiv c \in \overline{\Omega}$ oppure
    \item $h \in \text{Aut}(\Omega)$ e in questo caso $f \in \text{Aut}(\Omega)$.
  \end{nlist}
\end{prop}

\begin{proof}
  Sia $\displaystyle h=\lim_{\nu \longrightarrow +\infty} f^{k_{\nu}}$. Poniamo $m_{\nu}=k_{\nu+1}-k_{\nu}$. Possiamo supporre $m_{\nu} \longrightarrow +\infty$. Per Montel, a meno di una sottosuccessione possiamo supporre $f^{m_{\nu}} \xrightarrow{\nu \longrightarrow +\infty} g \in \text{Hol}(\Omega, \mathbb{C})$.
  Se $h$ è costante abbiamo finito. Se $h$ non è costante, per il teorema dell'applicazione aperta $h$ è aperta $\implies$ $h(\Omega)$ è aperto e per il primo teorema di Hurwitz è contenuto in $\Omega$.
  Se $z \in \Omega$, $\displaystyle g(h(z))=\lim_{\nu \longrightarrow +\infty} f^{m_{\nu}}(f^{k_{\nu}}(z))=\lim_{\nu \longrightarrow +\infty} f^{k_{\nu+1}}(z)=h(z) \implies g\restrict{h(\Omega)}=\id_{\Omega}$,
  ma per il principio di identità questo ci dà $g \equiv \id_{\Omega}$, dunque per il lemma \ref{pli} abbiamo che $f \in \text{Aut}(\Omega)$. A meno di sottosuccessioni è facile vedere che $f^{-k_{\nu}}=(f^{-1})^{k_{\nu}}$ converge a $h^{-1}$.
\end{proof}

\begin{prop}
  Sia $f \in \text{Hol}(\mathbb{D}, \mathbb{D})$, $f(z_0)=z_0, z_0 \in \mathbb{D}$, $f\not\in \text{Aut}(\mathbb{D})$. Allora $f^k \longrightarrow z_0$ (costante) uniformemente sui compatti.
\end{prop}

\begin{proof}
  A meno di coniugio possiamo supporre $z_0=0$. Per il lemma di Schwarz, $|f(z)|<|z|$ per ogni $z \in \mathbb{D}\setminus\{0\}$. Fissiamo $0<r<1$. In $\overline{\mathbb{D}}_r$, $\left|\dfrac{f(z)}{z}\right|$ ha un massimo $\lambda_r<1$.
  Per ogni $z \in \overline{\mathbb{D}}_r$, $|f(z)| \le \lambda_r|z| \implies |f^2(z)| \le \lambda_r|f(z)| \le \lambda_r^2|z| \implies |f^k(z)| \le \lambda_r^k|z| \le \lambda_r^kr \longrightarrow 0$ per $k \longrightarrow +\infty$ $\implies$ $f^k \longrightarrow 0$ (costante) uniformemente sui compatti.
\end{proof}

\begin{defn}
  Chiamiamo \textit{orociclo} di centro $\tau \in \partial\mathbb{D}$ e raggio $R>0$ l'insieme $E(\tau, R)=\left\{z \in \mathbb{D} \, \bigg| \, \dfrac{|\tau-z|^2}{1-|z|^2}<R \right\}$. Geometricamente, è un disco di raggio $\dfrac{R}{R+1}$ tangente a $\partial \mathbb{D}$ in $\tau$.
\end{defn}

\begin{exc}
  $\displaystyle E(\tau, R)=\left\{z \in \mathbb{D} \, \bigg| \, \lim_{w \longrightarrow \tau} [\omega(z, w)-\omega(0, w)]<\frac{1}{2}\log{R}\right\}$.
\end{exc}

\begin{lm}
  (Lemma di Wolff) Sia $f \in \text{Hol}(\mathbb{D}, \mathbb{D})$ senza punti fissi. Allora esiste un unico $\tau \in \partial\mathbb{D}$ t.c. per ogni $z \in \mathbb{D}$ $\dfrac{|\tau-f(z)|^2}{1-|f(z)|^2} \le \dfrac{|\tau-z|^2}{1-|z|^2}$ $(\star)$.
  In altre parole, per ogni $R>0$ $f(E(\tau, R)) \subseteq E(\tau, R)$.
\end{lm}

\begin{proof}
  Unicità: se ce ne fossero due, $\tau$ e $\tau_1$, prendiamo un orociclo centrato in $\tau$ e uno centrato in $\tau_1$ tangenti, allora il punto di tangenza verrebbe mandato in sé e sarebbe dunque un punto fisso in $\mathbb{D}$, assurdo.

  Esistenza: prendiamo $\{r_{\nu}\} \subset (0, 1)$ t.c. $r_{\nu} \nearrow 1^{-}$ e poniamo $f_{\nu}=r_{\nu}f$ $\implies$ $f_{\nu}(\mathbb{D}) \subseteq \mathbb{D}_{r_{\nu}} \subset\subset \mathbb{D}$,
  allora per il teorema di Ritt esiste $w_{\nu} \in \mathbb{D}$ t.c. $f_{\nu}(w_{\nu})=w_{\nu}$. A meno di sottosuccessioni, possiamo suppore $w_{\nu} \longrightarrow \in \overline{\mathbb{D}}$.
  Se $\tau \in \mathbb{D}$, $\displaystyle f(\tau)=\lim_{\nu \longrightarrow +\infty} f_{\nu}(w_{\nu})=\lim_{\nu \longrightarrow +\infty} w_{\nu}=\tau$, assurdo $\implies$ $\tau \in \partial\mathbb{D}$.
  Per Schwarz-Pick, $\left|\dfrac{f_{\nu}(z)-w_{\nu}}{1-\bar{w}_{\nu}f_{\nu}(z)}\right|^2 \le \left|\dfrac{z-w_{\nu}}{1-\bar{w}_{\nu}z}\right|^2 \implies 1-\left|\dfrac{f_{\nu}(z)-w_{\nu}}{1-\bar{w}_{\nu}f_{\nu}(z)}\right|^2 \ge 1-\left|\dfrac{z-w_{\nu}}{1-\bar{w}_{\nu}z}\right|^2 \implies \dfrac{|1-\bar{w}_{\nu}f_{\nu}(z)|^2}{1-|f_{\nu}(z)|^2} \le \dfrac{|1-\bar{w}_{\nu}z|^2}{1-|z|^2}$.
  Mandando $\nu \longrightarrow +\infty$ otteniamo $\dfrac{|1-\bar{\tau}f(z)|^2}{1-|f(z)|^2} \le \dfrac{|1-\bar{\tau}z|^2}{1-|z|^2}$ che moltiplicando per $\tau$ ($\tau\bar{\tau}=1$) dà la tesi.
\end{proof}

\begin{exc}
  Si ha l'uguaglianza in $(\star)$ nel lemma di Wolff $\iff$ $f$ è un automorfismo parabolico con punto fisso $\tau$ $\iff$ vale l'uguaglianza in $(\star)$ per ogni $z \in \mathbb{D}$.
\end{exc}

\begin{thm}
  (Wolff-Denjoy) Sia $f \in \text{Hol}(\mathbb{D}, \mathbb{D})$ senza punti fissi in $\mathbb{D}$. Allora esiste un unico $\tau \in \partial\mathbb{D}$ t.c. $f^k \longrightarrow \tau$ (costante) uniformemente sui compatti.
\end{thm}

\begin{proof}
  Se $f \in \text{Aut}(\mathbb{D})$ parabolico o iperbolico l'abbiamo già visto. Supponiamo $f \not\in \text{Aut}(\mathbb{D})$. Per Montel, $\{f^k\}$ è relativamente compatta in $\text{Hol}(\mathbb{D}, \mathbb{C})$. Useremo il seguente risultato di topologia che viene lasciato come esercizio.

  \begin{exc} \label{sct}
    Sia $X$ spazio topologico di Hausdorff. Sia $\{x_k\} \subset X$ con $\overline{\{x_k\}}$ compatta in $X$. Supponiamo che esista un unico $\bar{x} \in X$ t.c. ogni sottosuccessione convergente di $\{x_k\}$ converge a $\bar{x}$. Allora $x_k \longrightarrow \bar{x}$.
  \end{exc}

  Sia $\tau \in \partial\mathbb{D}$ dato dal lemma di Wolff. Sia $\displaystyle h=\lim_{\nu \longrightarrow +\infty} f^{k_{\nu}}$ un punto limite di $\{f^k\}$ (che esiste per Montel). Per la proposizione \ref{lim_aut}, $h \equiv \sigma \in \overline{\mathbb{D}}$.
  Se $\sigma \in \mathbb{D}$, $\displaystyle f(\sigma)=\lim_{\nu \longrightarrow +\infty} f(f^{k_{\nu}}(\sigma))=\lim_{\nu \longrightarrow +\infty} f^{k_{\nu}}(f(\sigma))=\sigma$, assurdo. Quindi $h \equiv \sigma \in \partial\mathbb{D}$.
  Vogliamo $\sigma=\tau$. Per il lemma di Wolff $f^{k_{\nu}}(E(\tau, R)) \subseteq E(\tau, R)$ per ogni $R>0$ $\implies$ $\{\sigma\}=h(E(\tau, R)) \subseteq \overline{E(\tau, R)} \cap \partial\mathbb{D}=\{\tau\}$ $\implies$ $\sigma=\tau$.
  Si conclude allora per l'esercizio \ref{sct}.
\end{proof}
